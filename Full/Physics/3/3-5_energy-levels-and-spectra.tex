\subsection{Energy Levels and Spectra}

A \textbf{prism} can be used to split a beam of white light from a filament lamp into a \textbf{continuous spectrum}. If we replace the light source with a tube of glowing gas, we see a \textbf{spectrum of discrete lines} of different colours instead of a continuous spectrum.
\begin{itemize}
    \item The wavelengths of the lines of a line spectrum of an element are \underline{characteristic of the atoms} of that element. By measuring the wavelengths of a line spectrum, we can therefore \textbf{identify the element} that produced the light.
\end{itemize}

No two elements produce the same pattern of light wavelengths, because the \textbf{energy levels} of each type of atom are unique to that atom, so photons emitted are characteristic of the atom.

\begin{enumerate}
    \item Each line in a line spectrum is due to light of a certain colour and therefore a \textbf{certain wavelength}.
    \item The photons that produce each line all have the \textbf{same energy}, which is different from the energy of the photon that produce any other line.
    \item Each photon is emitted when an atom \textbf{de-excites} due to one of its electrons moving to an inner shell.
\end{enumerate}
$$\text{energy of the emitted photon}=E_1-E_2$$
Given the energy level diagram for the atom, we can identify on the diagram the \textbf{transition that caused a photon of that wavelength} to be emitted.

\subsubsection*{Bohr's Atom}

The energy levels of the hydrogen atom, \textbf{relative to the ionisation level} are given by formula.
$$E=-\frac{13.6\text{eV}}{n^2}$$
where $n=1$ for the ground state.

When a hydrogen atom de-excites from energy level $n_1$ to $n_2$, the energy of the \textbf{emitted photon} is given by.
$$E=\left(\frac{1}{{n_2}^2}-\frac{1}{{n_1}^2}\right)\times13.6\text{eV}$$
