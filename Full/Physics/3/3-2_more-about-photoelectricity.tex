\subsection{More about Photoelectricity}

The energy of each vibrating atom is \textbf{quantised} - only certain levels of the energy are allowed, and energy could only be in multiples of a basic amount, or \textbf{quantum}.
\begin{itemize}
    \item If a conduction electron absorbs a photon but does not leave the metal, it \textbf{collides repeatedly} with other electrons and positive ions, quickly loses its extra kinetic energy.
\end{itemize}

\subsubsection*{The Vacuum Photocell}

A vacuum photocell is a glass tube that contains
\begin{itemize}
    \item A metal plate, referred to as the \textbf{photocathode}.
    \item A smaller metal electrode, referred to as the \textbf{anode}.
\end{itemize}

When light of a frequency \underline{greater than the threshold frequency} for the metal is directed at the photocathode, electrons are emitted from the cathode and are attracted to the anode. A microammeter can be used to measure the \textbf{photoelectric current} proportional to the number of electrons per second that transfer from the cathode to the anode.
\begin{itemize}
    \item The number of \textbf{photoelectrons} transferred per second is given by $n=I/e$.
    \item The photoelectric current is \textbf{proportional to the intensity} of the light incident on the cathode - each electron must have absorbed one photon to escape from the metal surface.
        \begin{itemize}
            \item \textbf{Light intensity} is a measure of energy per second carried by the incident light.
            \item Which is proportional to the number of photons per second incident on the cathode.
        \end{itemize}
    \item The intensity of the incident light does not affect the \textbf{maximum kinetic energy} of a photoelectron.
        \begin{itemize}
            \item No matter how intense the incident light is, the energy gained by a photoelectron is \underline{due to the absorption of one photon only}.
        \end{itemize}
    \item The \textbf{maximum kinetic energy} of the photoelectrons emitted for a given frequency of light can be measured using a photocell.
    \item A graph of $E_\text{Kmax}$ against $f$
        \begin{itemize}
            \item The gradient is $h$
            \item The x-intercept is the \textbf{threshold frequency}.
            \item The y-intercept is the \textbf{work function}.
        \end{itemize}
\end{itemize}
