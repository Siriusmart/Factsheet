\subsection{Energy Levels in Atoms}

\textbf{Atomic electrons} are trapped by the \textbf{electrostatic force of attraction} of the nucleus.

They move about the nucleus in allowed orbits, or \textbf{shells} surrounding the nucleus.
\begin{itemize}
    \item The energy of an electron in a shell is \underline{constant}.
    \item An electron in a shell near the nucleus has less energy than an electron in a shell further away.
    \item Each shell can only hold a certain number of electrons.
\end{itemize}

The lowest energy state of an atom is called its \textbf{ground state}. When an atom in ground state absorbs energy, one of its electrons \underline{moves to a shell at higher energy}, the atom is now in an \textbf{excited state}.

An \textbf{energy level diagram} for the atom can be constructed using \textbf{excitation energy measurements}. The diagram shows the allowed energy values of the atom - each correspondings to a certain \textbf{electron configuration} in the atom.

\textbf{Ionisation energy} is used as the \textbf{zero reference level} for energy. Energy levels below the ionisation level would need to be shown as negative values.

\subsubsection*{De-excitation}

Gases at low pressure emit light when they are made to conduct electricity. The atoms absorb energy as a result of excitation by collision, but they do not retain the absorbed energy permanently.

\begin{itemize}
    \item The \textbf{electron configuration} in an excited atom is unstable because an electron that moves to an outer shell leaves a vacancy in the shell it moves from.
    \item Sooner or later, the vacancy is filled by an electron from an outer shell transferring to it.
\end{itemize}

When this happens, the electron emits a photon, the \underline{atom} therefore moves to a lower energy level in the process of \textbf{de-excitation}. The energy of the photon is equal to the energy lost by the electron and therefore the atom.
$$\text{Energy of the emitted photon}\,E=hf$$

\subsubsection*{Excitation by Photons}
\begin{itemize}
    \item An electron in an atom can absorb a photon and move to an outer shell \underline{where a vacancy exists}.
    \item But only if the energy of the photon is \underline{exactly equal to the gain} in the electron's energy.
        \begin{itemize}
            \item The photon energy must be exactly equal to the difference between the final and initial energy levels of the atom.
            \item If the photon's energy is smaller or larger than the difference between two energy levels, it will not be absorbed by the electron.
        \end{itemize}
\end{itemize}

\subsubsection*{Fluorescence}
An atom in an excited state can \underline{de-excite directly or indirectly} to the ground state, regardless of how the excitation took place - an atom can absorb photons of certain energies and emit photons of \underline{same or lesser energies}.

Some substances \textbf{fluoresce} (glow with visible light) when they absorb ultraviolet radiation.
\begin{itemize}
    \item Atoms in the substance \textbf{absorb ultraviolet photons} and become excited.
    \item When the atoms de-excite, they \textbf{emit visible photons}.
    \item When the source of ultraviolet radiation is removed, the substance \textbf{stops glowing}.
\end{itemize}

A \textbf{fluorescent tube} is a glass tube with \textbf{fluorescent coating} on its inner surface, containing \textbf{mercury vapour at low pressure}. It emits visible light when turned on.
\begin{enumerate}
    \item \textbf{Ionisation and excitation} of the mercury atoms occur as they collide with each other and with electrons in the tube.
    \item The mercury atoms \textbf{emit ultraviolet photons}, as well as photons of less energy when they de-excite.
    \item The ultraviolet photons are absorbed by the atoms of the fluorescent coating, causing \textbf{excitation of the atoms}.
    \item The coating atoms \textbf{de-excite in steps} and emit visible photons.
\end{enumerate}
A fluorescent tube much more efficient than a \textbf{filament lamp} - it can produce the same light output with less power wasted as heat.
