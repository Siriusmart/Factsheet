\subsection{The Photoelectric Effect}

\textbf{Conduction electrons} move about freely inside the metal. It was found that sparks produced by a \textbf{spark detector} when ultraviolet radiation was directed at the spark gap.

Further investigation showed that electrons are emitted from the surface of a metal when electromagnetic radiation \underline{above a certain frequency} was direct at the metal. This is known as the \textbf{photoelectric effect}.

\begin{itemize}
    \item Photoelectric emission of electron from a metal surface does not take place if the frequency of the incident electromagnetic radiation is below a certain value known as the \textbf{threshold frequency}.
    \item The number of electron emitted per second is \underline{proportional to the intensity} of the incident radiation, provided the frequency is greater than the threshold frequency.
    \item Photoelectric emissions \textbf{occur without delay} as soon as the incident radiation is directed at the surface. Provided the frequency exceeds the threshold frequency, and reguardless of the intensity.
\end{itemize}

Observations from the photoelectric effect could not be explained using the \textbf{wave theory} of light.
\begin{itemize}
    \item The \textbf{existence of a threshold frequency} - each conduction electron at the surface of a metal should gain some energy from the incoming waves.
    \item Why photoelectric emission \textbf{occurs without delay}.
\end{itemize}

\subsubsection*{The Photon Model of Light}

The photon theory of light was put forward to explain the photoelectric effect.
\begin{itemize}
    \item Light is composed of \textbf{wavepackets} or photons.
    \item Each photon has energy $E=hf=hc/\lambda$.
\end{itemize}

To explain the photoelectric effect
\begin{itemize}
    \item When light is incident on a metal surface, an electron at the surface \textbf{absorbs a single photon} from the incident light and therefore gains energy equals to $hf$ - the energy of a photon.
    \item An electron can leave the metal surface if the \underline{energy gained from a single photon} exceeds the \textbf{work function} $\phi$ of the metal.
        \begin{itemize}
            \item The \textbf{work function} is the minimum energy needed by an electron to escape from the metal surface.
            \item Excess energy gained by the photoelectron becomes its kinetic energy.
        \end{itemize}
\end{itemize}
The maximum kinetic energy is given by
$$E_\text{Kmax}=hf-\phi$$
Emissions can take place provided $E_\text{Kmax}>0$, so the threshold frequency of the metal is
$$f_\text{min}=\frac{\phi}{h}$$

\subsubsection*{Stopping Potential}

Electrons that escape from the metal plate can be \textbf{attracted back} by giving the plate a sufficient positive charge.
\begin{itemize}
    \item The \underline{minimum potential} needed to stop photoelectric emission is called the \textbf{stopping potential} $V_s$.
    \item At this potential, the maximum kinetic energy of the emitted electron is \underline{reduced to zero} because each electron must do extra work $e\times V_s$ to leave the surface.
\end{itemize}
Conclusive experimental evidence of the photon theory was obtained by
\begin{itemize}
    \item Measuring the stopping potential for a range of metals.
    \item Using light of different frequencies.
\end{itemize}
The results fitting the photoelectric equation very closely.
