\subsection{Wave-particle Duality}

Light is part of the \textbf{electromagnetic waves}. The \textbf{theory of electromagnetic waves} predicted the existence of EM waves beyond the visible spectrum, subsequent discovery of X-rays and radio waves confirmed these predictions.

However, the \textbf{photoelectric effect} was not explained until the \textbf{photon theory of light}, where light consists of photons, which are particle-like packets of electromagnetic waves.

Light has a \textbf{dual-nature}, it can behave as a wave or as a particle, according to circumstances.
\begin{itemize}
    \item The \textbf{wave-like nature} is observed when \textbf{diffraction} of light takes place.
        \begin{enumerate}
            \item Light passes through a \textbf{narrow slit}.
            \item Light emerging from the slit \textbf{spreads out} in the same way water waves spreads out after passing through a gap.
            \item The \underline{narrower the gap} or the \underline{longer the wavelength}, the greater the diffraction.
        \end{enumerate}

    \item The \textbf{particle-like nature} is observed in the \textbf{photoelectric effect}.
\end{itemize}

\subsubsection*{Matter Waves}
\textbf{de Broglie's hypothesis} is the idea that matter particles also have a wave-like nature. Extending the idea of duality from photons to \textbf{matter particles}.
\begin{itemize}
    \item Matter particles have \textbf{dual wave-particle nature}.
    \item The wave-like behaviour of a matter particle is \underline{characterised by a wavelength} - its \textbf{de Broglie wavelength}.
\end{itemize}
$$\lambda=\frac{h}{p}$$

\subsection*{Electron Diffraction}

The wave-like nature of electrons was discovered when it was demonstrated that a \textbf{beam of electrons} can be diffracted. \underline{Further experimental evidence} using other types of particles confirmed the correctness of the theory.

\begin{enumerate}
    \item A narrow \textbf{beam of electrons} in a \textbf{vacuum tube} is directed at a thin \textbf{metal foil}.
        \begin{itemize}
            \item A metal is composed of many tiny \textbf{crystalline regions}.
            \item Each region consists of \textbf{positive ions} arranged in rows in a \textbf{regular pattern}.
            \item The rows of the atoms causes the electrons in the beam to be \textbf{diffracted} - just a beam of light is diffracted when it passes through a slit.
        \end{itemize}
    \item The electrons in the beam pass through the meal foil and are \textbf{diffracted in certain directions only}. Forming a pattern of \textbf{concentric rings} on a fluorescent screen at the end of the tube.
        \begin{itemize}
            \item Each ring is due to electrons \textbf{diffracted by the same angle to the incident beam} from regions of different orientations.
        \end{itemize}
\end{enumerate}

The beam of electrons is produced by \underline{attracting electrons from a heated filament wire} to a positively charged metal plate with a \underline{small hole in the centre}. Electrons that passes through the hole \textbf{forms the beam}.
\begin{itemize}
    \item The \textbf{speed of electrons} can be increased by \textbf{increasing the potential difference} between the filament and the metal plate.
    \item This makes the \underline{diffraction rings smaller} - increase of speed decreases the \textbf{de Broglie wavelength}, so \underline{less diffraction} occur and the rings become smaller.
\end{itemize}

\subsubsection*{Electron Energy Levels}

An electron in an atom has a \textbf{fixed amount of energy} depending on the shell it occupies.
\begin{itemize}
    \item Its \textbf{de Broglie wavelength} has to \underline{fit the shape and size} of the shell.
    \item That's why energy depends on the shell an electron occupies.
\end{itemize}
The \textbf{circumference} of its orbit must be equal to a \underline{whole number of de Broglie wavelengths}.
