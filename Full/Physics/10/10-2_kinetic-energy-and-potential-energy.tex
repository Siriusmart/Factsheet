\subsection{Kinetic Energy and Potential Energy}

\subsubsection*{Kinetic Energy}

\textbf{Kinetic energy} is the energy of an object \textbf{due to its motion} - the faster an object moves, the more kinetic energy it has.

Consider an object of mass m initially at rest, acted on by a constant force $F$ for a time $t$.
\begin{align*}
    \text{distance travelled}\ s&=\frac{1}{2}vt\qquad\text{where $u=0$}\\
    \text{acceleration}\ a&=\frac{v}{t}\\
    F&=ma=\frac{mv}{t}
\end{align*}

Therefore the work done $W$ by a force $F$ to move the object through a distance $s$ is
$$W=Fs=\frac{mv}{t}\times\frac{vt}{2}=\frac{1}{2}mv^2$$

Because the gain of kinetic energy is due to the work done
$$E_K=\frac{1}{2}mv^2$$

\subsubsection*{Potential Energy}

\textbf{Potential energy} is the energy of an object due to its position.

If an object of mass $m$ is raised through a vertical height $\Delta h$ at steady speed, the force needed to raise it is equal and opposite to its weight $mg$.

\begin{align*}
    \text{Work done}&=Fd\\
                    &=mg\Delta h
\end{align*}

The work done on the object increases its \textbf{gravitational potential energy}.
$$\text{Change in gravitational potential energy}\ \Delta E_p=mg\Delta h$$
