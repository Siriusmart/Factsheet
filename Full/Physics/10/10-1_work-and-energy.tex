\subsection{Work and Energy}

Energy is needed to make stationary objects move, to change their shape or to warm them up.

Objects can possess energy in different types of stores.
\begin{itemize}
    \item \textbf{Gravitational potential stores} - position of objects in a gravitational field.
    \item \textbf{Kinetic stores} - moving objects.
    \item \textbf{Thermal stores} - hot objects.
    \item \textbf{Elastic stores} - objects compressed or stretched.
\end{itemize}

Energy can be transferred between objects
\begin{itemize}
    \item By \textbf{radiation}.
    \item \textbf{Electrically}.
    \item \textbf{Mechanically}.
        \begin{itemize}
            \item E.g. by sound.
        \end{itemize}
\end{itemize}

Energy is measured in \textbf{joules} - one joule is equal to the energy needed to raise 1N weight through a vertical height of 1m.

Whenever energy is transferred, the total amount of energy is unchanged, this is known as the \textbf{principle of conservation of energy}.

Energy cannot be created and destroyed.

\subsubsection*{Work Done by Force}

\textbf{Work} is done on an object when a force acting on it makes it move, as a result energy is transferred to the object.
$$\text{Work done}=\text{force}\times\text{distance moved in the direction of the force}$$

The unit of work is the joule - equal to the work done when a force of 1N moves its point of application by a distance of 1m in the direction of the force.

\subsubsection*{Force-distance Graphs}
Force-distance graph is a graph of force against distance.

\begin{itemize}
    \item If a \textbf{constant force} $F$ acts on an object makes it move a distance $s$ in the direction of the force, the work done on the object is $W=Fs$. The area under the lien is a rectangle with the same area.

        Therefore the area under the line represents the work done.
    \item If a \textbf{variable force} $F$ acts on an object and causes it to move in the direction of the force, the work done for a small amount of distance $\Delta s$, $\Delta W=F\Delta s$.

        The total work done is therefore the \textbf{sum of the areas} under the line.
\end{itemize}

In both case, the area under the line of a force-distance graph represents the \textbf{total work done}.
