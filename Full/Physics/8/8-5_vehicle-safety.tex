\subsection{Vehicle Safety}

The \textbf{effect of a collision} of a vehicle can be measured in terms of the acceleration or deceleration of the vehicle, expressed in terms of $g$.

When objects collide, they are in contact with each other force a certain time.
\begin{itemize}
    \item The shorter the contact time, the greater the impact force for the same initial velocities of the objects.
    \item If two vehicles collide and remain tangled together, they exert forces on each other \textbf{until they are moving at the same velocity}.
\end{itemize}

The \textbf{impact time} $t$ is the duration of the impact force.
\begin{align*}
    \text{Impact time}\ t&=\frac{2s}{u+v}\\
    \text{Acceleration}\ a&=\frac{v-u}{t}\\
    \text{Impact force}\ F&=ma
\end{align*}

\subsubsection*{Car Safety Features}

The \textbf{impact force} is lessened if the \textbf{impact time is greater}. Design features increase the impact time to reduce the impact force.
\begin{itemize}
    \item \textbf{Vehicle bumpers} give way a little in a \textbf{low-speed impact} and so increase impact time. Impact force is reduced as a result.
    \item \textbf{Crumple zones} - The \textbf{engine compartment} of a car is designed to give way in a front-end impact.
        \begin{itemize}
            \item If the engine compartment were rigid, the impact time would be very short, so the impact force would be very large.
        \end{itemize}
    \item \textbf{Seat belts} restrains the wearer from \textbf{crashing into the vehicle frame} after the vehicle suddenly stops in a front-end impact.
        \begin{itemize}
            \item The restraining force on the wearer is much less than the impact force would be if the wearer hit the vehicle frame.
            \item With the seat belt on, the wearer is \textbf{stopped more gradually} than without it.
        \end{itemize}
    \item \textbf{Collapsible steering wheel} lessen the impact force if the driver makes contact with the steering wheel as a result of the steering wheel collapsing in the impact.
    \item \textbf{Air bags} \underline{acts as a cushion} and increase the impact time on the person, reducing the force on the person.
        \begin{itemize}
            \item The force of the impact is \textbf{spread over the contact area} - so the pressure on the body is less.
        \end{itemize}
\end{itemize}
