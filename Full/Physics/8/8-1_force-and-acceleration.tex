\subsection{Force and Acceleration}

An \textbf{air track} allows motion to be observed in the \underline{absence of friction}.
\begin{itemize}
    \item The \textbf{glider} on the air track floats on a \textbf{cushion of air}.
    \item Provided the track is level, the glider \textbf{moves at constant velocity} along the track because \underline{friction is absent}.
\end{itemize}

\textbf{Newton's first law of motion}: Objects either stay at rest or moves with constant velocity unless acted on by a force.

An object moving at constant velocity is either
\begin{itemize}
    \item Acted on by no forces, or
    \item The forces acting on it are \textbf{balanced}.
\end{itemize}

The inverse is true: when an object is acted on by a resultant force, the result is to change the objects velocity.

\textbf{Newton's second law of motion}: $F$ is proportional to $ma$.

By defining the \textbf{newton} as the amount of force that will give an object of mass 1kg an acceleration of 1ms$^{-2}$, the proportional statement can be written as
$$F=ma$$

\subsubsection*{Weight}

The force of gravity on an object is its \textbf{weight}.

The acceleration of a falling object acted on by gravity only is $g$. Because the force of is the only force acting on it, its weight can be given by
$$W=mg$$

\begin{itemize}
    \item When an object is in \textbf{equilibrium}, the \textbf{support force} on it is equal and opposite to its weight.
    \item An object placed on a \textbf{weighting balance} exerts a force on the balance equal to the weight of the object. Thus the balance measures the weight of the object.
\end{itemize}

The mass of an object is a measure of its \textbf{inertia} - its resistance to change of motion.
\begin{itemize}
    \item More force is needed to give an object a certain acceleration than to give an object with less mass the same acceleration.
\end{itemize}
