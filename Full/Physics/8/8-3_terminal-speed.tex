\subsection{Terminal Speed}

Any object moving through a fluid experiences a force that drags on it due to the fluid.

The \textbf{drag force} depends on
\begin{itemize}
    \item The \textbf{shape} of the object.
        \begin{itemize}
            \item The faster an object travels in a fluid, the greater the drag force on it.
        \end{itemize}
    \item Its \textbf{speed}.
    \item The \textbf{viscosity} of the fluid.
        \begin{itemize}
            \item Viscosity is a measure of how easily the fluid flows past a surface.
        \end{itemize}
\end{itemize}

\subsubsection*{Motion of an Object Falling in a Fluid}

\begin{enumerate}
    \item The speed of an object \textbf{released from rest} in a fluid \textbf{increases as it falls}.
    \item So the \textbf{drag force} on it due to the fluid increases.
    \item The \textbf{resultant force} on the object is the difference between the force of gravity on it and the drag force.
    \item As the drag force increases, the \textbf{resultant force decreases}, so the acceleration becomes less as it falls.
    \item If it continues falling, it attains \textbf{terminal speed} when the drag force on it is equal and opposite to its weight.
\end{enumerate}

At any instance, the resultant force $F=mg-D$ where $D$ is the drag force.

Therefore the acceleration of the object is
$$\frac{mg-D}{m}=g-\frac{D}{m}$$
\begin{itemize}
    \item \textbf{Initial acceleration} is $g$ because the speed is zero, therefore the drag force is zero.
    \item At the \textbf{terminal speed}, the potential energy of the object is transferred into internal internal energy of the fluid by the drag force.
\end{itemize}

\subsubsection*{Motion of a Powered Vehicle}

\begin{itemize}
    \item $F_E$ represents the \textbf{motive force} provided by the engine.
    \item $F_R$ represents the sum of the \textbf{resistive force} opposing the motion of the vehicle.
    \item The \textbf{resultant force} on it is $F_E-F_R$.
\end{itemize}
Therefore its acceleration is
$$a=\frac{F_E-F_R}{m}$$

The \textbf{terminal speed} is reached when $F_R$ becomes equal and opposite to $F_E$, and $a=0$.
