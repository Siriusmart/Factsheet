\subsection{On the Road}

\begin{itemize}
    \item \textbf{Thinking distance} is the distance travelled by a vehicle in the time it takes the driver to react.

        For a vehicle moving at constant speed $v$, the thinking distance
        $$s_1=vt_0$$
        where $t_0$ is the \textbf{reaction time} of the driver.

    \item \textbf{Braking distance} is the distance travelled by a car in the time it takes to stop safely - from when the brakes are first applied.

        Assuming \textbf{constant deceleration} $a$, to zero speed from speed $u$.
        $$s_2=\frac{u^2}{2a}$$
\end{itemize}

\begin{align*}
    \textbf{stopping distance}&=\text{thinking distance}+\text{braking distance}\\
                              &=ut_0+\frac{u^2}{2a}
\end{align*}

\subsubsection*{Skidding}

Friction between the tyres and the road \textbf{prevent slipping} so the wheels roll along the roll.

\begin{itemize}
    \item If the driver \textbf{accelerates too fast}, the wheels skid.
    \item This is because there is an \textbf{upper limit} to the amount of friction between the tyres and the road.
\end{itemize}

\subsubsection*{Testing Friction}

To measure the \textbf{limiting friction} between the underside of a block and the surface it is on
\begin{enumerate}
    \item Pull the block with an increasing force until it slides.
    \item The limiting frictional force on the block is equal to the pull force just before the sliding occurs.
\end{enumerate}
