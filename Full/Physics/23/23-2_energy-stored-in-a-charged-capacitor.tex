\subsection{Energy Stored in a Charged Capacitor}
When a capacitor is charged, energy is stored in it because electrons are \textbf{forced onto one of its plates} and taken off the other plate. This energy is stored in the capacitor as \textbf{electric potential energy}.

\begin{enumerate}
    \item To increase the charge on the plates by a small amount $\Delta q$ from $q$ to $q+\Delta q$. The energy stored $\Delta E$ in the capacitance is equal to the work done to force the extra charge onto the plate.
        $$\Delta E=v\Delta q$$
    \item $v\Delta q$ is represented by the area of the vertical strip of width $\Delta q$ and height $v$ under the line. Therefore the area of this strip represents the work done $\Delta E$ in this small step.
    \item \textbf{Consider all the steps} from zero pd to the final pd $V$, the total energy stored is obtained by adding up the energy stored in each small step.

        $E$ is represented by the total area under the line from zero pd to pd $V$, which is a triangle of height $V$ and base length $Q$.
\end{enumerate}
\begin{align*}
    \text{Energy stored by the capacitor}\ E&=\frac{1}{2}QV\\
                                            &=\frac{1}{2}CV^2\\
                                            &=\frac{1}{2}\frac{Q^2}{C}
\end{align*}

\subsubsection*{Energy in a Thundercloud}

The thundercloud and the Earth below are like a pair of charged parallel plates.
\begin{enumerate}
    \item Because the thundercloud is charged, an electric field exists between the thundercloud and the ground - the potential difference between the thundercloud and the ground is $V=Ed$.
    \item For a thundercloud carrying constant charge $Q$.
        $$E=\frac{1}{2}QV=\frac{1}{2}QEd$$
    \item If the thundercloud raise up to a new height $d'$, the new energy stored
        $$E=\frac{1}{2}QEd'$$
    \item The \textbf{increase in energy}
        $$\Delta E=\frac{1}{2}QEd'-\frac{1}{2}QEd=\frac{1}{2}QE\Delta d$$
        where $\Delta d=d'-d$.
\end{enumerate}

The energy stored increases because \textbf{work is done} by the force (of wind) to overcome the electrical attraction between the thundercloud and the ground. To make the charged thundercloud move away from the ground.
