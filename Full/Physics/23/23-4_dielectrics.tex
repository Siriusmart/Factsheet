\subsection{Dielectrics}

The charge stored on the plates can be increased by \textbf{inserting a dielectric} between the plates.

Dielectrics are \textbf{electrically insulating materials} that increase the ability of a parallel-plate capacitor to store charge when a dielectric is placed between the plates of the capacitor.

Polythene and waxed paper are examples of dielectrics.
\begin{enumerate}
    \item When a dielectric is placed between two oppositely charged parallel plates, each molecule of the dielectric becomes \textbf{polarised}.
    \item The surface of the dielectric near the positive plate gains negative charge, the other surface gains positive charge.
    \item The positive side of the dielectric \textbf{attracts more electrons} from the battery onto the negative plate.

        The negative side of the dielectric \textbf{pushes electrons back} to the battery from the positive plate.
\end{enumerate}

The effect of a dielectric is to \textbf{increase the charge stored} in a capacitor for any given pd across the capacitor terminals - to \textbf{increase the capacitance} of the capacitor.

\subsubsection*{Relative Permittivity}

The ratio of charge stored with the dielectric to the charge stored without the dielectric is defined as the relative permittivity of the dielectric substance.
$$\varepsilon_r=\frac{Q}{Q_0}$$
where $Q$ is the charge stored by a parallel-plate capacitor when the place between the plates of the capacitor is \textbf{completely filled with the dielectric substance}.

Since $\dfrac{Q}{Q_0}=\dfrac{C}{C_0}$, the relative permittivity may be defined by the equation
$$\varepsilon_0=\frac{C}{C_0}$$

The relative permittivity of a substance is also called its \textbf{dielectric constant}.

For a parallel-plate capacitor
$$C=\frac{A\varepsilon_0\varepsilon_r}{d}$$
where $A$ is the surface area of each plate.

High capacitance can be achieved by
\begin{itemize}
    \item Making the area $A$ as large as possible.
    \item Making the plate spacing $d$ as small as possible.
    \item Filling the space between the plates with a dielectric which as a relative permittivity $\varepsilon_r$ as large as possible.
\end{itemize}

\subsubsection*{Polarisation Mechanisms}
The relative permittivity of a dielectric in a constant electric field is due to three different polarisation mechanisms.
\begin{itemize}
    \item \textbf{Orientation polarisation} occurs in substances which contains molecules where covalent bonds are formed between atoms of different elements. The electrons in each covalent bond are \textbf{shared unequally between the two atoms} joined by the bonds.

        The two atoms form a \textbf{permanent electric dipole} which one atom is positively charged and the other negatively charged.
        
        When the electric field is applied, the two atoms in each bond are displaced in the opposite direction so the dipole \textbf{align with the field} by turning slightly.
    \item \textbf{Ionic polarisation} occurs in substances where ions are held together by ionic bonds.

        The oppositely-charged ions of each ionic bond are \textbf{displaced in opposite directions} when an electric field is applied. The ions of each bond forms a dipole to \textbf{align with the field}.
    \item \textbf{Electronic polarisation} occurs where the electrons of each atom are \textbf{displaced relative to the nucleus} of the atom when an electric field is applied.

        The electron distribution and the nucleus forms a dipole that align with the field.
\end{itemize}

In an alternating electric field, the \textbf{polar dipoles rotate} and the \textbf{non-polar dipoles oscillate} one way then the opposite way as the field strength increases and decreases.
\begin{itemize}
    \item At \textbf{low frequencies}, the three polarisation mechanisms alternate in phase with the field.
    \item As frequency increases, each mechanism ceases to work due to the \textbf{inertia of the particles} involved and the \textbf{resistive forces} that opposes the motion of the dipoles.

        As a result, relative permittivity decreases as the frequency of the applied field increases.
\end{itemize}

The mass of the particles being moved by the field determines which mechanism ceases first as frequency increases - the greater the inertia, the lower the frequency it ceases.
\begin{enumerate}
    \item Orientation polarisation ceases first.
    \item Ionic polarisation next.
    \item Electronic polarisation last.
\end{enumerate}
