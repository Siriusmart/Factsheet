\subsection{Capacitance}

A \textbf{capacitor} is a device designed to store charge.

\textbf{Two parallel metal plates} placed near each other form a capacitor.
\begin{itemize}
    \item When the plates are connected to a battery, electrons move through the battery.
        \begin{itemize}
            \item Electrons are forced from the negative terminal of the battery onto one of the plates.
            \item An equal number of electrons leave the other plate to return to the battery via its positive terminal.
        \end{itemize}

        So each plate gains an \textbf{equal and opposite charge}.
    \item When we say the charge stored by the capacitor is $Q$, we mean one conductor stores charge $+Q$ and the other conductor stores charge $-Q$.
\end{itemize}

\subsubsection*{Charging at Constant Current}

This can be achieved using a \textbf{variable resistor}, a switch, a microammeter, and a cell in series with the capacitor.
\begin{itemize}
    \item When the switch is closed, the variable resistor is continually adjusted to \textbf{keep the microammeter reading constant}.
    \item At any given time $t$ after the switch is closed, the charge $Q$ on the capacitor.
        $$Q=It$$
\end{itemize}

The \textbf{capacitance} $C$ of a capacitor is defined as the \textbf{charge stored per unit pd}.

The unit of capacitance is the \textbf{farad} (F), equal to one coulomb per volt.
$$C=\frac{Q}{V}$$
