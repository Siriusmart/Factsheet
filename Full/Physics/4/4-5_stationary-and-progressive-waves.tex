\subsection{Stationary and Progressive Waves}

A stationary wave is formed when \textbf{two progressive waves pass through each other}. 
\begin{itemize}
    \item This can be achieved on a string \textbf{in tension} by \textbf{fixing both ends} and making the middle part vibrate.
    \item Progressive waves travel towards each other - reflected at the ends, then \textbf{pass through each other}.
\end{itemize}

A string's \textbf{fundamental mode of vibration}, or \textbf{first harmonic} is the simplest stationary wave pattern on a string.
\begin{itemize}
    \item Consists of a \textbf{single loop}.
    \item Has a \textbf{node} at either end (points of no displacement)
    \item Has an \textbf{antinode} midway between the nodes (point of maximum displacement)
\end{itemize}

For this pattern to occur
$$\text{Distance between adjacent nodes}=\frac{1}{2}\lambda$$
If the frequency of the waves sent along the rope is raised, the first harmonic pattern disappears and a \textbf{new pattern with two equal loops} is observed on the rope.

Stationary waves that vibrate freely \textbf{do not transfer energy} to their surroundings, because the \textbf{amplitude of vibration is zero} at the nodes, so there is no energy at the nodes.
\begin{itemize}
    \item The amplitude of vibration is a maximum at the antinodes, so there is maximum energy at the antinodes.
    \item Because the nodes and antinodes are at \underline{fixed positions}, no energy is transferred in a freely vibrating stationary wave pattern.
\end{itemize}

\subsubsection*{Explanation of Stationary Waves}
\begin{itemize}
    \item When two progressive waves in opposite directions are \textbf{in phase}, they \textbf{reinforce each other} to produce a large wave.
    \item A quarter of a cycle later, the two waves moved one-quarter wavelength in opposite directions, they are now in \textbf{antiphase} so they cancel each other.
    \item After a further quarter cycle, they are back in phase, the resultant is again a large wave.
\end{itemize}

In general, in any stationary wave pattern.
\begin{itemize}
    \item Amplitude of a vibrating partial in a stationary wave pattern \textbf{varies with position} from zero at a node to \textbf{maximum amplitude} at an antinode.
    \item Phase difference between two particles.
        \begin{itemize}
            \item Zero if separated by an even number of nodes.
            \item 180$^\circ$ if separated by an odd number of nodes.
        \end{itemize}
\end{itemize}

\subsubsection*{Examples of Stationary Waves}
\textbf{Sound resonances} at certain frequencies in an \textbf{air-filled pipe} closed at one end. These \textbf{resonance frequencies} occur when there is an \underline{antinode at an open end} and a node at the other end.

\textbf{Microwaves} from a transmitter are directed normally at a metal plate, which reflects the microwaves back to wards the transmitter. The reflected waves and the waves from the transmitter forms a stationary waves.
