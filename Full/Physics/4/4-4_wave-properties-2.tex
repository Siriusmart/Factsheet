\subsection{Wave Properties 2}

When two waves meet, they combine for an instant before the move apart, this combining effect is known as \textbf{superposition}.

The \textbf{principle of superposition} states that when two waves meet,the total displacement at a point is equal to the sum of the individual displacements at a point.
\begin{itemize}
    \item When two crests meet, the two waves reinforce each other and creates a \textbf{supercrest}.
    \item When two troughs meet, the two waves reinforce each other and creates a \textbf{supertrough}.
    \item When a crest meets a trough of the \underline{same amplitude}, the \textbf{resultant displacement is zero} - the two waves cancel each other out.
        \begin{itemize}
            \item If they are not the same amplitude, the result is called a \textbf{minimum}.
        \end{itemize}
\end{itemize}

\subsubsection*{Water Waves in a Ripple Tank}

\textbf{Vibrating dippers} on a water surface sends out circular waves, two sets of circular waves passes through each other continuously.
\begin{itemize}
    \item \textbf{Points of cancellation} are created where \underline{a crest meets a trough}. These points of cancellation are seen as \textbf{gaps in the wavefront}.
    \item \textbf{Points of reinforcement} are created where two crests or two troughs meet.
\end{itemize}

Waves continuously passing through each other at \textbf{constant frequency} and at a \textbf{constant phase difference} creates \textbf{points of cancellation and reinforcement} at \underline{fixed positions}. This effect is know as \textbf{interference}.
\begin{itemize}
    \item \textbf{Coherent} sources of waves produce an interference pattern where they overlap, because they vibrate at the \textbf{same frequency} and a \textbf{constant phase difference}.
\end{itemize}

\subsubsection*{Wave Properties with Microwaves}

A microwave receiver and transmitter can be used to demonstrate wave properties.

\begin{enumerate}
    \item Place the receiver in the path of the microwave beam from the transmitter, and move the receiver gradually away from the transmitter.
        \begin{itemize}
            \item The \textbf{receiver signal decreases with distance} from the transmitter.
            \item This shows the microwaves \textbf{become weaker} as they travel away from the transmitter.
        \end{itemize}

    \item Place a \textbf{metal plate} between the transmitter and the receiver.
        \begin{itemize}
            \item This shows that microwaves cannot pass through metal.
        \end{itemize}

    \item Use two metal plates to make a \textbf{narrow slit}.
        \begin{itemize}
            \item It can be shown that if the slit is made wider, less diffraction occurs.
        \end{itemize}

    \item Use a narrow metal plate and two additional plates to make \textbf{a pair of slits}.
        \begin{itemize}
            \item Use the receiver to find \textbf{points of cancellation and reinforcement} where the microwaves from the two slits overlap.
        \end{itemize}
\end{enumerate}
