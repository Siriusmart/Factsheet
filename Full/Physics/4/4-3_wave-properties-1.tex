\subsection{Wave Properties 1}

A \textbf{ripple tank} can be used to study wave properties - a shallow transparent tray of water with \textbf{sloping sides}, which prevent waves reflecting off the sides of tank.
\begin{itemize}
    \item \textbf{Wavefronts} are lines of constant phase difference (crests).
    \item The direction in which a wave travels is at right angles to the wavefront.
\end{itemize}

\subsubsection*{Reflection}
Straight waves directed at a certain angle to a hard surface \textbf{reflect off at the same angle} - the angle between the reflected wavefront and the surface is the same as the angle between the incident wavefront and the surface.

Also observed when a \textbf{light ray} is directed at a \textbf{plane mirror} - the angle between the incident ray and the mirror is equal to the angle between the reflected ray and the mirror.

\subsubsection*{Refraction}

When waves pass across a boundary at which \textbf{wave speed changes}, the \textbf{wavelength also changes}. If the wavefronts approach at an angle to the boundary, they \underline{change direction} as well as changing speed.
\begin{itemize}
    \item Water waves in a ripple tank refracts when they pass across a boundary from deep to shallow water. Because they \underline{move more slowly in shallow water} therefore changes direction.
    \item \textbf{Refraction of light} is observed when a light ray is directed into a glass block at an angle. The light ray changes direction when it crosses the glass boundary because it travels more slowly in glass than in air.
\end{itemize}

\subsubsection*{Diffraction}

Diffraction occurs when waves \textbf{spread out after passing through a gap} or round an obstacle.

Can be seen in a ripple tank when straight waves are directed at a gap.
\begin{itemize}
    \item The \textbf{narrower the gap}, the more the waves spread out.
    \item The \textbf{longer the wavelength}, the more the waves spread out.
\end{itemize}

Consider each point on a wavefront as a \textbf{secondary emitter of wavelets}
\begin{itemize}
    \item The wavelets from the points along a wavefront \textbf{travels only in the direction which the wave is travelling}.
    \item The combine to \textbf{form new wavefronts} beyond the gap.
\end{itemize}
