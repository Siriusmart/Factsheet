\subsection{Waves and Vibrations}

Waves that pass through a substance are \underline{vibrations that pass through a substance}, they are often referred to as \textbf{mechanical waves}. When waves progress through a substance, the particles of the substance vibrate in a certain way which makes \underline{nearby particles vibrate in the same way} and so on.
\begin{itemize}
    \item Sound waves
    \item Seismic waves
    \item Waves on strings
\end{itemize}

\textbf{Electromagnetic waves} are oscillating electric and magnetic fields that progress through space \underline{without the need for a substance} - the vibrating \textbf{electric field} generates a vibrating \textbf{magnetic field}, which generates a vibrating electric field further away, and so on.
\begin{itemize}
    \item Radio waves
    \item Microwaves
    \item Infrared radiation
    \item Light
    \item Ultraviolet radiation
    \item X-rays
    \item Gamma radiation
\end{itemize}

\textbf{Longitudinal waves} are waves which the direction of vibration of the particles is \underline{parallel} to the direction in which the wave travels.
\begin{itemize}
    \item Sound waves
    \item Primary seismic waves
\end{itemize}

\textbf{Transverse waves} are waves which the direction of vibration is \underline{perpendicular} to the direction in which the wave travels.
\begin{itemize}
    \item Electromagnetic waves
    \item Secondary seismic waves
    \item Waves on a string
\end{itemize}

\subsubsection*{Polarisation}
Transverse waves are \textbf{plane-polarised} if the vibrations \underline{stay in one plane} only. Otherwise if vibrations changes from one plane to another, then the waves are \textbf{unpolarised}.

Longitudinal waves cannot be polarised.
\begin{itemize}
    \item If \textbf{unpolarised light} (e.g. light from a filament lamp) passes through a \textbf{polaroid filter}, the transmitted light is polarised.
        \begin{itemize}
            \item The filter only allow through light which vibrate in a certain direction.
            \item According to the alignment of its molecules.
        \end{itemize}
    \item If unpolarised light is passed through \textbf{two polaroid filters}, the transmitted \textbf{light intensity} changes if one polaroid is turned relative to the other one.
        \begin{itemize}
            \item The filters are said to be \textbf{cross} when the transmitted intensity is a minimum.
            \item At this position, the polarised light from the first filter cannot pass through the second filter - as the alignment of the second filter is $90^\circ$ to the first.
        \end{itemize}
\end{itemize}

The \textbf{plane of polarisation} of an electromagnetic wave is defined as the plane in which the electric field oscillates.

\textbf{Polaroid sunglasses} reduces the glare of light reflected by water or glass.
\begin{itemize}
    \item Light reflected by water or glass is \textbf{polarised}.
    \item The intensity of reflected light is reduced when it passes through the polaroid sunglasses.
\end{itemize}
