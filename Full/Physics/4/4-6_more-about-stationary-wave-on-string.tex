\subsection{More about Stationary Waves on Strings}

The \textbf{controlled arrangement for producing stationary waves} consist of
\begin{itemize}
    \item A string tired at one end to a \textbf{mechanical vibrator}.
    \item The other end \textbf{passes over a pulley} and \textbf{supports a weight}, which keeps the tension in the string constant.
\end{itemize}

\begin{enumerate}
    \item The \textbf{first harmonic wavelength} is given by $\lambda_1=2L$, the \textbf{first harmonic frequency} is given by $f_1=c/2L$
    \item The \textbf{second harmonic} has a wavelength given by $\lambda_2=L$, the \textbf{frequency} is given by $f_2=2f_1$
    \item The \textbf{third harmonic} has a wavelength given by $\lambda_3=2L/3$, and \textbf{frequency} is given by $f_3=3f_1$.
\end{enumerate}

\subsubsection*{Explanation of Stationary Wave Pattern on a String}
\begin{enumerate}
    \item When a \textbf{progressive wave} is sent out by the vibrator, the crest \textbf{reverses its phase} when it reflects at the \textbf{fixed end} and travel back along the string as a \textbf{trough}.
    \item When it reaches the vibrator, it reflects and \textbf{reverses phase again}, travelling away from the vibrator as a crest.
\end{enumerate}
If this crest is reinforced by a crest created by the vibrator, the \textbf{amplitude of the wave is increased} and a stationary wave is formed.

The \textbf{key condition} is that the time taken for a wave to travel along the string and back should be equal to the time taken for a \textbf{whole number of cycles} of the vibrator. Which is
$$\frac{2L}{c}=\frac{m}{f}\qquad\text{where $m$ is a whole number}$$

\subsubsection*{Pitch}

The pitch of a note \underline{corresponds to frequency} - the pitch of a node from a \textbf{stretched string} can be altered by
\begin{itemize}
    \item Raising the \textbf{tension} or shortening its \textbf{length} to increase the pitch.
    \item Lowering the \textbf{tension} or increasing the \textbf{length} to lower the pitch.
\end{itemize}

A vibrating string can be tuned to be the same pitch as a tuning fork. With difference
\begin{itemize}
    \item Sound from a vibrating stings \textbf{includes all harmonic frequencies}
    \item Whereas a tuning for vibrates only at a \textbf{single frequency}.
\end{itemize}

The \textbf{first harmonic frequency} is given by
$$f=\frac{1}{2L}\sqrt{\frac{T}{\mu}}$$
