\subsection{Measuring Waves}

\begin{itemize}
    \item The \textbf{displacement} of a vibrating particle is its distance and direction from its \textbf{equilibrium position}.
    \item The \textbf{amplitude} of a wave is the \textbf{maximum displacement} of a vibrating particle.
        \begin{itemize}
            \item Height of a wave crest for transverse waves.
        \end{itemize}
    \item One \textbf{complete cycle} of a wave is from maximum displacement to the next maximum displacement.
    \item The \textbf{period} of a wave is the time for one complete wave to pass a fixed point.
    \item The \textbf{frequency} of a wave is the number of complete waves passing a point per second.
        \begin{itemize}
            \item Or the \textbf{number of cycles of vibration} of a particle per second.
        \end{itemize}
\end{itemize}
\begin{align*}
    \text{Time period}\ T&=\frac{1}{f}\\
    \text{Wave speed}\ c&=f\lambda
\end{align*}

\begin{itemize}
    \item The \textbf{phase} of a vibrating particle at a certain time is the \underline{fraction of a cycle it has completed} since the start of the cycle.
    \item The \textbf{phase difference} between two particles vibrating \underline{at the same frequency} is the fraction of a cycle between the vibrations of the two particles.
\end{itemize}
$$\text{Phase difference}=\frac{2\pi d}{\lambda}$$
