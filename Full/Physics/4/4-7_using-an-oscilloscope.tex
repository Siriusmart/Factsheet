\subsection{Using an Oscilloscope}

\begin{itemize}
    \item If a pd is applied across the X-plates, the spot is deflected horizontally.
    \item If a pd is applied across the Y-plates, the spot is deflected vertically.
\end{itemize}
In both cases, the displacement of the spot is \textbf{proportional to the applied pd}.

\begin{itemize}
    \item The X-plates are connected to a \textbf{time base circuit} which makes the spot moves at constant speed from left to right across the screen, and back again much faster. The X-scale can be calibrated in \textbf{milliseconds per division}.
    \item The pd to be displayed is connected to the Y-plates via the Y-input so the spot moves up and down as it moves left to right across the screen. The Y-input is calibrated in \textbf{volts per division}.
\end{itemize}
The calibration value of the Y-input is also called the \textbf{Y-sensitivity}.

\subsubsection*{Measuring Ultrasound}

\begin{itemize}
    \item The \textbf{time base circuit} of an oscilloscope can be used to trigger an \textbf{ultrasound transmitter} so it sends out a short pulse of ultrasonic waves.
    \item The \textbf{receiver signal} can be applied to the Y-input of the oscilloscope, so the waveform of the received pulse can be seen on screen.
\end{itemize}
