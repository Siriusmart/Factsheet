\subsection{Leptons at Work}

\textbf{Leptons} and \textbf{antileptons} can interact to produce hadrons - an \textbf{electron-positron annihilation event} produces a quark and a corresponding antiquark moving away in opposite directions, producing a \textbf{shower of hadrons} in each direction.

\textbf{Neutrinos} are produced when particles in accelerators collide.
\begin{itemize}
    \item Travel almost as fast as light.
    \item Billions passing through the Earth every second with \textbf{almost no interaction}.
\end{itemize}

Experiments showed that neutrinos and antineutrinos produced in beta decays were different from those produced in muon decays.
\begin{itemize}
    \item Neutrinos and antineutrinos from muon decays \underline{create only muons and no electrons}.
    \item If there were only one type of neutrino and antineutrino, equal number of electrons and muons would be produced.
\end{itemize}
We use symbol $\nu_\mu$ for the \textbf{muon neutrino} and $\nu_e$ for the \textbf{electron neutrino}.

\subsubsection*{Lepton Rules}

\begin{itemize}
    \item Leptons can be \underline{changed into other lepton} through the \textbf{weak interaction}.
    \item Produced or annihilated in \textbf{particle-antiparticle interactions}.
    \item Experiments indicate they don't break down into non-leptons - they appear \textbf{fundamental}.
\end{itemize}

An interaction between a lepton and a hadron, a neutrino or antineutrino can change into or from a \textbf{corresponding charged lepton}. An electron neutrino can interact with a neutron to produce a proton and an electron.
$$\nu_e+n\to p+e^-$$

In \textbf{muon decay}, the muon changes into a muon neutrino.
$$\mu^-\to e^-+\bar{\nu_e}+\nu_\mu$$
\begin{itemize}
    \item An electron is created to conserve charge.
    \item A corresponding antineutrino is created to conserve \textbf{lepton number}.
\end{itemize}

The \textbf{lepton number} is $+1$ for any lepton, $-1$ for any antilepton, 0 for any non-lepton. Lepton number is always conserved.
