\subsection{Particle Sorting}

The new particles created through \textbf{high-energy interactions} (apart from the neutrino) decay into other particles and anti particles.
\begin{itemize}
    \item \textbf{Charged pions} were often produced in pairs, leading to the conclusion that the $\pi^+$ and $\pi^-$ mesons are a particle-antiparticle pair.
    \item Same for \textbf{charged kaons}.
\end{itemize}

\begin{center}
    \begin{tabular}{|c|c|c|c|}
        \hline
        Particle & Relative charge & Rest energy/MeV & Interaction\\
        \hline
        proton $p$ & $+1$ & 938 & strong, weak, electromagnetic\\
        neutron $n$ & 0 & 939 & strong, weak\\
        electron $e^-$ & $-1$ & 0.5 & weak, electromagnetic\\
        neutrino $\nu$ & 0 & 0 & weak\\
        pions $\pi^+,\pi^0,\pi^-$ & $+1,0,-1$ & $140,135,140$ & strong, electromagnetic ($\pi^+,\pi^-$)\\
        kaons $K^+, K^0, K^-$ & $+1,0,-1$ & $494,498,494$ & strong, electromagnetic ($K^+,K^-$)\\
        \hline
    \end{tabular}
\end{center}

The particle symbol for \textbf{antiparticles} as a $\overline{\text{bar above them}}$, except for the \textbf{positron} $e^+$ and \textbf{antimuon} $\mu^+$ which have their own symbols. The \textbf{charged pions} are antiparticles of each others.

\subsubsection*{Particles Classification}

Particles can be divided into two groups according to whether or not they \textbf{interact through the strong interaction}.

\textbf{Hadrons} are particles and antiparticles that can interact through the strong interaction.
\begin{itemize}
    \item Hadrons can interact through \textbf{all four fundamental interactions}.
    \item Including protons, neutrons, $\pi$ and $K$ mesons.
    \item Apart from the stable proton, hadrons tend to \underline{decay through the weak interaction}.
\end{itemize}

\textbf{Leptons} are particles and antiparticles that do not interact through the strong interaction.
\begin{itemize}
    \item Leptons interact through the \textbf{weak} interaction, \textbf{gravitational} interaction and \textbf{electromagnetic} interaction (if charged) only.
    \item Including electrons, muons and neutrinos.
\end{itemize}

The \textbf{Large Hadron Collider} is a ring-shaped accelerator that boosts the kinetic energy of charged particles. Particles collide head-on to produce new particles.
\begin{align*}
    \text{total energy of particles}&=\text{rest energy of particles}+\text{kinetic energy of particles}\\
    \text{rest energy of products}&=\text{total energy before}-\text{kinetic energy of products}
\end{align*}
A collision can produce a range of products, as long as their total energy does not exceed the total energy before the collision, provided \textbf{conservation rules} are obeyed.

\subsubsection*{Hadron Classification}
Short-lived particles created through the \textbf{strong interaction} are hadrons. Some decay into protons as well as into pions, whereas \textbf{kaons} never decay into protons.

Hadrons can be divided into two groups
\begin{itemize}
    \item \textbf{Baryons} are protons and all other hadrons that decay into protons, directly or indirectly.
    \item \textbf{Mesons} are hadrons that do not include protons in their decay products.
\end{itemize}

Baryons and mesons are composed of smaller particles called \textbf{quarks} and \textbf{antiquarks}.
