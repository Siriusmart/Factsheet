\subsection{Quarks and Antiquarks}

\textbf{Strange particles} all decay through the weak interaction
\begin{itemize}
    \item \textbf{Kaons} decay into pions only.
    \item Other particles such as the $\Sigma$ particle
        \begin{itemize}
            \item Have rest masses which were always greater than the proton's rest mass.
            \item Decay in sequence or directly into protons and pions.
        \end{itemize}
\end{itemize}
Strange particles are created in twos.

To explain why certain reactions were not observed, a \textbf{strangeness number} was introduced for each particle and antiparticle so that strangeness is \underline{conserved in strong interactions}.

Starting with $+1$ for the $K^+$ meson. The strangeness for other strange particles and antiparticles can then be deduced from the observed reactions.

\begin{itemize}
    \item Strangeness is \textbf{always conserved in strong interaction}.
    \item And can change by $0,+1,-1$ in weak interactions.
\end{itemize}

The properties of \textbf{hadron}, such as charge, strangeness, and rest mass can be explained by assuming they are composed of smaller particles known as \textbf{quarks and antiquarks}.

\begin{center}
    \begin{tabular}{|c|c|c|c|}
        \hline
        & up $u$ & down $d$ & strange $s$\\
        \hline
        charge $Q$ & $+2/3$ & $-1/3$ & $-1/3$\\
        strangeness $S$ & 0 & 0 & $-1$\\
        baryon number $B$ & $+1/3$ & $+1/3$ & $+1/3$\\
        \hline
    \end{tabular}
\end{center}

\subsubsection*{Quark Combinations}

\textbf{Mesons} are hadrons each consisting of a quark and an antiquark.
\begin{itemize}
    \item A $\pi^0$ meson can be any quark-corresponding antiquark combination/
    \item Each pair of charged meson is a particle-antiparticle pair.
    \item There are two uncharged kaons: $K^0$ and $\bar{K^0}$.
\end{itemize}

\textbf{Baryons} are hadrons that consist of three quarks. 
\begin{itemize}
    \item Proton: $uud$
    \item Neutron $udd$
    \item A $\Sigma$ particle is a baryon containing a strange quark.
\end{itemize}

In terms of quarks, in $\beta^-$ decay, a down quark changes to an up quark, releasing an electron and an electron neutrino. The $\beta^+$ case is similar.
