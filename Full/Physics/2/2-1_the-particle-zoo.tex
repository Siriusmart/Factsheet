\subsection{The Particle Zoo}

\textbf{Cosmic rays} are high-energy particles that travel through space from the stars. Short-lived particles and antiparticles, and photons are created were created when cosmic rays enter the Earth's atmosphere.

They were thought to be from terrestrial radioactive substances, but it was disproved when found the ionising effect of the rays was significantly greater at 5000m than at ground level.

Most cosmic rays were \textbf{fast-moving protons} or \textbf{small nuclei}. They \underline{collide with gas atoms} in the atmosphere, creating showers of particles and antiparticles that can be detected at ground level with \textbf{cloud chambers} or other detectors.
\begin{itemize}
    \item The \textbf{muon} $\mu$ (heavy electron)
        \begin{itemize}
            \item Negatively charged particle.
            \item With rest mass over 200 times the rest mass of the electron.
        \end{itemize}
    \item The \textbf{pion} $\pi$ (pi meson)
        \begin{itemize}
            \item Can be positively ($\pi^+$), negatively ($\pi^-$) charged, or neutral ($\pi^0$).
            \item With rest mass greater than a muon but less than a proton.
        \end{itemize}
    \item The \textbf{kaon} $K$ (K meson)
        \begin{itemize}
            \item Can be positively ($K^+$), negatively ($K^-$) or neutral ($K^0$).
            \item With rest mass greater than a pion but less than a proton.
        \end{itemize}
\end{itemize}

The existence of the \textbf{exchange particle} for the strong nuclear force between nucleons were predicted, with a range no more than 1fm. The predicted mass is between the electron and proton mass, they are called \textbf{mesons} for being in the middle of the two masses.

The $\mathbf{\pi}$ mesons were discovered from microscopic tracks found in photographic emulsion exposed to cosmic rays at high altitude, proving the prediction correct.

The \textbf{muon} was discovered from an unusually long track, indicating it lasted much longer than a strongly interacting particle should. Further investigation showed it to be a \textbf{heavy electron} that \textbf{decays through the weak interaction}.

\subsubsection*{Strange Particles}

Further cloud chamber photographs showed the existence of short-lived particles known as \textbf{kaons}.
\begin{itemize}
    \item \underline{Produced in twos} through the \textbf{strong interaction} - similar to pions.
    \item When protons crash into nuclei at high speed, they \underline{travel far beyond the nucleus} in which they originate before they decay.
    \item The decay of kaons \underline{took longer than expected} and include \textbf{pions} as the product - this mean kaons must decay via the \textbf{weak interaction}.
\end{itemize}
These properties led to them being called \textbf{strange particles}.

\subsubsection*{Decay Pathways}

Exotic particles can be created using \textbf{accelerators} in which protons \textbf{collide head-on} with other protons at high speed. The kinetic energy of the protons is \underline{converted into mass} in the creation of new particles.

These created particles could be studied in \textbf{controlled conditions}, their \textbf{rest mass}, \textbf{charge} and \textbf{lifetimes} were measured. Their decay modes were worked out.
\begin{itemize}
    \item A \textbf{kaon} can decay into
        \begin{itemize}
            \item Pions
            \item A muon and an antineutrino
            \item An antimuon and a neutrino
        \end{itemize}
    \item A \textbf{charged pion} can decay into
        \begin{itemize}
            \item A muon and an antineutrino
            \item An antimuon and a neutrino
        \end{itemize}
    \item A \textbf{$\pi^0$ meson} can decay into \textbf{high-energy photons}.
\end{itemize}

Decays always obey the conservation rules for energy, momentum, and charge.
