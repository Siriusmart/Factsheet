\subsection{Diffraction}

Diffraction is the \textbf{spreading of waves} when they pass through a gap or by an edge.

\begin{itemize}
    \item \textbf{Diffraction of water waves} through a gap can be observed using a \textbf{ripple tank}. The diffracted waves spread out more if
        \begin{itemize}
            \item The gap is made narrower.
            \item The wavelength is made larger.
        \end{itemize}
    \item \textbf{Diffraction of light} by a single slit can be demonstrated by directing a \textbf{parallel beam of light} at the slit.
\end{itemize}

The diffracted light forms a pattern that can be observed on a white screen.
\begin{itemize}
    \item A \textbf{central fringe} with further fringes either side of the central fringe.
    \item The \textbf{intensity} of the fringes is greatest at the centre of the central fringe.
    \item The central fringe is \textbf{twice as wide} as each of the other fringes.
    \item The \textbf{peak intensity} of each fringe decreases with distance from the centre.
    \item Each outer fringe is the \textbf{same width}.
    \item The outer fringe is much \textbf{less intense} than the central fringe.
\end{itemize}

The width is measured from \textbf{minimum to minimum intensity}.

For a \textbf{monochromatic light} with wavelength $\lambda$
$$w=\frac{2D\times\lambda}{a}$$
where $a$ is the width of the single slit.
\begin{itemize}
    \item The greater the \textbf{wavelength}, the wider the fringes.
    \item The narrower the \textbf{slit}, the wider the fringes.
\end{itemize}
