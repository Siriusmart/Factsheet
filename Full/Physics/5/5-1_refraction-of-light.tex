\subsection{Refraction of Light}

When considering the effect of lenses or mirrors on the path of light, we draw diagrams using \textbf{light rays} and \textbf{normals}.
\begin{itemize}
    \item \textbf{Light rays} represent the \underline{direction of travel} of wavefronts.
    \item The \textbf{normals} is an imaginary line \underline{perpendicular to a boundary} between two materials or a surface.
\end{itemize}

\textbf{Refraction} is the \underline{change of direction} that occur when light \textbf{passes at an angle} across a boundary between two \textbf{transparent substances}. When entering a glass block from air, the light ray bends
\begin{itemize}
    \item \textbf{Towards the normal} when it passes from air into glass.
    \item \textbf{Away from the normal} when it passes from glass into air.
\end{itemize}
\textbf{No refraction} takes place if the incident light ray is \underline{along the normal}.

At a boundary between two transparent substances, the ray bends towards the normal if it passes into a \textbf{more dense substance}.

\subsubsection*{Investigating Refraction by Glass}
\begin{enumerate}
    \item Use a \textbf{ray box} to direct a light ray into a \textbf{rectangular glass block} at different angles of incident at point $P$ on one of the sides.
    \item For each angle of incidence, mark point $Q$ where the light leaves the block.
\end{enumerate}

The \textbf{angle of incidence} is the angle between the incident light ray and the \textbf{normal} at the point of incident. The \textbf{angle of reflection} is the angle between the refracted light ray and the normal at the point of incident.
\begin{itemize}
    \item The angle of diffraction $r$ is always less than the angle of incident $i$.
    \item \textbf{Snell's law}: the ratio $\sin i/\sin r$ is the same for each light ray.
        \begin{itemize}
            \item The ratio is referred to as the \textbf{refractive index} $n$ of glass.
        \end{itemize}
\end{itemize}
$$\text{refractive index of the substance}\ n=\frac{\sin i}{\sin r}$$
\textbf{Partial reflection} also occur when a light ray in air enters any refractive substance.

The angle of refraction of the light ray emerging from a rectangular glass block is the same as the \textbf{angle of incidence} of the ray entering the block.
\begin{itemize}
    \item The two side of the block are \textbf{parallel to each other}.
    \item Refractive index when entering the glass is $n$, when leaving the glass is $1/n$, so the combined effect is $n=1$.
\end{itemize}
