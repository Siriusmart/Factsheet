\subsection{More about Interference}

For a double slit illuminated by light from a single slit, each wave crest always pass through one of the double slits a \textbf{fixed time after it passes through the other slit}. The double slits therefore emit wavefronts with a \textbf{constant phase difference}.

Light from two nearby lamp bulbs does not form an interference pattern because the two light sources \underline{emit light waves at random}, the points of cancellation and reinforcement would \textbf{change at random}, so no interference pattern is possible.

\subsubsection*{Light sources}

\begin{itemize}
    \item \textbf{Vapour lamps and discharge tubes} produce light with a \underline{dominant colour}. They are in effect a \textbf{monochromatic light source} because its spectrum is dominated by light of a certain colour.
    \item \textbf{Light from a filament lamp or the Sun} is composed of the \underline{colours of the spectrum} and covers a \textbf{continuous range of wavelengths}.
        \begin{itemize}
            \item Light from a filter is a particular colour because it contains a \textbf{much narrower range of wavelengths}.
        \end{itemize}
    \item \textbf{Laser} is almost \textbf{perfectly parallel} and \textbf{highly monochromatic} - its wavelength can be specified to within a nanometre.
        \begin{itemize}
            \item A laser is a convenient source of \textbf{coherent light}.
        \end{itemize}
\end{itemize}

\subsubsection*{White Light Fringes}
As we know blue light fringes are much closer together than the red light fringes.

\begin{itemize}
    \item The \textbf{central fringe} is white because every colour contributes at the centre of the pattern.
    \item The \textbf{inner fringes} are tinged with blue on the inner side and red on the outer side, this is because the two fringe patterns do not overlap exactly.
\end{itemize}
