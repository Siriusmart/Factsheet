\subsection{Total Internal Reflection}
\begin{itemize}
    \item If the \textbf{angle of incidence} is increased to a certain value known as the \textbf{critical angle}, the light ray \underline{travels along the boundary}.
    \item If the angle of incidence is increased beyond the critical angle, the light ray undergoes \textbf{total internal reflection}.
\end{itemize}
Total internal reflection takes place if
\begin{itemize}
    \item The incidence substance has a \textbf{larger refractive index} than the other substance.
    \item The angle of incidence \textbf{exceeds the critical angle}.
\end{itemize}

When light enters a diamond
\begin{enumerate}
    \item It is \textbf{split into the colours of the spectrum} as diamond's very \underline{high refractive index} separates the colours more than any other substance does.
    \item The high refractive index gives a \textbf{small critical angle}
        \begin{itemize}
            \item So a light ray entering a diamond may be \textbf{totally internally reflected many times} before it emerges.
            \item So its colours \textbf{spreads out more and more}.
        \end{itemize}
\end{enumerate}

\subsubsection*{Optical Fibres}
Optical fibres are used in \textbf{medical endoscopes} to see inside the body, and communications to \textbf{carry light signals}.
\begin{itemize}
    \item The light ray is \textbf{totally internally reflected} each time it reaches the fibre boundary.
    \item At each point where the light ray reaches the boundary, the angle of incidence \textbf{exceeds the critical angle} of the fibre.
\end{itemize}
Unless the radius of the bend is too small, then the light will not totally internally reflect.

A \textbf{communications optical fibre} allows \textbf{pulses of light} that enter at one end from a \underline{transmitter} to reach a \underline{receiver} at the other end.
\begin{itemize}
    \item Fibres are \textbf{highly transparent} to minimise absorption of light
        \begin{itemize}
            \item Otherwise would reduce the amplitude of the pulses the further they travel in the fibre.
        \end{itemize}
\end{itemize}

Each fibre consists of a \textbf{core surrounded by a layer of cladding}.
\begin{itemize}
    \item Total internal reflection takes place at the \textbf{core-cladding boundary}
        \begin{itemize}
            \item If there were no cladding, such crossover would mean the signal \textbf{would not be secure} - they would \textbf{reach the wrong destination}.
        \end{itemize}
\end{itemize}

The core must be \textbf{very narrow} to prevent \textbf{modal dispersion}.
\begin{itemize}
    \item This occurs in a \textbf{wide core} as light travelling along the axis of the core \textbf{travels a shorter distance per metre of fibre} than light that repeatedly undergoes TIR.
        \begin{itemize}
            \item A pulse of light sent would be \textbf{longer than it ought to be}.
            \item If it is too long, it would merge with the next pulse.
        \end{itemize}
\end{itemize}

\textbf{Material dispersion} occurs if white light is used instead of \textbf{monochromatic light}, because the speed of light in glass of the optical fibre \underline{depends on the wavelength} of light travelling through it.
