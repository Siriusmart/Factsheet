\subsection{Electric Field Strength}

Provided the object's size and charge are both sufficiently small, the object may be used as a \textbf{test charge} to measure the strength of the field at any position in the field.

The electrical field strength $E$ at a point in the field is defined as the \textbf{force per unit charge} on a positive test charge placed on that point.

The unit of $E$ is the \textbf{newton per coulomb} N\,C$^{-1}$.
$$E=\frac{F}{Q}$$

\subsubsection*{The Lighting Rod}

Air is an insulator provided it is not subjected to an electric field that is too strong - such a field \textbf{ionises the air molecules} by pulling electrons out of the molecules.

In a lightening strike to the ground
\begin{enumerate}
    \item A cloud becomes more and more charged.
    \item The electric field in the air becomes stronger and stronger.
    \item The \textbf{insulating property of air suddenly breaks down}.
    \item A massive discharge of electric charge occurs between the cloud and the ground.
\end{enumerate}

When there is a lightening rod connected to the ground.
\begin{enumerate}
    \item When a charged cloud is overhead, it creates a \textbf{very strong electric field near the tip} of the rod.
    \item The air molecules near the tip are ionised by this very strong field.
    \item The \textbf{ions discharge the thundercloud} making a lightening strike less likely.
\end{enumerate}

\subsubsection*{Electric Field Between Parallel Plates}

Field lines between two oppositely charge parallel plates are
\begin{itemize}
    \item Parallel to each other.
    \item At \textbf{right angles to the plates}.
    \item From the positive plate to the negative plate.
\end{itemize}

The field between the plates is \textbf{uniform}, because the electric field has the \textbf{same magnitude and direction} everywhere between the plates.
$$\text{Electric field strength}\ E=\frac{V}{d}$$
where $V$ is the potential difference between the plates, and $d$ their separation.

\textbf{Proof}
\begin{enumerate}
    \item The force on a small charge in the field is given by $F=QE$.
    \item If the charge is moved from the positive to the negative plate, the work done $W=Fd=QEd$.
    \item The potential difference is the \textbf{work done per unit charge} when a small charge is moved through the field.
        \begin{align*}
            V&=\frac{W}{Q}\\
            V&=Ed\\
            E&=\frac{V}{d}
        \end{align*}
\end{enumerate}

\subsubsection*{Field Factors}
\begin{itemize}
    \item Around any charged body, the greater the charge on the body, the stronger the electric field is.
    \item For a metal conductor, the more concentrated the charge is on the surface, the greater the strength of the electric field is above the surface.
\end{itemize}
For a charge on a plate of surface area $A$, the electric field strength between the plates $E\propto\dfrac{Q}{A}$.

Introducing a constant of proportionality $\varepsilon_0$ satisfying
$$\frac{Q}{A}=\varepsilon_0E$$
where $\varepsilon_0=8.85\times10^{-12}$F\,m$^{-1}$.
