\subsection{Field Patterns}

Like charges repel, unlike charges attract.

\textbf{Electrons} are responsible for charging in most situations.
\begin{itemize}
    \item An \textbf{uncharged atom} contains an equal number of protons and electrons.
    \item An \textbf{uncharged solid} contains equal number of electrons and protons.
\end{itemize}

Most plastic materials can be charged quite easily by \textbf{rubbing with a dry cloth}.
\begin{enumerate}
    \item Electrons are \textbf{transferred from the cloth} to the rod when rubbed.
    \item So the rod becomes positively charged, and the cloth becomes negatively charged.
\end{enumerate}

\textbf{Electrical conductors} such as metals contains lots of \textbf{free electrons}, which move about inside the metal and are not attached to any one atom.

To charge a metal
\begin{enumerate}
    \item It must be \textbf{isolated from the Earth}.

        Otherwise, any charge given to it is neutralised by electrons transferring between the conductor and the Earth.
    \item Then the isolated conductor can be \textbf{charged by direct contact} with any charged object.
\end{enumerate}

If a positively charged isolated conductor is earthed, electrons transfer from the Earth to the conductor to \textbf{discharge it}.

\textbf{Electrically insulating materials} do not contain free electrons - all electrons in an insulator are \textbf{attached to individual atoms}. Some insulators are easy to charge because their surface atom easily gain or lose electrons.

\subsubsection*{The Shuttling Ball Experiment}

The shuttling ball experiment shows that an \textbf{electric current is a flow of charge}. A conducting ball is suspended by an insulating thread between two vertical plates.

When a high voltage is applied across the two plates, the ball bounces back and forth between the two plates.
\begin{enumerate}
    \item Each time it \textbf{touches the negative plate}, the ball  gains some electrons and becomes negatively charged.
    \item It is then repelled by the negative plate and pull across to the positive plate.
    \item When the contact is made, electrons on the ball \textbf{transfer to the positive plate}.
    \item The ball is now positively charged and is repelled back to the negative plate to repeat the cycle.
\end{enumerate}

The shuttling ball causes a current around the circuit, because the electrons are transferred from the negative plate to the positive plate by the shuttling ball.

For a ball shuttling back and forth at frequency $f$.
$$I=\frac{\Delta Q}{\Delta t}=Qf$$

\subsubsection*{Gold Leaf Electroscope}

The gold leaf electroscope is used to \textbf{detect charge}.
\begin{enumerate}
    \item If a charge object is \textbf{in contact with the metal cap} of the electroscope, some of the charge on the object \textbf{transfers to the electroscope}.
    \item As a result, the gold leaf and the metal stem which is attached to the cap \textbf{gain the same type of charge}.
    \item The leaf rises because it is repelled by the stem.
\end{enumerate}

If another object with the same type of charge is brought near the electroscope, the leaf \textbf{rises further} because the object forces some charge on the cap to transfer to the leaf and stem.

\subsubsection*{Field Lines and Patterns}

Any two charged objects exert \textbf{equal and opposite forces} on each other without being directly in contact.
\begin{itemize}
    \item An \textbf{electric field} is said to surround each charge.
    \item If a \textbf{small positive test charge} is placed near a body with a \textbf{much bigger charge}, the path a free positive test charge follow sis called a field line.
\end{itemize}

The direction of an electric field line is the direction a positive test charge would move along.
