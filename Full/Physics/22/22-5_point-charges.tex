\subsection{Point Charges}

\begin{itemize}
    \item A \textbf{point charge} is a convenient expression for a charged object in a situation where distances under consideration are much \textbf{greater than the size of the object}.
    \item A \textbf{test charge} in an electric field is a point charge that \textbf{does not alter the electric field} in which it is placed.
\end{itemize}

Consider the electric field due to a point charge $+Q$
\begin{align*}
    F&=\frac{1}{4\pi\varepsilon_0}\frac{Qq}{r^2}\\
    E&=\frac{F}{q}=\frac{1}{4\pi\varepsilon_0}\frac{Q}{r^2}
\end{align*}

\subsubsection*{Radial Fields}

The electric field lines of force surrounding a point charge are \textbf{radial} - the \textbf{equipotentials} are concentric circles centred on $Q$.

At distance $r$ from $Q$, the electric field strength $E=\frac{1}{4\pi\varepsilon_0}\frac{Q}{r^2}$.

The curve is an \textbf{inverse-square law} curve because $E$ is proportional to $\dfrac{1}{r^2}$.
\begin{itemize}
    \item The field strength equations for both gravitational and electric field are inverse-square relationships.
    \item Because both the force between two point charges and the force between two point masses vary with distance according to the inverse-square law.
\end{itemize}
$$V=\frac{1}{4\pi\epsilon_0}\frac{Q}{r^2}$$

\begin{itemize}
    \item Gravitational potential in a gravitational field is \textbf{always negative}, because the force is always attractive.
    \item Electric potential in the electric field near a point charge $Q$ can be positive or negative according to whether $Q$ is a positive or negative charge.
\end{itemize}

The area under a section of the graph of \textbf{electric field strength against distance} gives the work done per unit charge when a positive test charge is moved through the distance represented by that section.
