\subsection{Electric Potential}

\subsubsection*{The Van de Graff Generator}

Charge created when the rubber belt \textbf{rubs against a pad} is carried by the belt up to the metal dome of the generator. As charge gathers on the dome, the \textbf{potential difference} between the dome and Earth increases until \textbf{sparking} occurs.

\begin{enumerate}
    \item \textbf{Work is done} to charge the dome because a force is needed to move the charge on the belt up the dome.
    \item \textbf{Electrical energy} of the dome increases as it charges up.
    \item Some of this energy is \textbf{transferred from the dome} when a spark is created.
\end{enumerate}

The \textbf{electric potential} at a certain position in any electric field is defined as the \textbf{work done per unit positive charge} on a positive test charge when it is moved from infinity to that position.
\begin{itemize}
    \item The position of zero potential is infinity.
    \item The unit of electric potential is the \textbf{volt} equal to 1J\,C$^{-1}$.
\end{itemize}
For a positive test charge
$$V=\frac{E_p}{Q}$$

\subsubsection*{Potential Gradients}
\textbf{Equipotentials} are surfaces of constant potential.
\begin{itemize}
    \item A test charge moving a long an equipotential has \textbf{constant potential energy}.
    \item \textbf{No work is done} by the electric field on the test charge because the force due to the field is at right angles to the equipotential.
\end{itemize}

Both equipotentials for an electric and gravitational field are \textbf{surfaces of constant potential energy} for an appropriate test object - in one case a test charge, the other a test mass.

The \textbf{potential gradient} at any position in an electric field is the \textbf{change in potential per unit change of distance} in a given direction.
\begin{itemize}
    \item If the field is \textbf{non-uniform}, the potential gradient varies according to position.

        The closer the equipotentials are, the greater the potential gradient is at right angles to the equipotentials.
    \item If the field is \textbf{uniform}, the equipotentials between the plates are \textbf{equally spaced lines} parallel to the plates.

        The potential gradient between two parallel plates is
        \begin{itemize}
            \item Constant.
            \item Such that the potential increases in the opposite direction to the electric field.
            \item Equal to $E=\dfrac{V}{d}$
        \end{itemize}
\end{itemize}

The electric field strength is equal to the negative of the potential gradient.
$$E=-\frac{dV}{dx}$$
