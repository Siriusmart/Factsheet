\subsection{Particle Interaction}

When a single force acts on an object, it \underline{changes the momentum} of the object.

The \textbf{momentum} of an object is its mass multiplied by its velocity.

Two objects exert \textbf{equal and opposite forces} on each other when they interact - momentum is transferred between the objects by these forces.

The \textbf{electromagnetic force} between two charged objects is due to the exchange of \textbf{virtual photons}. They cannot be detected directly, and if intercepted, would stop the force acting. A \textbf{Feynman diagram} can be used to represent the interaction.

\subsubsection*{The Weak Nuclear Force}

\textbf{Beta decay} cannot be caused by the electromagnetic force as the neutron is uncharged. The \textbf{weak nuclear force} changes a neutron into a proton and vice versa in beta decays. It is weaker than the strong nuclear force because it doesn't affect the weak nuclear force.

In both beta decays, an electron or positron, and a neutrino or an antineutrino is created. Neutrinos hardly interact with other particles, but there are exceptions.
\begin{itemize}
    \item A neutrino interact with a \textbf{neutron} and change it into a proton - $\beta^-$ emission.
    \item A antineutrino interact with a \textbf{proton} and change it into a neutron - $\beta^+$ emission.
\end{itemize}

These interactions are due to the exchange of \textbf{W posons}:
\begin{itemize}
    \item Have a non-zero rest mass.
    \item Have a very short range less than 0.001fm.
    \item Can be positively $W^+$ or negatively $W^-$ charged.
\end{itemize}

\textbf{W bosons} was first detected when protons and antiprotons at very high energies were made to collide and annihilate each other. At sufficiently high energies, these annihilation events produce \textbf{W bosons and protons}. The $\beta$ particles from the W boson decays were detected as predicted.

If no neutrino or antineutrino is present:
\begin{itemize}
    \item $W^-\to\beta^-+\bar\nu$
    \item $W^+\to\beta^++\nu$
\end{itemize}

\subsubsection*{Electron Capture}

A proton in a \textbf{proton-rich nucleus} turns into a neutron as a result of interacting through the weak interaction with an \textbf{inner-shell electron} from outside the nucleus. Where a $W^+$ boson changes the electron into a neutrino.
\begin{align*}
    p&\to n+W^+\\
    e^-+W^+&\to\nu
\end{align*}

The same can happen when a proton and electron collide at very high speeds.
\begin{itemize}
    \item For an electron with sufficient energy, the overall change could occur as a $W^-$ exchange \underline{from the electron} to the proton.
\end{itemize}

The photon and W bosons are known as \textbf{force carriers} because they are exchanged when the electromagnetic force and the weak nuclear force act respectively.
