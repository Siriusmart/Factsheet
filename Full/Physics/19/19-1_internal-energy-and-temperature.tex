\subsection{Internal Energy and Temperature}

\textbf{Energy transfer} between two objects takes place if
\begin{itemize}
    \item One object \textbf{does work} on another object by \textbf{exerting a force}.
    \item One object is hotter than the other project, so \textbf{energy is transferred by heating} because of a \textbf{temperature difference} between the two objects.
\end{itemize}

\subsubsection*{Internal Energy}

\begin{itemize}
    \item The \textbf{internal energy} of an object is the energy of its molecules due to their individual movements and positions.
    \item The internal energy of an object due to its temperature is called \textbf{thermal energy}.
\end{itemize}

The internal energy of an object is increased by
\begin{itemize}
    \item Energy transfer \textbf{heating} the object.
    \item \textbf{Work done} on the object, e.g. electricity.
\end{itemize}

If internal energy of an object \textbf{stays constant}, then either
\begin{itemize}
    \item There is \textbf{no energy transfer} by heating and no work is done.
    \item Energy transfer by heating and work done \textbf{balance each other out}.
\end{itemize}

The \textbf{first law of thermodynamics} states
$$\text{Change of internal energy of the object}=\text{total energy transfer due to work done and heating}$$

The \textbf{direction of energy transfers} determines whether the overall internal energy of the object increased or decreases.

\subsubsection*{States of Matter}

A \textbf{molecule} is the smallest particle of a pure substance that is \textbf{characteristic of the substance}.

An \textbf{atom} is the smallest particle of an element that is characteristic of the element.

\begin{itemize}
    \item In a \textbf{solid}, the molecules are \textbf{held to each other} by forces due to the electrical charges of the protons and electrons in the atoms.
        \begin{itemize}
            \item Molecules \textbf{vibrate randomly} about fixed positions.
            \item The higher the temperature of the solid, the more the molecules vibrate.
            \item The energy supplied to raise the temperature of a solid \textbf{increases the kinetic energy of the molecules}.
            \item If the temperature is raised enough, the solid \textbf{melts} - the molecules vibrate so much that they \textbf{break free from each other} and the substance loses its shape.
            \item The energy supplied to melt a solid \textbf{raises the potential energy} of the molecules because they break free from each other.
        \end{itemize}
    \item In a \textbf{liquid}, the molecules move about at random \textbf{in contact with each other}.
        \begin{itemize}
            \item The forces between the molecules are not strong enough to hold the molecules in fixed positions.
            \item The higher the temperature of a liquid, the faster its molecules move.
            \item The energy supplied to a liquid to raise its temperature \textbf{increases the kinetic energy of the liquid molecules}.
            \item Heating the liquid further causes it to \textbf{vaporise} - the molecules have \textbf{sufficient kinetic energy} to break free and move away from each other.
        \end{itemize}
    \item In a \textbf{gas}, the molecules move about at random but \textbf{much further apart} on average than in a liquid.
        \begin{itemize}
            \item Heating a gas makes the molecules speed up and so \textbf{gain kinetic energy}.
        \end{itemize}
\end{itemize}

The internal energy of an object is the \textbf{sum of the random distribution} of the kinetic and potential energies of its molecules.

Increasing the internal energy of a substance increases the kinetic/potential energy associated with the random motion and positions of its molecules.

\subsubsection*{The Temperature Scale}

\begin{itemize}
    \item The temperature of an object is a measure of the \textbf{degree of hotness} of the object. The hotter an object the more \textbf{internal energy} it has.
    \item For any two objects at the same temperature, they are in \textbf{thermal equilibrium}, and \textbf{no overall energy transfer by heating} will take place.
\end{itemize}

A temperature scale is defined in terms of \textbf{fixed points} - standard degrees of hotness that can be accurately reproduced.

The \textbf{Celsius scale} of temperature in units C$^\circ$ is defined in terms of
\begin{itemize}
    \item \textbf{Ice point} 0C$^\circ$ - the temperature of pure melting ice.
    \item \textbf{Steam point} 100C$^\circ$ - the temperature of steam at standard atmospheric pressure.
\end{itemize}

The \textbf{absolute scale} of temperature in units kelvins is defined in terms of
\begin{itemize}
    \item \textbf{Absolute zero} 0K - the lowest possible temperature.
    \item \textbf{Triple point of water} - 273.1K, the temperature which ice, water and water vapour co-exist in thermodynamic equilibrium.
\end{itemize}
$$\text{Temperature in Celsius}=\text{absolute temperature in kelvins}-273.1$$

The \textbf{absolute zero} is the lowest possible temperature, because an object at absolute zero has \textbf{minimum internal energy}, regardless of the substance the object consists of.
