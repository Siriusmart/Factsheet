\subsection{Change of State}

\begin{itemize}
    \item The \textbf{density of a gas} is much less than the density of the same substance in liquid or solid state.

        Molecules of liquid and solid are \textbf{packed together in contact} with each other. Whereas the molecules of gas are on average \textbf{separated from each other} by relatively large distances.
    \item Liquid and gases \textbf{can flow} but solids can't.

        The atoms in a solid are \textbf{locked together} by strong force bonds, which atoms are unable to break free from.

        In a liquid or gas, the molecules are not locked together because they have \textbf{too much kinetic energy}, and the force bonds are \textbf{not strong enough} to keep the molecules fixed to each other.
\end{itemize}

\subsubsection*{Latent Heat}

\begin{itemize}
    \item When a \textbf{solid is heated at its melting point}, its atoms vibrate so much that they break free from each other. The solid therefore \textbf{becomes a liquid} due to the energy supplied at the melting point.

        The energy needed to melt a solid at its melting point is called \textbf{latent heat of fusion}.

        During melting, \textbf{no temperature change} takes place even though the solid is being heated.
    \item Latent heat is released when a \textbf{liquid solidifies}.

        The liquid molecules \textbf{slow down as the liquid cools} until the temperature decreases to the melting point, where the molecules move slowly enough for the force bonds to \textbf{lock the molecules together}.

        Some latent heat is released \textbf{keeps the temperature at the melting point} until all liquid has solidified.
    \item When a \textbf{liquid is heated at its boiling point}, the molecules gain kinetic energy to \textbf{overcome the bonds} that hold them close together. The molecules therefore \textbf{break away from each other} to form bubbles of vapour in the liquid.

        The energy needed to vaporise a liquid is called the \textbf{latent heat of vaporisation}.
    \item Latent heat is released when a \textbf{vapour condenses}.

        Vapour molecules slow down as the vapour is cooled. The molecules move slowly enough for the force bonds to \textbf{pull the molecules together} to form a liquid.
\end{itemize}

Some solids \textbf{vaporise directly} when heated in a process called \textbf{sublimation}.

In general, much more energy is needed to vaporise a substance than to melt it.
\begin{itemize}
    \item The \textbf{specific latent heat of fusion} of a substance is the energy needed to \textbf{change the state of unit mass of the substance} from solid to liquid without change in temperature.
    \item The \textbf{specific latent heat of vaporisation} of a substance is the energy needed to \textbf{change the state of unit mass of the substance} from liquid to gas without change in temperature.
\end{itemize}
$$\text{Energy needed}\ Q=ml$$
The unit of specific latent heat is J\,kg$^{-1}$.
