\subsection{The Experimental Gas Laws}

The \textbf{pressure} of a gas is the force per unit area that the gas exerts normally on a surface.

The unit of pressure is the \textbf{pascal}, $1\text{Pa}=1\text{N\,m}^{-2}$.

The pressure of gas depends on
\begin{itemize}
    \item Its \textbf{temperature}.
    \item The \textbf{volume} of the gas container.
    \item The \textbf{mass} of gas in the container.
\end{itemize}

\subsubsection*{Boyle's Law}

Boyle's law states that for a \textbf{fixed mass} of gas at \textbf{constant temperature}.
$$pV=\text{constant}$$

For a constant temperature, the measurements plotted on a graph of \textbf{pressure against $\dfrac{\mathbf{1}}{\text{volume}}$} is a straight line through the origin.

Any change at constant temperature is called an \textbf{isothermal change}.

\subsubsection*{Charles' Law}

For a \textbf{fixed mass} of gas at \textbf{constant pressure}. Charles' law states the relation between volume and temperature in kelvins can be written as
$$\frac{V}{T}=\text{constant}$$

For a constant pressure, the measurements plotted on a graph of \textbf{volume against temperature} is a straight line through the origin - no matter how much gas is used, the volume of an ideal gas is zero at \textbf{absolute zero}.

Any change at constant pressure is called an \textbf{isobaric change}.
\begin{itemize}
    \item When work is done to change the volume of gas, energy must be \textbf{transferred by heating} to keep the pressure constant.
\end{itemize}
$$\text{Work done}=p\Delta V$$

\subsubsection*{The Pressure Law}

For a \textbf{fixed mass} of gas at \textbf{constant volume}, the pressure law states the relation between pressure and temperature can be written as
$$\frac{p}{T}=\text{constant}$$
