\subsection{The Ideal Gas Law}

The molecules of a gas \textbf{move at random with different speeds}, when a molecule collides with another molecule or with a solid surface, it bounces off \textbf{without losing speed}.

The pressure of a gas on a surface is due to the gas molecules hitting the surface.
\begin{enumerate}
    \item Each impact causes a tiny force on the surface.
    \item Because there is a very large number of impacts each second.
    \item The overall result is that the gas exerts a measurable pressure on the surface.
\end{enumerate}

Smoke particles \textbf{wriggle about unpredictably}
\begin{enumerate}
    \item Because it is \textbf{bombarded unevenly} and randomly by individual molecules.
    \item The particle therefore experiences forces due to these impacts.
    \item Which \textbf{change its magnitude and direction} at random.
\end{enumerate}
This type of motion is called \textbf{Brownian motion}, it showed the existence of molecules and atoms.

\subsubsection*{Molar Mass}

\begin{itemize}
    \item The \textbf{Avogadro constant} $N_A$ is defined as the number of atoms in exactly 12g of the carbon isotope $^{12}_{\hspace{3pt}6}C$.
        $$N_A=6.02\times10^{23}$$
    \item One \textbf{atomic mass unit} $u$ is $\dfrac{1}{12}$ of the mass of a $^{12}_{\hspace{3pt}6}C$ atom.
        $$1u=1.66\times10^{-27}\text{kg}$$
    \item One \textbf{mole} of substance consists of identical particles is defined as the quantity of substance that contains $N_A$ particles.
    \item The number of moles in a given quantity is its \textbf{molarity}.

        The unit of molarity is the \textbf{mol}.
    \item The \textbf{molar mass} of a substance is the mass of 1 mol of that substance.

        The unit of molar mass is kg\,mol$^{-1}$
\end{itemize}
\begin{align*}
    \text{Number of moles}\ n&=\frac{\text{mass of substance}\ M_S}{\text{molar mass of substance}\ M}\\
    \text{Number of molecules}&=nN_A
\end{align*}

\subsubsection*{The Ideal Gas Equation}

An ideal gas is a \underline{gas that obeys Boyle's law}.

The three experimental gas laws can be combined to give the equation
$$\frac{pV}{T}=\text{constant}$$
where $T$ is the absolute temperature.

\textbf{Equal volumes} of ideal gases at the same temperature and pressure contains \textbf{equal number of moles}.

For 1 mol of any ideal gas, the value of $\dfrac{pV}{T}=8.61$J\,mol$^{-1}$K$^{-1}$, this value is called the \textbf{molar gas constant} $R$ - the graph of $pV$ against against $T$ for $n$ moles is a \textbf{straight line through absolute zero} and has a gradient equal to $nR$.

For $n$ moles of ideal gas
$$pV=nRT$$
This equation is called the \textbf{ideal gas equation}.

The \textbf{Boltzmann constant} $k=\dfrac{R}{N_A}=1.38\times10^{-23}$JK$^{-1}$
\begin{align*}
    pV&=nRT\\
      &=\frac{N}{N_A}RT\\
      &=N\frac{R}{N_A}T\\
      &=NkT
\end{align*}

\subsubsection*{Density of Ideal Gas}
Since the mass of a substance $M_S=\text{molar mass}\ M\times n$, and $n=\dfrac{pV}{RT}$.
$$M_S=M\times\frac{pV}{RT}$$

And the \textbf{density} of an ideal gas with molar mass $M$
$$\rho=\frac{M_S}{V}=\frac{pM}{RT}$$

Showing for an ideal gas at \textbf{constant pressure}
$$\rho\propto\frac{1}{T}$$
