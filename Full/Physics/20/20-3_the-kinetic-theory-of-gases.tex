\subsection{The Kinetic Theory of Gases}

Experimental laws can be explained by assuming the gas consists of \textbf{point molecules} moving about at random, continually colliding with the container walls.

\begin{itemize}
    \item \textbf{Boyle's law}: The pressure of gas at constant temperature is increased by reducing volume because the gas molecules \textbf{travel less distance between impacts} at the wall due to reduced volume.

        There are more impacts per second, so pressure is greater.

    \item \textbf{Charles' law}: The pressure of gas at constant volume is increased by raising temperature. The \textbf{average speed of the molecules} is increased by raising the gas temperature.

        The impacts of molecules on the container walls are \textbf{harder and more frequent}, so the pressure is raised as a result.
\end{itemize}

Molecules in an ideal gas have a \textbf{continuous spread of speeds}.
\begin{itemize}
    \item The speed of an individual molecule changes when it collides with another gas molecule.
    \item But the \textbf{distribution stays the same}, as long as the temperature doesn't change.
\end{itemize}

The \textbf{root mean square} speed of the molecules
$$c_\text{rms}=\sqrt{\frac{{c_1}^2+{c_2}^2+\cdots+{c_N}^2}{N}}$$
Where $c_1,c_2,\dots$ represent the speeds of the individual molecules.

If the \textbf{temperature is raised}.
\begin{itemize}
    \item The molecules \textbf{move faster on average}, so the root mean square speed increases.
    \item The distribution curve becomes \textbf{flatter and broader} because more molecules are moving at higher speeds.
\end{itemize}

\subsubsection*{The Kinetic Theory Equation}

Assumptions made about the molecules in a gas
\begin{itemize}
    \item Molecules are \textbf{point particles} - the volume of each molecule is negligible compared with the volume of the gas.
    \item They \textbf{do not attract each other} - if they did the force of their impacts on the container surface would be reduced.
    \item They move about in \textbf{continual random motion}.
    \item The collisions they undergo with each other and with the container surface are \textbf{elastic collisions}.
    \item Each collision with the container surface is of much shorter duration than the time between impacts.
\end{itemize}

Consider \textbf{one molecule} of mass $m$ in a rectangular box.
\begin{enumerate}
    \item Each impact of the molecule with the (shaded) surface \textbf{reverses the x-component of velocity}. So the change in momentum is
        $$\Delta p=2mv_x$$
    \item The \textbf{time between successive impacts} on this face is
        $$\Delta t=\frac{2L_x}{v_x}$$
    \item The \textbf{force on the molecule}, and by Newton's third law - on the surface is
        $$F=\frac{\Delta p}{\Delta t}=\frac{m{v_x}^2}{L_x}$$
    \item The pressure of the molecule on the surface is
        $$p=\frac{F}{A}=\frac{m{v_x}^2}{L_xL_yL_z}=\frac{m{v_x}^2}{V}$$
    \item For $N$ molecules in the box moving at different velocities, the \textbf{total pressure} is the sum of the individual pressures $p_1,p_2,\cdots,p_N$.
        \begin{align*}
            p&=\frac{m{v_{x1}}^2}{V}+\frac{m{v_{x2}}^2}{V}+\cdots+\frac{m{v_{xN}}^2}{V}\\
             &=\frac{m}{V}({v_{x1}}^2+{v_{x2}}^2+\cdots+{v_{xN}}^2)\\
             &=\frac{Nm\bar{v_x}^2}{V}
        \end{align*}
        where $\bar{v_x}=\dfrac{{v_{x1}}^2+{v_{x2}}^2+\cdots+{v_{xN}}^2}{N}$
    \item Because the motion of the molecules is random, there is \textbf{no preferred direction} of motion. The equation above is also true for
        $$p=\frac{Nm\bar{v_y}^2}{V}=\frac{Nm\bar{v_z}^2}{V}$$
    \item Since ${c_\text{rms}}^2=\bar{v_x}^2+\bar{v_y}^2+\bar{v_z}^2$
        \begin{align*}
            3p&=\frac{Nm}{V}(\bar{v_x}^2+\bar{v_y}^2+\bar{v_z}^2)\\
            pV&=\frac{1}{3}Nm{c_\text{rms}}^2
        \end{align*}
\end{enumerate}

\subsubsection*{Temperature and Internal Energy}

The \textbf{mean kinetic energy} of a molecule of a gas
$$\bar{E_K}=\frac{1}{2}m{c_\text{rms}}^2$$
The higher the temperature of a gas, the greater the mean kinetic energy of a molecule of the gas.

Assuming the \textbf{mean kinetic energy} of a molecule is given by
$$\frac{1}{2}m{c_\text{rms}}^2=\frac{3}{2}kT$$

Then
\begin{align*}
    m{c_\text{rms}}^2&=3kT\\
    \intertext{Substitute into}
                  pV&=\frac{1}{3}Nm{c_\text{rms}}^2\\
                    &=NkT
\end{align*}

For an ideal gas at temperature $T$, the \textbf{mean kinetic energy} of a molecule of an ideal gas is $\dfrac{3}{2}kT$.

The \textbf{total kinetic energy} for an ideal gas is its \textbf{internal energy}
$$\text{Internal energy}=\frac{3}{2}NkT=\frac{3}{2}nRT$$
