\subsection{Moving Charges in a Magnetic Field}

\begin{itemize}
    \item Each electron in the beam experiences a force due to the magnetic field.
    \item The beam \textbf{follows a circular path} because the direction of the force on each electron is \textbf{perpendicular to the motion} of the electron.
\end{itemize}

A current-carrying wire in a magnetic field experiences a force is that the electrons moving along the wire are \textbf{pushed to one side} by the force of the field.

Magnetic fields are used in particle physics detectors to \textbf{separate different charged particles} out, and measure their momentum from the curvature of the tracks they create.

For a charge moving in a magnetic field
\begin{itemize}
    \item In time $t$, it travels distance $l=vt$.
    \item Its passage is equivalent to current $I=\dfrac{Q}{t}$
\end{itemize}
$$F=BIl=B\left(\frac{Q}{t}\right)(vt)=BQv$$

If the direction of motion of a charged particle in a magnetic field is at angle $\theta$ to the lines of the field, then the component of $B$ perpendicular to the motion of the charged particle is $B\sin\theta$, giving
$$F=BQv\sin\theta$$

\subsubsection*{The Hall Probe}

Hall probes are used to \textbf{measure magnetic flux density}. Consists of
\begin{itemize}
    \item A slice of semiconducting material.
    \item A magnetic field \textbf{perpendicular to the flat side} of the semiconductor.
    \item A \textbf{constant current} passes through the slice.
\end{itemize}

The charge carriers are deflected by the field, creating a potential difference between the top and bottom edges of the slice. This is known as the \textbf{Hall effect}.

The \textbf{Hall's voltage} is proportional to magnetic flux density.

Once the voltage is created,
\begin{align*}
    F_\text{mag}&=BQv\\
    F_\text{elec}&=\frac{QV_h}{d}\\
    \frac{QV_h}{d}&=BQv\\
    V_h&=BQd
\end{align*}
which is proportional to $B$.
