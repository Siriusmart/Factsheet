\subsection{Charged Particles in Circular Orbits}

The force of the magnetic field on a moving charged particle is \textbf{at right angles} to the direction of motion of the particle.
\begin{itemize}
    \item \textbf{No work is done} by the magnetic field on the particle, as the force \textbf{always acts at right angles} to the velocity of the particle.
    \item The force causes a \textbf{centripetal acceleration} because it is perpendicular to the velocity. The path is a \textbf{complete circle} because the magnetic field is uniform and the particle remains in the field.
\end{itemize}

\begin{align*}
    BQv&=\frac{mv^2}{r}\\
    r&=\frac{mv}{BQ}
\end{align*}

\begin{itemize}
    \item $r$ decreases if $B$ is increased or if $v$ is decreased.
    \item If particles with a \textbf{larger specific charge} $\dfrac{Q}{m}$ are used.
\end{itemize}

\subsubsection*{The Cyclotron}

A cyclotron is used to \textbf{produce high-energy beams}.

\begin{itemize}
    \item Two hollow \textbf{D-shaped electrodes} in a vacuum chamber.
    \item A \textbf{uniform magnetic field} applied perpendicular to the plane of the electrodes.
    \item A \textbf{high-frequency alternating voltage} is applied between the electrodes.
\end{itemize}

Charged particles are directed into one of the electrodes near the centre of the cyclotron.
\begin{enumerate}
    \item The charged particles are \textbf{forced on a circular path} by the magnetic field, causing them to emerge from the electrode they were directed into.
    \item As they cross into the other electrode, the alternating voltage reverses so they are \textbf{accelerated into the other electrode} where they were once again forced on a circular path by the magnetic field.
\end{enumerate}

This occurs because the time taken by a particle to move around its semi-circular path in each electrode \textbf{does not depend on the particle speed}.
\begin{align*}
    r&=\frac{mv}{BQ}\\
    T_\text{semi-circle}&=\frac{\pi r}{v}\\
                        &=\frac{m\pi}{BQ}\\
    T&=\frac{2m\pi}{BQ}
\end{align*}

\subsubsection*{The Mass Spectrometer}

A mass spectrometer is used to analyse the \textbf{type of atoms present} in a sample.
\begin{enumerate}
    \item The atoms of the sample are \textbf{ionised} and directed in a narrow beam \textbf{at the same velocity} into a \textbf{uniform magnetic field}.
    \item Each ion is \textbf{deflected in a semi-circle} my the magnetic field onto a detector.
    \item The radius of curvature \textbf{depends on the specific charge} of the ion in accordance with $r=\dfrac{mv}{BQ}$.
    \item The detector is linked to a computer to show the \textbf{relative abundance} of each type of ion in the sample.
\end{enumerate}

