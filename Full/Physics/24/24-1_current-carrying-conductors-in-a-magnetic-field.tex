\subsection{Current Carrying Conductors in a Magnetic Field}

A \textbf{magnetic field} is a force field surrounding a \textbf{magnet} or \textbf{current-carrying wire} which acts on any other magnets or current-carrying wire placed in the field.
\begin{itemize}
    \item The magnetic field of a bar magnet is \textbf{strongest at its ends} referred to as north-seeking and south-seeking poles.
    \item A \textbf{magnetic field line} of a magnetic field is a line along which a \textbf{north pole would move} in the field.
\end{itemize}

The \textbf{Earth's magnetic field} is caused by circulation currents in the molten iron core.

\subsubsection*{The Motor Effect}

A current-carrying wire placed at a \textbf{non-zero angle} to the field lines of an external magnetic field experiences a force due to the field, this effect is known as the \textbf{motor effect}.

The force is perpendicular to the wire and the field lines.

The magnitude of the force depends on the
\begin{itemize}
    \item Current
    \item Magnetic flux density
    \item Length of the wire
    \item The angle between the field lines and the current direction.

        The force is greatest when the wire is at right angles to the magnetic field, zero when the wire is parallel to the magnetic field.
\end{itemize}

Observations show that the force $F$ on the wire is proportional to the \textbf{current} and the \textbf{length of the wire}.
\begin{itemize}
    \item The \textbf{magnetic flux density} of the magnetic field is defined as the force per metre length per unit current on a current-carrying conductor at \textbf{right angles} to the magnetic field lines.
    \item For a wire of length $l$ carrying a current $I$ in a uniform magnetic field $B$ at \textbf{90 deg to the field lines}.
        $$F=BIl$$
\end{itemize}
The unit of $B$ is the \textbf{tesla}, equal to 1Nm$^{-1}$A$^{-1}$.

For a straight wire at angle $\theta$ to the magnetic field lines, the force on the wire is due to the \textbf{component of magnetic field perpendicular to the wire}.
$$F=BIl\sin\theta$$

\subsubsection*{Couple on a Coil}

Consider a \textbf{rectangular current-carrying coil} of $n$ turns, and can rotate about a vertical axis.
\begin{itemize}
    \item The long sides of the coil are vertical, each experiences a force $F=n(BIl)$.
    \item The pair of forces acting \textbf{forms a couple} as they are not directed along the same line.
        $$\tau=Fd$$
        where $d$ is the \textbf{perpendicular distance} between the line of action of forces.
    \item If the plane of coil is at angle $\alpha$ to the field lines, $d=w\cos\alpha$ where $w$ is the width of the coil.
    \item Therefore, $\tau=Fw\cos\alpha=nBIlw\cos\alpha=nBIA\cos\alpha$.
\end{itemize}
