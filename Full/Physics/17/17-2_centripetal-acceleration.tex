\subsection{Centripetal Acceleration}

\begin{itemize}
    \item The velocity of an object in \textbf{uniform circular motion} at any point along its path is \textbf{tangent to the circle at that point}.
    \item The \textbf{direction of velocity changes} continually as the object moves along the circular path.
    \item This change in direction of velocity is \textbf{towards the centre of the circle}.
\end{itemize}

This acceleration is called \textbf{centripetal acceleration}.
$$a=\frac{v^2}{r}=\omega^2r$$

\subsubsection*{Proof}
\begin{enumerate}
    \item Consider an object in circular motion at speed $v$ moving in  a short time interval $\delta t$ from position A to B along the perimeter of a circle of radius $r$.
        $$\text{AB}=\delta s=v\delta t$$
    \item The line from the object to the centre of the circle at C turns through angle $\delta\theta$ when the object moves from A to B, the velocity direction of the object turns through the same angle $\delta\theta$.
    \item The change of velocity $\delta v$ is $v_B-v_A$.
    \item The triangle ABC and the velocity vector triangle have the same shape because they both have two sides of equal length with the same angle $\delta\theta$ between the two sides.
\end{enumerate}
When $\delta v$ is small
\begin{align*}
    \frac{\delta v}{v}&=\frac{\delta s}{r}\\
                      &=\frac{v\delta t}{r}\\
    \frac{\delta v}{\delta t}&=\frac{v^2}{r}
\end{align*}

\subsubsection*{Centripetal Force}

To make an object move on a circular path, it must be acted on by a resultant force that \textbf{changes its direction of motion}.

The \textbf{resultant force} on an object moving around a circle at constant speed is called the \textbf{centripetal force}, it acts in the same direction as the centripetal acceleration - towards the centre of the circle.
\begin{itemize}
    \item For an object swung around the end of a string, the \textbf{tension in the string} provides the centripetal force.
    \item For a satellite moving around the Earth, the \textbf{force of gravity} between the satellite and the Earth is the centripetal force.
    \item For a planet moving around the sun, the \textbf{gravity on the planet} due to the sun is the centripetal force.
    \item For a capsule on a ferris wheel, the \textbf{resultant of the support force and its weight} is the centripetal force.
    \item For charged particles travelling through a magnetic field in a circular path, the \textbf{magnetic force} on the moving charged particles is the centripetal force.
\end{itemize}
$$F=\frac{mv^2}{r}=m\omega^2r$$
