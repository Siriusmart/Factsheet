\subsection{The Principle of Moments}

The \textbf{moment of a force} about any point is defined as the force $\times$ perpendicular distance from the line of action of the force to the point.
\begin{itemize}
    \item The unit of the moment of a force is the newton metre (Nm).
\end{itemize}
$$\text{The moments of a force}=F\times d$$

\begin{itemize}
    \item An object that is not a point object is referred to as a \textbf{body}.
    \item Such object turns if a force is applied to it anywhere other than through its \textbf{centre of mass}.
    \item The \textbf{centre of mass} of a body is the point through which a single force on the body has no turning effect.
        \begin{itemize}
            \item It is the point where we consider the weight of the body to act when study the effect of forces on the body.
        \end{itemize}
\end{itemize}

The \textbf{principle of moments} states if a body is \underline{acted on by more than one force} and it is in equilibrium, the \textbf{turning effects of the forces} must balance out.
Consider the moments of the forces about \underline{any point}
$$\text{sum of clockwise moments}=\text{sum of anticlockwise moments}$$
