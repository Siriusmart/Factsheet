\subsection{Vectors and Scalars}

\begin{itemize}
    \item A \textbf{vector} is any physical quantity that has a direction as well as a magnitude.
        \begin{itemize}
            \item Displacement, velocity, acceleration, force.
        \end{itemize}
    \item A \textbf{scalar} is any physical quantity that is not directional.
        \begin{itemize}
            \item Mass, density, volume, energy.
        \end{itemize}
\end{itemize}
A vector can be \textbf{represented as an arrow} - the length of the arrow represents the magnitude of the vector quantity, the direction arrow gives the direction of the vector.

\textbf{Distance travelled} depends on the route, whereas the \textbf{direct distance} is always the same.
\begin{itemize}
    \item \textbf{Displacement} is distance in a given direction.
    \item \textbf{Velocity} is speed in a given direction.
\end{itemize}

\subsubsection*{Vector Addition}

Vectors can be added using a \textbf{scale diagram}.
$$OB=OA+AB$$
Vector addition gives the \textbf{overall effect} of the vectors. Adding two forces gives the \textbf{resultant} of the forces.
\begin{itemize}
    \item The \textbf{resultant} is the combined effect of two forces.
\end{itemize}

Vectors can also be added using a \textbf{calculator}.

In general, if the two perpendicular forces are $F_1$ and $F_2$
\begin{itemize}
    \item The \textbf{magnitude} of the resultant $F=\sqrt{{F_1}^2+{F_2}^2}$
    \item The angle $\theta$ between the resultant and $F_1$ is given by $\tan\theta=F_2/F_1$.
\end{itemize}

\subsubsection*{Resolving a Vector into Two Perpendicular Components}

Is the process of working out the \textbf{components of a vector} in two perpendicular directions given the magnitude and direction of the vector.

A force $F$ can be resolved into two perpendicular components
\begin{itemize}
    \item $F\cos\theta$ parallel to a line at angle $\theta$ to the line of action of the force.
    \item $F\sin\theta$ perpendicular to the line.
\end{itemize}
