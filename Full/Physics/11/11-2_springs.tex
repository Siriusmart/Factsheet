\subsection{Springs}

A \textbf{stretched spring} exerts a pull on the object holding each end of the spring, this full is referred to as the \textbf{tension} of the spring.
\begin{itemize}
    \item The tension in the spring is equal and opposite to the force needed to stretch the spring.
\end{itemize}

\textbf{Hook's law} states that the force needed to stretch a spring is directly proportional to the extension of the spring from its natural length.
$$F=k\Delta L$$
where $k$ is the \textbf{string constant} and $\Delta L$ the extension from its natural length $L$.
\begin{itemize}
    \item $[k]=\text{Nm}^{-1}$, the greater the value of $k$, the stiffer the spring is.
    \item The graph of $F$ against $\Delta L$ is a straight line of gradient $k$ through the origin.
\end{itemize}

If a spring is stretched beyond its \textbf{elastic limit}, it will not regain its initial length when the force applied to it is removed.

\subsubsection*{Springs in Parallel}

If weight is supported by two springs $P$ and $Q$ in parallel, where the extension $\Delta L$ of each spring is the same.
\begin{itemize}
    \item $F_P=k_P\Delta L$
    \item $F_Q=k_Q\Delta L$
\end{itemize}

Since the weight is supported by both springs.
$$W=F_P+F_Q=k_P\Delta L+k_Q\Delta L=k\Delta L$$
giving \textbf{effective spring constant} $k=k_P+k_Q$

\subsubsection*{Springs in Series}

If a weight is supported by two springs joined end-on in series with each other, the tension in each spring is the same and equal to $W$.

\begin{itemize}
    \item $\Delta L_P=\dfrac{W}{k_P}$
    \item $\Delta L_Q=\dfrac{W}{k_Q}$
\end{itemize}

Therefore total extension
$$\Delta L=\Delta L_P+\Delta L_Q=\frac{W}{k_P}+\frac{W}{k_Q}=\frac{W}{k}$$
giving \textbf{effective spring constant}
$$\frac{1}{k}=\frac{1}{k_P}+\frac{1}{k_Q}$$

\subsubsection*{Elastic Potential Energy}

Elastic potential energy is the energy stored in a stretched spring.
\begin{itemize}
    \item If a spring is released, the elastic energy stored in it is \textbf{transferred into kinetic energy} of the spring.
\end{itemize}

The \textbf{work done} to stretch a spring by extension $\Delta L$ from its unstretched length is $\dfrac{1}{2}F\Delta L$. Giving
$$E_P=\frac{1}{2}F\Delta L=\frac{1}{2}k\Delta L^2$$
