\subsection{More about Stress and Strain}

\subsubsection*{Loading/unloading Curves}

Loading/unloading curves are used to study how the strength of a material changes as a result of being stretched.

\begin{enumerate}
    \item The tension in a strip of material is increased by \textbf{increasing the weight} it support in steps.
    \item At each step, the extension of the material is measured.
    \item The measurements can be plotted as loading/unloading curves.
\end{enumerate}

\begin{itemize}
    \item For a \textbf{metal wire} - the loading and unloading curves are the same, provided its \textbf{elastic limit} is not exceeded, so the wire \textbf{returns to its original length} when unloaded.
        \begin{itemize}
            \item Beyond its elastic limit, the unloading line is \textbf{parallel to the loading line}.
            \item In this case the wire is slightly longer when unloaded - it has a \textbf{permanent extension}.
        \end{itemize}
    \item For a \textbf{rubber band} - the rubber band returns to the same unstretched length, but the unloading curve is \textbf{below the loading curve} except at zero and maximum extensions.
        \begin{itemize}
            \item The rubber band \textbf{remains elastic} as it regains its initial length.
            \item But it has a \textbf{low limit of proportionality}.
        \end{itemize}
    \item For a \textbf{polythene strip}, the extension during unloading is \textbf{greater than during loading} - the strip does not return to the same initial length when it is completely unloaded.
        \begin{itemize}
            \item The polythene strip has a \textbf{low limit of proportionality}.
            \item It suffers \textbf{plastic deformation}.
        \end{itemize}
\end{itemize}

\subsubsection*{Strain Energy}

The work done to deform an object is referred to as strain energy. The area under the line of a force-extension graph is equal to the work done to stretch the wire.

\subsubsection*{Metal Wire}

Provided the \textbf{limit of proportionality} is not exceeded, the stretch a wire to an extension $\Delta L$, the work done is $\dfrac{1}{2}T\Delta L$
$$\text{Elastic energy stored in a stretched wire}=\frac{1}{2}T\Delta L$$
Because the graph of tension against extension is the same for loading and unloading, \textbf{all energy stored in the wire can be recovered} when the wire is unloaded.

\subsubsection*{Rubber Band}

\begin{itemize}
    \item Work done to stretch the rubber band is represented by the area under the loading curve.
    \item Work done by the rubber band when it is unloaded is represented by the area under the unloading curve.
\end{itemize}

The \textbf{area between the loading and unloading curve} represents the difference between the energy stored in the rubber band when it is stretched and the useful energy recovered from it when it is unstretched. 

The difference occurs because some of the energy stored in the rubber band becomes the \textbf{internal energy of the molecules} when the rubber band unstretches.

\subsubsection*{Polythene}

As it does not regain its initial length, the areabetween the loading and unloading curves represents the internal enregy retained by the polythene when it unstretches.
