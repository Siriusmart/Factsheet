\subsection{Deformation of Solids}

The \textbf{elasticity} of a solid material is its ability to regain its shape after it has been deformed or distorted and the forces that deformed it have been released.
\begin{itemize}
    \item Deformations that stretches an object is \textbf{tensile}.
    \item Deformations that compresses an object is \textbf{compressive}.
\end{itemize}

Measurements for \textbf{tensile-extension graphs}.
\begin{enumerate}
    \item A material is held at its upper end and \textbf{loaded by hanging weights} at its lower end.
    \item A set square attached to the bottom of the weights to \textbf{measure extension} of the material.
    \item The weight of the load is \textbf{increased in steps} then \textbf{decreased to zero}.
    \item The \textbf{extension} of the strip of material at each step is its increase of length from its unloaded length.
\end{enumerate}
The measurements can be plotted as a \textbf{tension-extension graph}.
\begin{itemize}
    \item A \textbf{steel spring} gives a \underline{straight line} in accordance with Hooke's law.
    \item A \textbf{rubber band} extends easily when stretched, but becomes \textbf{fully stretched} and very difficult to stretch further when it has been lengthened considerably.
    \item A \textbf{polythene strip} stretches easily after its initial stiffness is overcome. But after `giving' easily, it extends little and becomes difficult to stretch.
\end{itemize}

For a wire of length $L$ and cross section area $A$ under tension.
\begin{align*}
    \text{Tensile stress}\ \sigma&=\frac{T}{A}\qquad[\sigma]=\text{Nm}^{-2}\\
    \text{Tensile strain}\ \varepsilon&=\frac{\Delta L}{L}\quad\ \text{$\varepsilon$ is a ratio and has no unit}
\end{align*}

\begin{enumerate}
    \item From 0 to the \textbf{limit of proportionality}, tensile stress is proportional to tensile strain - the value of stress/strain is constant and known as the \textbf{Young's modulus} of the material.
        $$\text{Young modulus}\ E=\frac{\sigma}{\varepsilon}=\frac{TL}{A\Delta L}$$
    \item Beyond the limit of proportionality, \underline{the line curves} and continue beyond the \textbf{elastic limit} to the \textbf{yield point} where the wire weakens temporarily.
        \begin{itemize}
            \item The \textbf{elastic limit} is the point beyond which the wire is permanently stretched and suffers \textbf{plastic deformation}.
        \end{itemize}
    \item Beyond the yield point, a small increase in tensile stress causes a large increase in tensile strain as the material of the wire undergoes \textbf{plastic flow}.
    \item Beyond the \textbf{ultimate tensile stress}, the wire loses its strength, extends and \textbf{becomes narrower} at its weakest point. Increase of tensile stress occurs due to the reduce cross section area at this point until the wire breaks.
\end{enumerate}

The \textbf{ultimate tensile stress} is the maximum tensile stress, also called the \textbf{breaking stress}.

\subsubsection*{Stress-strain Curves}

The \textbf{stiffness} of different materials can be compared using the gradient of the stress-strain line - equal to the \textbf{Young's modulus} of the material.
\begin{itemize}
    \item A \textbf{brittle} material snaps without giving any noticeable yield.
    \item A \textbf{ductile} material can be drawn into a wire.
\end{itemize}
