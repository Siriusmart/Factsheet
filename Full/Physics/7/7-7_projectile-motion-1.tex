\subsection{Projectile Motion 1}

A \textbf{projectile} is any object acted on only by the force of gravity.
\begin{itemize}
    \item The acceleration of the object is \textbf{always equal to $\mathbf{g}$} and is \textbf{always downwards}, because the force of gravity acts downwards.
        \begin{itemize}
            \item The acceleration only affects the \textbf{vertical motion} of the object.
        \end{itemize}
    \item The horizontal velocity of the object is constant because the acceleration of the object \underline{does not have a horizontal component}.
    \item The motions in the horizontal and vertical directions are \textbf{independent of each other}.
\end{itemize}
\begin{align*}
    v&=u-gt\\
    y&=ut-\frac{1}{2}gt^2
\end{align*}

\subsubsection*{Horizontal Projection}

If the initial projection of an object projected off a cliff is horizontal.
\begin{itemize}
    \item Its \textbf{path through the air} becomes steeper as it drops.
    \item The faster it is projected, the further away it will fall into the sea.
    \item The time taken for it to fall into the sea \textbf{does not depend on how fast it is projected}.
\end{itemize}

If ball A is released from rest and ball B is projected horizontally.
\begin{itemize}
    \item The \textbf{horizontal position} of B changes by equal distances per unit time.
        \begin{itemize}
            \item The \underline{horizontal component of B's velocity is constant}.
        \end{itemize}
    \item The \textbf{vertical position} of A and B changes at the same rate - at any instant A is at the same level as B.
        \begin{itemize}
            \item A and B has the same vertical component of velocity at any instant.
        \end{itemize}
\end{itemize}

For an object projected horizontally with initial velocity $U$.
\begin{align*}
    v_x&=Ut\\
    v_y&=\frac{1}{2}gt^2
\end{align*}
