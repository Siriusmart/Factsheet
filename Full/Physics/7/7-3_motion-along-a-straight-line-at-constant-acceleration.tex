\subsection{Motion Along a Straight Line at Constant Acceleration}

Consider an object that \textbf{accelerates uniformly} from initial velocity $u$ to final velocity $v$ in time $t$ without change of direction.

From the definition of acceleration $a=\dfrac{v-u}{t}$
\begin{equation}
    v=u+at
\end{equation}

From $s=\bar{v}\times t$ and $\bar{v}=\dfrac{u+v}{2}$
\begin{equation}
    s=\frac{(u+v)t}{2}
\end{equation}

Combining (1) and (2):
\begin{equation}
    s=ut+\frac{1}{2}at^2
\end{equation}

Multiply $a=\dfrac{v-u}{t}$ and $s=\dfrac{(u+v)t}{2}$
\begin{equation}
    v^2=u^2+2as
\end{equation}

\subsubsection*{Finding Displacement using a Velocity-time Graph}

Consider an object moving at \textbf{constant velocity}. The displacement in time $t$ is
$$s=vt$$
so the displacement is represented on graph by the \textbf{area under the line} between the start and time $t$ which is a rectangle with width $t$ and height $v$.

Consider an object moving at \textbf{constant acceleration} $a$.
$$s=\frac{(u+v)t}{2}$$
so the displacement is represented on graph by the \textbf{area under the curve} between the start and time $t$ which is a trapezium with area $(u+v)t/2$.

Consider an object moving at a \textbf{changing acceleration}.
\begin{enumerate}
    \item Let $v$ represent the velocity at time $t$ and $v+\delta v$ represent the velocity a short time later at $t+\delta t$.
    \item Because $\delta v$ is small compared with velocity $v$, the displacement $\delta s$ in the short time interval $\delta t$ is $v\delta t$.
    \item By considering the whole area under the line in strips of similar width, the \textbf{total displacement} from the start to time $t$ is represented by the sum of the area of every strip, which is the \textbf{total area under the line}.
\end{enumerate}

Therefore whatever the shape of the line of a velocity time graph.
$$\text{displacement}=\text{area under the line of a velocity-time graph}$$
