\subsection{Projectile Motion 2}

Any form of motion where an object experiences a \textbf{constant acceleration} in a different direction to its velocity will be like projectile motion.
\begin{itemize}
    \item The path of a ball rolling across an \textbf{inclined board}.
    \item The path of a beam of electrons directed between two oppositely charged parallel plates.
\end{itemize}
Their paths are \textbf{parabolic} because it is subjected to constant acceleration acting in a different direction to its velocity.

\subsubsection*{Drag Force}

A projectile moving through air experiences a force that \textbf{drags on it} because of the \textbf{resistance of the air} it passes through.
\begin{itemize}
    \item Caused by \textbf{friction between layers} of air near the projectile's surface where air flows over the surface.
    \item Acts in the \textbf{opposite direction to the direction of motion}
    \item Has a \textbf{horizontal component} that reduces both the horizontal speed of the projectile and its range.
    \item Has a \textbf{vertical component} that reduces the maximum height of the projectile if its initial direction is above the horizontal and makes its descent steeper than its ascent.
\end{itemize}

The \textbf{shape of the project} its affects the drag force, and may also cause a \textbf{lift force}.
\begin{itemize}
    \item The shape of the projectile causes the air to flow faster over the top of the object than underneath it.
    \item As a result, the pressure of the air on the top surface is less than that on the bottom surface.
    \item The pressure difference causes a lift force on the object.
\end{itemize}
E.g. the cross-sectional shape of an aircraft wing creates a lift force.

A \textbf{spinning ball} experiences a force due to the same effect, the force can be in any direction depending on \textbf{how the ball is made to spin}.
