\subsection{More about Resistance}

\subsubsection*{Resistors in Series}

Since resistors in series \textbf{pass the same current}. For two resistors $R_1$ and $R_2$ in series
\begin{itemize}
    \item $V_1=IR_1$
    \item $V_2=IR_2$
\end{itemize}
The total pd is given by
$$V=V_1+V_2=IR_1+IR_2$$
So the total resistance is given by
$$R=\frac{V}{I}=\frac{IR_1+IR_2}{I}=R_1+R_2$$

So the \textbf{total resistance} is equal to the sum of the individual resistances.
$$R=R_1+R_2+R_3+\cdots$$

\subsubsection*{Resistors in Parallel}

Since resistors in parallel have the \textbf{same pd}. For two resistors $R_1$ and $R_2$ in parallel.
\begin{itemize}
    \item $I_1=\dfrac{V}{R_1}$
    \item $I_2=\dfrac{V}{R_2}$
\end{itemize}

The \textbf{total current} through the combination is
$$I=I_1+I_2=\frac{V}{R_1}+\frac{V}{R_2}$$

Since total current is $I=\dfrac{V}{R}$
\begin{align*}
    \frac{V}{R}&=\frac{V}{R_1}+\frac{V}{R_2}\\
    \frac{1}{R}&=\frac{1}{R_1}+\frac{1}{R_2}
\end{align*}

So the \textbf{total resistance} is given by
$$\frac{1}{R}=\frac{1}{R_1}+\frac{1}{R_2}+\frac{1}{R_3}+\cdots$$

\subsubsection*{Resistance Heating}

The \textbf{heating effect of an electric current} in any component is due tot he resistance of the component. 
\begin{itemize}
    \item The charge carriers \textbf{repeatedly collide} with the positive ions of the conducting material.
    \item There is a \textbf{net transfer of energy} from the charge carriers to the positive ions as a result of these collisions.
    \item The force due to the pd across the material accelerates the charge carrier until it collides with another positive ions.
\end{itemize}

Since $V=IR$, $P=IV=I^2R=\dfrac{V^2}{R}$

\begin{itemize}
    \item So the \textbf{energy transferred to the object} by electric current in time $\Delta t$
        $$\Delta E=I^2R\Delta t$$
    \item The energy transfer per second to the component \textbf{does not depend on the direction} of the current.
\end{itemize}
