\subsection{Electromotive Force and Internal Resistance}

The internal resistance of a source is the \textbf{loss of potential difference per unit current} in the source when current passes through the source.
\begin{itemize}
    \item The internal resistance of a source is due to the \textbf{opposition to the flow of charge} through the source. This causes electrical energy produced by the source to be dissipated inside the source when the charge flows through it.
    \item The \textbf{electromotive force} of a source is the \textbf{electrical energy per unit charge} produced by the source.
        $$\varepsilon=\frac{\Delta E}{\Delta Q}$$
    \item The \textbf{terminal pd} of the source is the \textbf{electrical energy per unit charge} delivered by the source when it is in a circuit.

        The terminal pd is \textbf{less than the emf} whenever current passes through the source, the difference is due to the \textbf{internal resistance} of the source.
\end{itemize}
$$\varepsilon=IR+Ir$$

\begin{itemize}
    \item The \textbf{lost volts} inside the cell is equal to the difference between the cell emf and the terminal pd.
    \item The lost volts is the \textbf{energy per coulomb} dissipated inside the cell due to internal resistance.
\end{itemize}

\subsubsection*{Power}
$$P=I\varepsilon=I^2R+I^2r$$

\begin{itemize}
    \item The peak of the \textbf{power curve} is at $R=r$.
    \item When a source delivers power to a load, the \textbf{maximum power} is delivered when the load resistance is equal to the internal resistance of the source.
\end{itemize}

\subsubsection*{Measurement of Internal Resistance}

\begin{itemize}
    \item Connect a \textbf{voltmeter directly across the terminals} of the cell to measure the \textbf{terminal pd}.
    \item A \textbf{ammeter in series} to measure the \textbf{cell current}.
    \item A \textbf{variable resistor} to adjust the current.
    \item A \textbf{fixed resistor} to limit the \textbf{maximum current} that can pass through the cell.
\end{itemize}

Measurements of terminal pd and current for a cell can be plotted as graph.
\begin{itemize}
    \item The terminal pd is \textbf{equal to the cell emf at zero current}. Because lost volts is zero at zero current.
    \item The graph is a \textbf{straight line with negative gradient}.
\end{itemize}
$$V=\varepsilon-Ir$$

By \textbf{comparison with the standard equation for a straight line} $y=mx+c$.
\begin{itemize}
    \item The \textbf{y-intercept} is $\varepsilon$.
    \item The gradient of the line is $-r$.
\end{itemize}

\begin{itemize}
    \item For current $I_1$, $V_1=\varepsilon-I_1r$
    \item For current $I_2$, $V_2=\varepsilon-I_2r$
\end{itemize}

\begin{align*}
    V_1-V_2&=(\varepsilon-I_1r)-(\varepsilon-I_2r)\\
           &=(I_2-I_1)r\\
    r&=\frac{V_1-V_2}{I_2-I_1}
\end{align*}
