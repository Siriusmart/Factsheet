\subsection{The Potential Divider}

A potential divider consists of
\begin{itemize}
    \item Two or more resistors in series with each other.
    \item A source of \textbf{fixed potential difference}.
\end{itemize}

The potential difference of the source is \textbf{divided between components}.

A potential divider can
\begin{itemize}
    \item Supply a pd which is fixed at any value between zero and source pd.
    \item Supply a variable pd.
    \item Supply a pd that \textbf{varies with physical condition}.
\end{itemize}

\subsubsection*{Supply a Fixed PD}

Consider two resistors $R_1+R_2$ in series connected to a source of fixed pd $V_0$.
\begin{align*}
    I&=\frac{V_0}{R_1+R_2}\\
    V_2&=\frac{V0R_2}{R_1+R_2}
\end{align*}

The \textbf{ratio of pds} across each resistor is equal to the \textbf{resistance ratio} of the two resistors.
$$\frac{V_1}{V_2}=\frac{R_1}{R_2}$$

\subsubsection*{Supply a Variable PD}
\begin{itemize}
    \item A source pd connected to a \textbf{fixed length of uniform resistance wire}.
    \item Giving a variable pd between the \textbf{contact and one end of the wire}.
\end{itemize}

A \textbf{variable potential divider} is used for
\begin{itemize}
    \item a \textbf{simple audio volume control} to change the loudness of the sound of a loudspeaker, where the \textbf{audio signal is supplied as a potential difference}.
    \item To vary the \textbf{brightness of a light bulb} between zero and normal brightness.
\end{itemize}

\subsubsection*{Sensor Circuits}

A sensor circuit produces an output pd which \textbf{changes as a result of a change of physical variable}.
\begin{itemize}
    \item A \textbf{temperature sensor} is a potential divider made using a thermistor and a variable resistor.
    \item A \textbf{light sensor} is a potential divider made using a light-dependent resistor and a variable resistor.
\end{itemize}
