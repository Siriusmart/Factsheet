\subsection{Stable and Unstable Nuclei}

We know the \textbf{strong nuclear force} exist because the nuclei of a stable isotope do not disintegrate - there must be a force holding them together. The strong nuclear force
\begin{itemize}
    \item Overcomes the \textbf{electrostatic force of repulsion} between the protons in the nucleus.
    \item Keeps the protons and neutrons together.
\end{itemize}

The strong nuclear force has the same effect between any two \textbf{nucleons}.

It has a range \underline{no more than 3fm} - about the same as the diameter of a small nucleus. In contrast of the electrostatic force's infinite range. It is an \textbf{attractive force} from 3fm to 0.5fm, and a \textbf{repulsive force} smaller than that to prevent neutrons and protons being pushed into each other.

$1\text{fm}=10^{-15}\text{m}$

The equilibrium separation is where the \textbf{force curve} (picture it in your head) crosses the x-axis.

\underline{Naturally occurring} radioactive isotopes release three types of radiation.

\subsubsection*{Alpha radiation $\alpha$}
consists of \textbf{alpha particles} $^4_2\alpha$ each comprise of two protons and two neutrons.
$$^A_ZX\to\,^{A-4}_{Z-2}Y+^4_2\alpha$$
The product nucleus with $Z-2$ protons belongs to a different element $Y$.

\subsubsection*{Beta radiation $\beta$}
consists of fast-moving electrons know as \textbf{beta particles} $^{\hspace{6pt}0}_{-1}\beta$.

A neutron in the nucleus changes to a proton, a beta particle is created when the change happens and is emitted instantly, an \textbf{antineutrino} $\bar\nu$ is also emitted. The atomic number \textbf{increases by 1} to $Z+1$.
$$^A_ZX\to\,^{\hspace{10pt}A}_{Z+1}Y+\,^{\hspace{6pt}0}_{-1}\beta+\bar\nu$$
This type of change happens to nuclei with too many neutrons.

\subsubsection*{Gamma radiation $\gamma$}
is an electromagnetic radiation emitted by an unstable nucleus.
\begin{itemize}
    \item Passes through thick metal plates.
    \item Has no mass and no charge.
\end{itemize}
It is emitted by a \underline{nucleus with too much energy}, following an alpha or beta emission.

\subsubsection*{Discovery of the neutrino}

The energy spectrum of beta particles showed that beta particles were released with kinetic energies \underline{up to a maximum} depending on the isotope, where each unstable nucleus lost a certain amount of energy in the process.

Then either
\begin{itemize}
    \item Energy was not conserved in the change, or
    \item Some of it was carried away by mystery particles - called \textbf{neutrinos} and \textbf{antineutrino}.
\end{itemize}

The hypothesis was proven after 20 years, where antineutrinos were detected from their interaction with the \textbf{cadmium nuclei} in a tank of water, installed next to a nuclear reactor as a controllable source of neutrinos.

We now know billions of neutrinos from the sun pass through our bodies without interacting.
