\subsection{Inside the Atom}

Atoms can only be imaged with electron microscopes. Although we cannot see inside atoms we know from \textbf{Rutherford's alpha-scattering investigations} that every atom contains
\begin{itemize}
    \item A positively charged \textbf{nucleus} composed of protons and neutrons.
    \item \textbf{Electrons} surrounding the nucleus.
\end{itemize}

A \textbf{nucleon} is a proton or a neutron \underline{in the nucleus}.

Electrons are negatively charged, they are held in the atom by the electrostatic force of attraction between them and the nucleus.

\begin{itemize}
    \item The nucleus \underline{contains most of the mass} of the atom.
    \item Its diameter is of the order of 0.00001 times the diameter of the atom.
\end{itemize}

\begin{center}
\begin{tabular}{|c|c|c|c|c|}
    \hline
    & Charge/C & Relative charge & Mass/kg & Relative mass\\
    \hline
    proton & $+1.6\times10^{-19}$ & 1 & $1.6\times10^{-27}$ & 1\\
    neutron & $0$ & 1 & $1.6\times10^{-27}$ & 1\\
    electron & $-1.6\times10^{-19}$ & 1 & $9.1\times10^{-31}$ & 0.0005\\
    \hline
\end{tabular}
\end{center}
where relative charges and masses are relative to that of the proton.

Notice that
\begin{itemize}
    \item The electron has a much smaller mass than the proton or neutrons.
    \item The proton and neutron have almost equal mass.
    \item The electron has equal and opposite charge to the proton, the neutron is uncharged.
\end{itemize}

An \textbf{uncharged atom} has equal number of protons and electrons. An uncharged atom gains or loses electrons to become an \textbf{ion}.

\subsubsection*{Isotopes}

Every atom of a given element has the \underline{same number of protons}.
\begin{itemize}
    \item The \textbf{proton number} of an element is also called the \textbf{atomic number} $Z$.
    \item The \textbf{nucleon number} of the atom $A$ is the total number of protons and neutrons in an atom.
\end{itemize}
The nucleon number $A$ is sometimes called the \textbf{mass number} because it is approximately the mass of the atom in relative units, as the mass of a proton or neutron is approximately 1.

\textbf{Isotopes} are atoms with the same number of protons and different number of neutrons.

Each type of nucleus is called a \textbf{nuclide}, and is labelled using the \textbf{isotope notation}.
$$^A_ZX$$
where $X$ is the chemical symbol of the element. The number of neutrons is given by $A-Z$.

\subsubsection*{Specific Charge}

The \textbf{specific charge} of a charge particle is defined as its charge divided by mass
$$\text{specific charge}=\frac{\text{charge}}{\text{mass}}$$

The electron has the largest specific charge of any particle.
