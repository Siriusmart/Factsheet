\subsection{Satellite Motion}

Any small mass that orbits a larger mass is a \textbf{satellite}.

\begin{enumerate}
    \item The \textbf{gravitational attraction} between each planet and the Sun is the \textbf{centripetal force} that keeps the planet on its orbit.
    \item The gravitational force is given by $\dfrac{GMm}{r^2}$.
    \item The \textbf{gravitational field strength} therefore equals \textbf{centripetal acceleration}.
        \begin{align*}
            \frac{v^2}{r}&=\frac{GM}{r^2}\\
            v^2&=\frac{GM}{r}
        \end{align*}
    \item Since $v=\dfrac{2\pi r}{T}$, we have
        \begin{align*}
            \left(\frac{2\pi r}{T}\right)^2&=\frac{GM}{r}\\
            \frac{r^3}{T^2}&=\frac{GM}{4\pi^2}
        \end{align*}
\end{enumerate}
Because $\dfrac{GM}{4\pi^2}$ is the same for all the planets, then $\dfrac{r^3}{T^2}$ is the same for all the planets.

Therefore Kepler's third law can be proved by assuming the force of attraction varies with distance according to the \textbf{inverse-square law}.

\subsubsection*{Geostationary Satellites}

A geostationary satellite orbits the Earth \textbf{directly above the equator} and has a \textbf{time period of 24h}. It therefore remains in a \textbf{fixed position above the equator} because it has the same time period as the Earth's rotation.

The radius of orbit of a geostationary satellite can be calculated using the equation
$$\frac{r^3}{T^2}=\frac{GM}{4\pi^2}$$

\subsubsection*{Energy of Satellite}

Since the speed of a satellite is given by $v^2=\dfrac{GM}{r}$
\begin{align*}
    E_k&=\frac{1}{2}mv^2=\frac{GMm}{2r}\\
    E_p&=mV=-\frac{GMm}{r}
\end{align*}

Therefore the \textbf{total energy} of the satellite
$$E=E_p+E_k=-\frac{GMm}{2r}$$
