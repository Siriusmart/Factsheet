\subsection{Gravitational Potential}

\textbf{Gravitational potential energy} is the energy of an object due to its position in a gravitational field.
\begin{itemize}
    \item The position for zero GPE is at infinity - where the object is so far away that that \textbf{gravitational force on it is negligible}.
    \item At the surface, the GPE is negative as \textbf{work needs to be done} to escape from the field completely.
\end{itemize}

The \textbf{gravitational potential} at a point is the \textbf{work done per unit mass} to move a small test mass from infinity to that point.
$$V=\frac{W}{m}$$
The unit of gravitational potential is J\,kg$^{-1}$.

The work done to move a mass from $V_1$ to $V_2$ is equal to its \textbf{change of gravitational potential energy}.
$$\Delta W=m\Delta V$$

\begin{itemize}
    \item $\Delta E_p=mg\Delta h$ can only be applied for values of $\Delta h$ that are very small compared with the Earth's radius.
    \item $\Delta E_p=m\Delta V$ can always be applied.
\end{itemize}

\subsubsection*{Potential Gradients}

\textbf{Equipotentials} are surfaces of \textbf{constant potential}, so \textbf{no work needs to be done} to move along an equipotential surface.

\begin{itemize}
    \item The equipotentials near the Earth are \textbf{circles}.
    \item At increasing distance from the surface, the gravitational field becomes weaker, so the \textbf{gain of GPE per metre height} becomes less.
    \item Away from the Earth's surface, the equipotentials for equal increases of potential are \textbf{spaced further apart}.
\end{itemize}
But near the surface \textbf{over a small region}, the equipotentials are horizontal. This is because the gravitational field over a small region is uniform.

The \textbf{potential gradient} at a point in a gravitational field is the \textbf{change of potential per metre} at that point.

For a change of potential $\Delta V$ over a small distance $\Delta r$
$$\text{Potential gradient}=\frac{\Delta V}{\Delta r}$$

Consider a test mass $m$ being moved away from a planet. To move $m$ by a small distance $\Delta r$ in the opposite direction to the gravitational force $F_\text{grav}$ on it, its gravitational potential energy is increased
\begin{itemize}
    \item By an equal and opposite force $F$ acting through the distance $\Delta r$.
    \item By an equal amount of energy equal to the work done by $F$
        $$\Delta W=F\Delta r$$
\end{itemize}
\begin{align*}
    \Delta V&=\frac{F\Delta r}{m}\\
    F&=\frac{m\Delta V}{\Delta r}\\
    F_\text{grav}&=-F\\
    g&=\frac{F_\text{grav}}{m}=-\frac{\Delta V}{\Delta r}
\end{align*}

So gravitational field strength is the \textbf{negative of the potential gradient}. Where the minus sign shows that $g$ acts in the opposite direction to the potential gradient.
\begin{itemize}
    \item The closer the equipotentials are, the greater the potential gradient and the \textbf{stronger the field} is.
    \item Where the equipotentials show equal changes of potential for equal changes of spacing, the \textbf{potential gradient is constant}, so the gravitational field strength is constant and the field is \textbf{uniform}.
    \item The gradient is always at \textbf{right angles to the equipotentials}, so the field lines are always perpendicular to equipotentials.
\end{itemize}
