\subsection{Gravitational Field Strength}

\textbf{Any two masses} exerts a gravitational pull on each other, but usually the force is too weak to be noticed unless at least one of the masses is very large.

The force field around a mass is called a \textbf{gravitational field strength}.
\begin{enumerate}
    \item The mass of an object \textbf{creates a force field} around itself.
    \item Any other mass placed in the field is \textbf{attracted towards the object}.
    \item The second mass also has a force field around itself that \textbf{pulls on the first object} with an equal force in the opposite direction.
\end{enumerate}

The \textbf{magnetic field strength} $g$ is the force per unit mass on a small test mass placed in the field. The path which the smaller mass would follow is called a \textbf{field line} or \textbf{line of force}.
$$g=\frac{F}{m}$$
The unit of gravitational field strength is the \textbf{newton per kilogram}.

The test mass needs to be small, otherwise it might pull so much on the other object that it changes its position and alters the field.

\subsubsection*{Free Fall}

The weight of an object is the force of gravity on it, $F=mg$ for an object of mass $m$ in a gravitational field.
$$a=\frac{F}{m}=\frac{mg}{m}=g$$
Therefore $g$ is the acceleration of a freely falling object.

The object is described as \textbf{unsupported} because it is acted on by the force of gravity alone.

\subsubsection*{Field Patterns}

\begin{itemize}
    \item A \textbf{radial field} is where the field lines are \textbf{always directed to the centre}, since the force of gravity on a small mass near a much bigger spherical mass is always directed to the centre of the larger mass, regardless of position.

        The magnitude of $g$ in a radial field decreases with increasing distance from the massive body.
    \item A \textbf{uniform field} is where the gravitational field strength is the \textbf{same in magnitude and direction throughout the field}.

        The field lines are therefore \textbf{parallel} to one another and \textbf{equally spaced}.
\end{itemize}

The gravitational field strength of the Earth is radial because it falls with increasing distance. But \textbf{over small distances} compared to the Earth's radius, the change in gravitational field strength is insignificant so the field can be \textbf{considered uniform}.
