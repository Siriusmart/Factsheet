\subsection{The Laws of Electromagnetic Induction}

A magnetic field is produced in and around a coil when a current is passed through it. The pattern of the magnetic field lines is \textbf{like the pattern for a bar magnet}. The field lines pass through the solenoid and loop round outside the solenoid from one end to the other end.
\begin{itemize}
    \item Current passes anticlockwise round the north pole end.
    \item Current passes clockwise round the south pole end.
\end{itemize}

\subsubsection*{Lenz's Law}

Lenz's law states that the direction of the induced current is always such as to oppose the change that causes the current.

The explanation to Lenz's law is that energy is never created or destroyed - the induced current would never be in a direction to help the change that causes it, as it would produce electrical energy out of nowhere.

\subsubsection*{Faraday's Law of Electromagnetic Induction}

Consider a conductor a length $l$ which is part of a complete circuit cutting through lines of a magnetic field $B$.

If the conductor moves a distance $\Delta s$ in time $\Delta t$
\begin{itemize}
    \item The conductor experiences a force $F=BIl$ due to carrying a current in a magnetic field.
    \item Word done by the applied force $W=F\Delta s=BIl\Delta s$
    \item The charge transfer in this time is $Q=I\Delta t$.
    \item The induced emf is $\varepsilon=\dfrac{W}{Q}=\dfrac{BIl\Delta s}{I\Delta t}=\dfrac{Bl\Delta s}{t}=\dfrac{BA}{\Delta t}$
\end{itemize}
where $A$ is the area swept out by the conductor in time $\Delta t$.

\begin{itemize}
    \item \textbf{Magnetic flux} is the product of magnetic flux density and the area swept out.
        $$\phi=BA$$
    \item \textbf{Magnetic flux linkage} through a coil of $N$ turns $\Phi=N\phi=NBA$.
\end{itemize}

The unit of magnetic flux is the \textbf{weber}, equal to 1\,Tm$^2$.

When the magnetic field is at angle $\theta$ to the normal at the coil face, the flux linkage through the coil is $N\phi=BAN\cos\theta$.

\textbf{Faraday's law of electromagnetic induction} states that the induced emf in a circuit is equal to the rate of change of flux linkage through the circuit.
$$\varepsilon=-N\frac{d\phi}{dt}$$

\begin{itemize}
    \item For a moving conductor
        \begin{align*}
            \varepsilon&=\frac{Bl\Delta s}{\Delta t}\\
                       &=Blv
        \end{align*}
    \item For a fixed coil in a changing magnetic field
        \begin{align*}
            \varepsilon&=\frac{N\Delta\phi}{\Delta t}\\
                       &=\frac{NA\Delta B}{\Delta t}
        \end{align*}
\end{itemize}
