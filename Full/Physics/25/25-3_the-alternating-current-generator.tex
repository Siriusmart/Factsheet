\subsection{The Alternating Current Generator}

The simple AC generator consists of a \textbf{rectangular coil} that spins inside a uniform magnetic field.
\begin{enumerate}
    \item When the coil spins at a steady rate, the \textbf{flux linkage changes continuously}.
    \item At an instant when the normal to the plane of the coil is at angle $\theta$ to the field lines.
        $$N\phi=BAN\cos\theta$$
    \item For a coil spinning at a steady frequency, $\theta=\omega t$
        \begin{align*}
            N\phi&=BAN\cos\omega t\\
            \varepsilon=-N\frac{d\phi}{dt}&=BAN\omega\sin\omega t
        \end{align*}
\end{enumerate}
Or $\varepsilon=\varepsilon_0\sin\omega t$ where $\varepsilon_0$ is the peak voltage.
\begin{itemize}
    \item The \textbf{induced emf is zero} when the sides of the coil move parallel to the field lines.

        The rate of change of flux is zero and the sides of the coil do not cut the field lines.
    \item The \textbf{induced emf is a maximum} when the sides of the coil cut the field lines at right angles.

        The emf induced in each wire of each side is $Blv$, where $v$ is the speed of each wire.
        $$\varepsilon_0=2NBlv$$
\end{itemize}

The peak emf can be increased by
\begin{itemize}
    \item Increasing the speed (frequency of rotation).
    \item Using a stronger magnet.
    \item Bigger coil.
    \item Coil with more turns.
\end{itemize}

\subsubsection*{Back Emf}

An emf is induced in the spinning coil of an electric motor because the flux linkage through the coil changes. The induced emf \textbf{acts against the pd applied} to the motor in accordance of Lenz's law.
\begin{align*}
    V-\varepsilon&=IR\\
    IV-I\varepsilon&=I^2R\\
    IV&=I\varepsilon+I^2R
\end{align*}
$$\text{power supplied}=\text{power transferred to mechanical power}+\text{power wasted due to circuit resistance}$$

$$\text{efficiency of an electric motor}=\frac{\text{mechanical power output}}{\text{electrical power supplied}}$$
