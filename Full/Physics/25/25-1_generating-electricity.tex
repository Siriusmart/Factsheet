\subsection{Generating Electricity}

When a magnet is moved near a wire, an \textbf{emf is induced} in the wire. This effect is known as \textbf{electromagnetic induction}.
\begin{itemize}
    \item Occurs whenever a wire \textbf{cuts across magnetic field lines}.
    \item If the wire is part of a complete circuit, the induced emf \textbf{forces electrons around the circuit}.
    \item The induced emf can be increased by
        \begin{itemize}
            \item Moving the wire \textbf{faster}.
            \item Using a \textbf{stronger magnet}.
            \item Making the wire into a \textbf{coil}, and pushing the magnet in or out of the coil.
        \end{itemize}
\end{itemize}

\textbf{No emf} is induced in the wire if the wire is \textbf{parallel to the magnetic field lines} as it moves through the field - the wire must cut across field lines for an emf to be induced.

Other methods of generating an induced emf
\begin{itemize}
    \item Using an \textbf{electric motor} in reverse.
    \item Using a \textbf{cycle dynamo}.
\end{itemize}
In both cases an emf is induced because there is \textbf{relative motion} between the coil and the magnet.

\subsubsection*{Energy Changes}

An electric current \textbf{transfers energy} from the source of the emf in a circuit to the other components in the circuit. The induced emf becomes zero when the relative motion between the magnet and the wire ceases.
\begin{itemize}
    \item \textbf{Work must be done} to keep it spinning.
    \item The energy transferred by the coil is equal to the work done on the coil to keep it spinning.
\end{itemize}

\subsubsection*{Electromagnetic Induction in a Metal Rod}
\begin{enumerate}
    \item A metal rod is a tube containing lots of \textbf{free electrons}.
    \item If the rod is moved across a magnetic field, the field \textbf{forces the free electrons} in the rod to one end away from the other end.
    \item One end of the rod becomes negative and the other end positive, an emf is induced in the rod.
\end{enumerate}

If the relative motion ceases, the induced emf becomes zero because the magnetic field no longer exerts a force on the electrons in the rod.

The \textbf{dynamo rule}, or Fleming's right-hand rule can be used to find the direction of the induced current.
