\subsection{Transformers}

A transformer changes an alternating pd to a different peak value.
\begin{itemize}
    \item A transformer consists of \textbf{two coils} - the primary coil and the secondary coil.
    \item The two coils have the \textbf{same iron core}.
\end{itemize}

When the primary coil is connected to a source of \textbf{alternating pd}.
\begin{enumerate}
    \item An \textbf{alternating magnetic field} is produced in the core.
    \item The field passes through the secondary coil, so an \textbf{induced emf} is induced in the secondary coil by the changing magnetic field.
\end{enumerate}

A transformer is designed so that all the magnetic flux produced by the primary coil passes through the secondary coil.

\subsubsection*{The Transformer Rule}

Let $\phi$ be the flux in the core passing through each turn when an alternating pd $V_p$ is applied to the primary coil.
\begin{itemize}
    \item The flux linkage in the secondary coil is $N_s\phi$, by Faraday's law
        $$V_s=N_s\frac{d\phi}{dt}$$
    \item The flux linkage in the primary coil is $N_p\phi$, by Faraday's law
        $$V_p=N_p\frac{d\phi}{dt}$$
\end{itemize}

The induced emf in the primary coil \textbf{opposes the pd applied} to the primary coil. Assuming the resistance of the primary coil is negligible, all the applied pd acts against the induced emf in the primary coil.
\begin{align*}
    \text{Applied pd}&=V_p\\
    \frac{V_s}{V_p}&=\frac{N_s\frac{d\phi}{dt}}{N_p\frac{d\phi}{dt}}\\
                   &=\frac{N_s}{N_p}
\end{align*}
which is the \textbf{transformer rule}.

\begin{itemize}
    \item A \textbf{step-up transformer} has more turns on the secondary coil than the primary coil, so the secondary voltage is stepped up compared to the primary voltage.
    \item A \textbf{step-down transformer} has fewer turns on the secondary coil than the primary coil, so the secondary voltage is stepped down compared with the primary voltage.
\end{itemize}

\subsubsection*{Transformer Efficiency}

Transformers are almost 100\% efficient because they are designed with
\begin{itemize}
    \item \textbf{Low-resistance windings} to reduce power wasted due to the heating effect of the current.
    \item A \textbf{laminated core} which consists of layers of iron separated by layers of insulator. \textbf{Eddy currents} are reduced so the magnetic flux is as high as possible, and the heating effect of induced currents in the core is reduced.
    \item A core of \textbf{soft iron} core which is easily magnetised and demagnetised. This reduced power wasted through repeated magnetisation and demagnetisation of the core.
\end{itemize}
$$\text{Transformer efficiency}=\frac{I_sV_s}{I_pV_p}$$

\subsubsection*{The Grid System}

Electricity from power stations in the UK is fed into the National Grid System, which \textbf{supplies electricity} to most parts of the country.

The national grid is a \textbf{network of transformers and cables}, which covers all regions o the UK.
\begin{enumerate}
    \item \textbf{Step-up transformers} at the power station increases the alternating voltage to 400kV for long-distance transmission via the grid system.
    \item \textbf{Step-down transformers} operate in stages to reduce the transmitted voltage to 230V.
\end{enumerate}

Transmission of electrical power over long distances is much more efficient at high voltage than at low voltage.
\begin{itemize}
    \item If the voltage is increased, power wasted due to the heating effect of the current is reduced.
    \item To deliver power $P$ at voltage $V$, the current required is $I=\dfrac{P}{V}$.

        If the resistance of the cables is $R$, the power in heating the cables is $I^2R=\dfrac{P^2R}{V^2}$
    \item Therefore the higher the voltage is, the smaller the ratio of wasted power to the power transmitted is.
\end{itemize}
