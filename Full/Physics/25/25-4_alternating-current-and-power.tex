\subsection{Alternating Current and Power}

An alternating current is a current that repeatedly reverses its direction.
\begin{itemize}
    \item The \textbf{frequency} of an alternating current is the number of cycles it passes through each second.
    \item Mains electricity has a frequency of 50Hz.
    \item The \textbf{peak value} of an alternating current is the maximum current in either direction. The peak current in a circuit depends on the \textbf{peak pd} of the alternating current source, and the components in the circuit.
    \item The \textbf{peak pd} in a mains current is 325V.
    \item The \textbf{peak-to-peak value} is twice the peak value.
\end{itemize}

The heating effect of an electric current varies according to the \textbf{square of the current}.
$$\text{Power supplied to the heater}\ P=I^2R$$
\begin{itemize}
    \item At \textbf{peak current} $I_0$, the maximum power supplied is equal to ${I_0}^2R$.
    \item At zero current, zero power is supplied.
\end{itemize}

\subsubsection*{The Root Mean Square}


For a \textbf{sinusoidal current}, the \textbf{mean power} over a full cycle is \textbf{half the peak power}.
$$P_\text{mean}=\frac{1}{2}{I_0}^2R$$

The root mean square value of an alternating current is the value of direct current that would give the same heating effect as the alternating current in the same resistor.
\begin{align*}
    (I_\text{rms})^2R&=\frac{1}{2}{I_0}^2R\\
    I_\text{rms}&=\frac{1}{\sqrt{2}}I_0
\end{align*}

The root mean square value of an alternating pd is
$$V_\text{rms}=\frac{1}{\sqrt{2}}V_0$$
