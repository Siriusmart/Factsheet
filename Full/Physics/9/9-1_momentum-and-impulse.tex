\subsection{Momentum and Impulse}

\textbf{Force} is needed to change the velocity of an object.

The \textbf{effect of a force} on an object depends on.
\begin{itemize}
    \item The \textbf{mass} of the object.
    \item The amount of force.
\end{itemize}

Newton's laws of motion only apply where
\begin{itemize}
    \item \textbf{Gravity is weak}, and
    \item The speed of objects is \textbf{much less than the speed of light}.
\end{itemize}

The \textbf{momentum} of an object is defined as its $\text{mass}\times\text{velocity}$.
$p=mv$
\begin{itemize}
    \item The unit of momentum is kgms$^{-1}$.
    \item Momentum is a \textbf{vector quantity} with the same direction as the object's velocity.
\end{itemize}

\subsubsection*{Momentum and Newton's Laws of Motion}

\textbf{Newton's first law} tells us that a force is needed to change the momentum of an object.
\begin{itemize}
    \item If the momentum is constant, there is \underline{no resultant force} acting on it.
    \item If a moving object with constant momentum \textbf{gains or loses mass}, its \textbf{velocity would change} to keep its momentum constant.
        \begin{itemize}
            \item E.g. a cyclist speeding past a service point collects a water bottle and \textbf{gains mass} therefore \textbf{loses velocity}.
        \end{itemize}
\end{itemize}

\textbf{Newton's second law} states the force is proportional to the \textbf{change in momentum per second}.

\begin{align*}
    F\propto\frac{\text{change of momentum}}{\text{time taken}}&=\frac{mv-mu}{t}\\
                                                               &=\frac{m(v-u)}{t}\\
                                                               &=ma
\end{align*}

The change of momentum of an object can be written as $\Delta(mv)$.
\begin{align*}
    F&=\frac{m\Delta v}{\Delta t}\qquad\text{for constant $m$}\\
    F&=\frac{v\Delta m}{\Delta t}\qquad\text{for constant $v$}
\end{align*}

The \textbf{impulse of a force} is defined as $\text{force}\times\text{time for which the force acts}$.
$$I=F\Delta t=\Delta(mv)$$
The impulse of a force is equal to the \textbf{change of momentum} of the object.

The unit of momentum can be given in the \textbf{newton second} (Ns) or kgms$^{-1}$

\subsubsection*{Force-time Graphs}

The area under the line of a force-time graph represents the change of momentum or the impulse of the force.
