\subsection{Conservation of Momentum}

For the forces to be considered a \textbf{force pair}, they must be the same type. E.g. weight and normal reaction would not constitute a force pair.
\begin{itemize}
    \item The Earth exerts a \textbf{force due to gravity} on an object, which exerts and equal and opposite force on the Earth.
    \item A jet engine exerts a \textbf{force on the hot gas} in the engine to expel the gas, which exerts an equal and opposite force on the engine.
\end{itemize}

If no external resultant force acts on the object, interactions between objects can \textbf{transfer momentum} between them, but the total momentum does not change.

The \textbf{principle of conservation of momentum} states that for a system of interacting objects, the total momentum remains constant, provided no external resultant force acts on the system.
\begin{enumerate}
    \item Consider two objects collide with each other then separate.
    \item They exert \textbf{equal and opposite forces} on each other when they are in contact.
    \item So the change of momentum of one object is equal and opposite to the change of momentum of the other object.
    \item So the total amount of momentum is unchanged.
\end{enumerate}
$$\text{Impact force}\ F_I=\frac{m_Av_A-m_Au_A}{t}$$

And
$$\text{total final momentum}=\text{total initial momentum}$$

If colliding objects stick together as a result of the collision, they have the \textbf{same final velocity}, so
$$(m_A+m_B)V=m_Au_A+m_Bu_B$$
