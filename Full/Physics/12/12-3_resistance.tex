\subsection{Resistance}

\begin{itemize}
    \item The resistance of a component is a measure of the \textbf{difficulty of making current pass through} the component.
    \item Resistance is caused by the \textbf{repeated collisions} between the charge carriers in the material with each other and with the fixed positive ions in the materials.
\end{itemize}

The resistance of any component is defined as
$$\frac{\text{pd across the component}}{\text{current through it}}$$

For a component which passes current $I$ when the pd across it is $V$
$$R=\frac{V}{I}$$

The unit of resistance is the \textbf{ohm} ($\Omega$) equal to 1 volt per ampere.

\subsubsection*{Resistance Measurements}

\begin{itemize}
    \item The \textbf{ammeter} is used to measure the current through the resistor.
        \begin{itemize}
            \item In series with the resistor so the same current passes through both the resistor and the ammeter.
        \end{itemize}
    \item The \textbf{voltmeter} is used to measure the pd across the resistor.
        \begin{itemize}
            \item Parallel with the resistor so they both have the same pd.
        \end{itemize}
    \item A \textbf{variable resistor} is used to adjust the current and pd as necessary.
\end{itemize}

\begin{enumerate}
    \item The variable resistor is \textbf{adjusted in steps}, at each step the current and pd are recorded from the ammeter and the voltmeter.
    \item The measurements can be plotted on a graph of \textbf{pd against current}.
\end{enumerate}

The graph is a \textbf{straight line through the origin}.
\begin{itemize}
    \item Resistance is the same regardless of the current.
    \item Resistance is equal to the \textbf{gradient of the graph}.
\end{itemize}

\textbf{Ohm's law} states that the pd across a metallic conductor is proportional to the current through it, provided the physical conditions do not change.

\subsubsection*{Resistivity}

For any conductor of length $L$ and uniform cross-sectional area $A$
\begin{align*}
    R&\propto L\\
    R&\propto\frac{1}{A}
\end{align*}

Hence $R=\dfrac{\rho L}{A}$ where $\rho$ is a constant for the material known as its resistivity.

Rearranging for resistivity
$$\rho=\frac{RA}{L}$$
The unit of resistivity is the \textbf{ohm meter} ($\Omega$m).

\subsubsection*{Resistivity Measurements}
\begin{enumerate}
    \item \textbf{Measure the diameter} $d$ of the wire using a micrometer at several points along the wire.
    \item \textbf{Calculate the cross-sectional area} $A$ using the mean value for $d$.
    \item \textbf{Measure the resistance} $R$ for different lengths $L$ of the wire.
    \item \textbf{Plot a graph} of $R$ against $L$.
\end{enumerate}

The resistivity of the wire is given by $\text{graph gradient}\times A$

\subsubsection*{Superconductivity}

A \textbf{superconductor} is a material that has \textbf{zero resistivity} at and below a \textbf{critical temperature} that depends on the material. This property of the material is called \textbf{superconductivity}.

The wire has zero resistance below the critical temperature of the material. When a current passes through it, there is \textbf{no pd across it} because its resistance is zero, so the current has \textbf{no heating effect}.

A \textbf{high-temperature superconductor} is any material with a critical temperature above the \textbf{boiling point of nitrogen}.

Superconductors are used to
\begin{itemize}
    \item Make \textbf{high-power electromagnets} that generate very strong magnetic fields.
        \begin{itemize}
            \item MRI scanners.
            \item Particle accelerators.
            \item Lightweight electric motors.
        \end{itemize}
    \item \textbf{Power cables} that transfer electrical energy without energy dissipation.
\end{itemize}
