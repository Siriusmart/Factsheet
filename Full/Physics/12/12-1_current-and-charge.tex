\subsection{Current and Charge}

To make an electric current pass round a circuit.
\begin{itemize}
    \item The circuit must be \textbf{complete}.
    \item There must be a \textbf{source of potential difference}.
\end{itemize}

The electric current is the \textbf{rate of flow of charge}.

Electric current is due to the passage of \textbf{charged particles}, they are referred to as \textbf{charge carriers}.
\begin{itemize}
    \item In \textbf{metals}, the charge carriers are \textbf{conduction electrons}.
        \begin{itemize}
            \item They move about inside the metal.
            \item Repeatedly colliding with each other and the fixed positive ions in the metal.
        \end{itemize}
    \item In a \textbf{salt solution}, the charge is carried by \textbf{ions}, which are charged atoms or molecules.
\end{itemize}

\subsubsection*{Test for Conduction}

The meter shows a non-zero reading whenever any conducting material is connected into the circuit.
\begin{enumerate}
    \item The battery forces the charge carrier \textbf{through the conducting material}.
    \item Causes them to \textbf{pass through the battery} and the meter.
    \item Electrons enter the battery at its positive terminal and leave at the negative terminal.
\end{enumerate}

\textbf{Conventional current} flows from positive to negative.

\begin{itemize}
    \item The unit of current is the \textbf{ampere} - defined in terms of the magnetic force between two parallel wires when they carry the same current.
    \item The unit of charge is the \textbf{coulomb} - equal to the \textbf{charge flow} in one second when the current is one ampere.
\end{itemize}

For current $I$, charge flow $\Delta Q$ in time $\Delta t$ is given by
$$\Delta Q=I\Delta t$$

\subsubsection*{Charge Carriers}
\begin{itemize}
    \item In an \textbf{insulator}, each electrons is attached to an atom and cannot move away from the atom.

        When a voltage is applied across an insulator, no current passes through the insulator, because no electrons can move through the insulator.
    \item In a \textbf{metallic conductor}, some electrons are \textbf{delocalised} - they are the \textbf{charge carriers in the metal}.

        When a voltage is applied across the metal, these conduction electrons are attracted towards the positive terminal of the metal.
    \item In a \textbf{semiconductor}, the number of charge carriers increases with an increase in temperature.
        
        The resistance of a semiconductor therefore decreases as its temperature is raised.
        \begin{itemize}
            \item Conduction is due to electrons that \textbf{break free from the atoms} of the semiconductor.
        \end{itemize}
\end{itemize}

A pure semiconducting material is referred to as an \textbf{intrinsic semiconductor}.
