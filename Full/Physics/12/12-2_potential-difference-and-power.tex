\subsection{Potential Difference and Power}

\begin{enumerate}
    \item Each electron moves around the circuit and \textbf{takes a fixed amount of energy from the battery} as it passes through it.
    \item Each electron passing through a circuit component \textbf{does work} to pass through the component and therefore transfers some of its energy.
\end{enumerate}

\textbf{Potential difference} is defined as the energy transfer per unit charge.

The unit of potential difference is the \textbf{volt}, equal to one joule per coulomb.
$$V=\frac{\Delta E}{\Delta Q}$$

The \textbf{emf of a source} of electricity is defined as the electric energy produced per unit charge passing through the source.
$$\text{Electrical energy produced}=Q\varepsilon$$
The unit of emf is also the volt.

\subsubsection*{Energy Transfer in Devices}
\begin{itemize}
    \item Any device with \textbf{resistance}, the work done of the device is transferred as thermal energy.

        \textbf{Charge carriers} repeatedly collide with atoms in the device the transfer energy to them, so atoms vibrate more and the resistor becomes hotter.
    \item In an \textbf{electric motor} turning at a \textbf{constant speed}, the work done on the motor is equal to the energy transferred to the load and surroundings by the motor.

        The electrons need to be \textbf{forced through the wires} against the opposing force on the electrons due to the motor's magnetic field.
    \item In a \textbf{loudspeaker}, work done on the loudspeaker is transferred as \textbf{sound energy}.

        Electrons need to be \textbf{forced through the wires} of the coil against the force on them due to the loudspeaker magnet.
\end{itemize}

\subsubsection*{Electrical Power}

Consider a component with pd $V$ across its terminals and a current $I$ passing through it.
\begin{align*}
    \Delta Q&=I\Delta t\\
    \Delta E&=\Delta QV\\
     &=IV\Delta t
\end{align*}

So
$$\text{Electrical power}\ P=\frac{IV\Delta t}{\Delta t}=IV$$
