\subsection{Components and Their Characteristics}

Each type of component has its own symbol which is used to \textbf{represent the component in a circuit diagram}.

\begin{itemize}
    \item A \textbf{cell} is a source of electrical energy.
    \item A \textbf{battery} is a combination of cells.
    \item A \textbf{diode} allows current in one direction only.
    \item A \textbf{light-emitting diode} emits light when it conducts.
        \begin{itemize}
            \item The direction in which it conducts is referred to as its \textbf{forward direction}.
            \item The opposite direction is referred to as its \textbf{reverse direction}.
            \item Diodes are used in the \textbf{protection of DC circuits} in case the voltage supply is connected the wrong way round.
        \end{itemize}
    \item A \textbf{resistor} is a component designed to have a certain resistance.
    \item The resistance of a \textbf{thermistor} decreases with increasing temperature.
    \item The resistance of a \textbf{light-dependent resistor} decreases with increasing light intensity.
\end{itemize}

\subsubsection*{I/V Characteristics of Components}

\begin{itemize}
    \item A \textbf{potential divider} varies the pd from zero.
    \item A \textbf{variable resistor} varies the current to a minimum.
\end{itemize}

A potential divider allows the current through the component and the pd across it to be \textbf{reduced to zero}, this is not possible with a variable resistor.

Measurements can be plotted as a graph of \textbf{current against pd}.
\begin{itemize}
    \item A \textbf{resistor} at constant temperature gives a \textbf{straight line through origin} - the resistance of a resistor does not change when the current changes.
    \item A \textbf{filament bulb} gives a curve with \textbf{decreasing gradient} because its resistance increases as it becomes hotter.
    \item A \textbf{thermistor} at constant temperature gives a straight line. The \textbf{higher the temperature, the greater the gradient} of the line.
    \item A \textbf{diode} conducts easily in its forward direction above a pd of about 0.6V, and hardly at all below 0.6V or in the opposite direction.
\end{itemize}

\subsubsection*{Temperature Coefficients}

\begin{itemize}
    \item A metal have a \textbf{positive temperature coefficient} because its resistance increases with increase of temperature.
        \begin{itemize}
            \item The positive ions \textbf{vibrate more} when its temperature is increased.
            \item The conduction electrons therefore \textbf{cannot pass through as easily}.
        \end{itemize}
    \item An intrinsic semiconductor has a \textbf{negative temperature coefficient} because the number of charge carriers increases when the temperature is increased.
\end{itemize}

Because its \textbf{percentage change of resistance per kelvin change of temperature} is much greater than for a metal, thermistors are often used as the temperature-sensitive component in a temperature sensor.
