\subsection{Oscillations}

The \textbf{equilibrium position} is the position with the \textbf{least potential energy} - it is where the oscillation will eventually come to a standstill. The motion is described as \textbf{oscillating about equilibrium}.

\begin{itemize}
    \item An \textbf{object on a spring} moves up and down repeatedly.
    \item A \textbf{pendulum} moving back and forth.
    \item A boat rocking from side to side.
\end{itemize}

An oscillating object moves repeatedly \textbf{one way then in the opposite direction} through its equilibrium position.

The \textbf{displacement} of the object is its distance and direction from equilibrium. It \textbf{continually changes} during the motion, in \textbf{one full cycle} after being \textbf{released from a non-equilibrium position}.
\begin{enumerate}
    \item \textbf{Decreases} as it returns to equilibrium,
    \item \textbf{Reverses and increases} as it moves away from equilibrium in the opposite direction.
    \item Decreases as it returns to equilibrium.
    \item Increases as it moves away from equilibrium towards its starting position.
\end{enumerate}

\begin{itemize}
    \item The \textbf{amplitude} of the oscillations is the \textbf{maximum displacement} of the oscillating object from equilibrium.
    \item If the \textbf{amplitude is constant} and \textbf{no frictional forces} are present, the oscillations are described as \textbf{free vibrations}.
    \item The \textbf{time period} of the oscillating motion is the time for \textbf{one complete cycle of oscillation}.
    \item One \textbf{full cycle} after passing through any position, the object passes through that same position in the same direction.
    \item The \textbf{frequency} of oscillations is the \textbf{number of cycles per second} made by an oscillating object.

        The unit of frequency is the \textbf{hertz} (Hz), which is one cycle per second.
    \item The \textbf{angular frequency} of the oscillating motion is defined as
        $$\omega=\dfrac{2\pi}{T}$$
\end{itemize}

The \textbf{phase difference} of two oscillating objects stays the same if they oscillates with the same frequency.

If $\Delta t$ is the time between successive instants when the two objects are at \textbf{maximum displacement in the same direction}, one object is always $\dfrac{\Delta t}{T}$ cycles ahead. So for two objects oscillating in the same frequency.
$$\text{Phase difference}=\frac{2\pi\Delta t}{T}=2\pi\Delta tf$$
