\subsection{Forced Vibrations and Resonance}

\begin{itemize}
    \item A \textbf{periodic force} is a force applied at regular intervals.
    \item When a system oscillates \textbf{without a periodic force} being applied to it, the system's frequency is called its \textbf{natural frequency}.
    \item The system undergoes \textbf{forced vibrations} when a periodic force is applied to it.
\end{itemize}

\subsubsection*{Investigating Forced Vibrations}

A fixed mass is attached to \textbf{stretched strings}.
\begin{itemize}
    \item The bottom end of the bottom spring is attached to a \textbf{mechanical oscillator} connected to a signal generator.
    \item The top end of the top spring is fixed.
\end{itemize}

The mechanical oscillator \textbf{pulls repeatedly} on the lower string, and its frequency is called the \textbf{applied frequency}. As the applied frequency increases

\begin{enumerate}
    \item The amplitude of oscillations of the system \textbf{increases until it reaches a maximum amplitude} at a particular frequency, and then the amplitude decreases again.
    \item The \textbf{phase difference} between the displacement and the periodic force increases from zero to $\dfrac{\pi}{2}$ at the maximum amplitude, then from $\dfrac{\pi}{2}$ to $\pi$ as the frequency increases further.
\end{enumerate}

\subsubsection*{Resonance}

When the system is oscillating at the maximum frequency, the periodic force is \textbf{exactly in phase with the velocity of the oscillating system}, and the system is in \textbf{resonance}.

The frequency at the maximum amplitude is called the \textbf{resonant frequency}.

The larger the damping
\begin{itemize}
    \item The \textbf{larger the maximum amplitude} becomes at resonance.
    \item The closer the resonant frequency to its \textbf{natural frequency} of the system.
\end{itemize}

As the \textbf{applied frequency becomes increasingly larger} than the resonant frequency of the mass-spring system.
\begin{itemize}
    \item The \textbf{amplitude decreases} more and more.
    \item The \textbf{phase difference} between the displacement and the periodic force increases from $\dfrac{\pi}{2}$ until the displacement is $\pi$ radians out of phase with the periodic force.
\end{itemize}

For an oscillating system with \textbf{little to no damping}.
$$\text{Resonant frequency}=\text{natural frequency of the system}$$

\subsubsection*{Barton's Pendulums}

Simple pendulums of different lengths hangs from a support thread stretched between two fixed points.
\begin{enumerate}
    \item A \textbf{driver pendulum} is displaced and released so it oscillates in a plane perpendicular to the plane of the pendulums at rest.
    \item The effects of the oscillating motion of the driver is \textbf{transmitted along the support thread}.
    \item Each of the other pendulums are subjected to \textbf{forced oscillations}.
    \item The pendulum with the \textbf{same length as the driver} responds much more than any other pendulum.
\end{enumerate}

\begin{itemize}
    \item This is because it has the \textbf{same length} and therefore the \textbf{same time period} as the driver.
    \item So its natural frequency is the \textbf{same as the natural frequency} of the driver.
    \item Therefore it \textbf{oscillates in resonance with the driver} because it is subjected to the same frequency as its own natural frequency.
\end{itemize}

The oscillations of each other pendulum \textbf{depends on how close its length} is to the length of the driver.

\subsubsection*{Bridge Oscillations}

A bridge can oscillate because of its \textbf{springiness} and its \textbf{mass}.

A bridge not fitted with dampers can be made to oscillate at resonance if the bridge span is \textbf{subjected to a suitable periodic force}.
