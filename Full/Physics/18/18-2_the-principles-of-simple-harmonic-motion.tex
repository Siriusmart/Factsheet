\subsection{The Principles of Simple Harmonic Motion}

The velocity of an object is given by the \textbf{gradient of the displacement-time graph}.
\begin{itemize}
    \item The \textbf{magnitude of the velocity} is greatest when the object is at at \textbf{zero displacement when the object passes through equilibrium}.
    \item The \textbf{velocity is zero} when the object is at \textbf{maximum displacement} in either direction.
\end{itemize}

The acceleration of an object is given by the \textbf{gradient of the velocity-time graph}.
\begin{itemize}
    \item The \textbf{acceleration is greatest} when the \textbf{velocity is zero} at maximum displacement in the opposite direction.
    \item The \textbf{acceleration is zero} when the displacement is zero and the velocity is a maximum.
\end{itemize}

\subsubsection*{Conditions for Simple Harmonic Motion}

Simple harmonic motion is defined as oscillating motion in which acceleration is
\begin{itemize}
    \item Proportional to the displacement, and
    \item Always in the \textbf{opposite direction to the displacement}
        $$a\propto-x$$
        or in other words, $a=-kx$.
\end{itemize}

The acceleration for simple harmonic motion is
$$a=-\omega^2x$$
