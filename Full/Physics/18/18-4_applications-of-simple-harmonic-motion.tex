\subsection{Applications of Simple Harmonic Motion}

For any oscillating object, the resultant force is described as a \textbf{restoring force} because it always \textbf{acts towards the equilibrium}.

The frequency of an oscillating object can be reduced by
\begin{itemize}
    \item \textbf{Adding extra mass} increases the \textbf{inertia} of the system. At a given displacement there will be \textbf{less acceleration} than if the extra mass is not added. So each cycle of oscillation would therefore be longer.
    \item \textbf{Use weaker springs} so the restoring force at any given displacement would be less, and the \textbf{acceleration and speed at any given displacement would be less}. So each cycle of oscillation would therefore be longer.
\end{itemize}

\subsubsection*{Mass-spring System}

Consider an object of mass $m$ attached to a spring.
\begin{enumerate}
    \item Assuming the spring obeys \textbf{Hooke's law}, the tension in the spring
        $$T=k\Delta L$$
    \item When the object is at displacement $x$ from its equilibrium position, the \textbf{change in tension} from its equilibrium position is
        $$\Delta T=-kx$$
        Where the minus sign represents the fact that change of tension always tries to \textbf{restore the object to its equilibrium position}.
    \item So the \textbf{restoring force} on the object $F=-kx$.
    \item An acceleration
        $$a=\frac{-kx}{m}$$
\end{enumerate}

The equation can be written in form $a=-\omega^2x$ where $\omega^2=\dfrac{k}{m}$.

So time period
$$T=\frac{2\pi}{\omega}=2\pi\sqrt{\frac{m}{k}}$$

\subsubsection*{Simple Pendulum}

Consider a simple pendulum of a bob of mass $m$ attached to a thread of length $L$.
\begin{enumerate}
    \item At displacement $s$ from the lowest point, the thread is at angle $\theta$ to the vertical.
    \item Weight $mg$ has components.
        \begin{itemize}
            \item Perpendicular to path of bob: $mg\cos\theta$
            \item Along the path towards the equilibrium position: $mg\sin\theta$, which is the \textbf{restoring force}.
        \end{itemize}
    \item The acceleration is
        $$a=\frac{-mg\sin\theta}{m}=-g\sin\theta$$
    \item As long as $\theta$ does not exceed 10$^\circ$, $\sin\theta\approx\frac{s}{L}$
        $$a-\frac{g}{L}s=-\omega^2s$$
        where $\omega^2=\dfrac{g}{L}$
\end{enumerate}

So time period
$$T=\frac{2\pi}{\omega}=2\pi\sqrt{\frac{L}{g}}$$
