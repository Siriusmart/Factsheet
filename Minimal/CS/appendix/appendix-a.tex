\subsection*{Appendix A}

Fixed point binary number assumes a \textbf{pre-determined number of bits} before and after the point.

\subsubsection*{Floating Point Binary Numbers}

Floating point binary allows very large numbers to be represented.

In \textbf{scientific notation} $m\times 10^n$, $m$ is known as the \textbf{mantissa} and $n$ the \textbf{exponent}.
\begin{itemize}
    \item The leftmost bit of both the mantissa and the exponent are sign bits.
    \item Since both numbers are represented using \textbf{two's complement}.
    \item The binary point is to the right of the sign bit.
\end{itemize}

\subsubsection*{Normalisation}

Normalisation is the process of moving the binary point of a floating point number to provide the \textbf{maximum level of precision} for a given number of bits.

This is achieved by ensuring that the first digit after the binary point is a significant digit. In normalised floating point form,
\begin{itemize}
    \item A positive number has a sign bit of 0, and the next bit is always 1.
    \item A negative number has a sign bit of 1, and the next bit is always 0.
\end{itemize}

The size of the mantissa determine the \textbf{precision} of the number, the size of the exponent determine the \textbf{range} of the numbers that can be held.

\textbf{Rounding errors} are unavoidable and result in a loss of accuracy.
\begin{itemize}
    \item The \textbf{absolute error} is calculated as the difference between the number to be represented, and the actual binary number that is the closest possible approximation in the given number of bits.
    \item The \textbf{relative error} is the absolute error divided by the number.
\end{itemize}

\subsubsection*{Advantages and Disadvantages}
\begin{itemize}
    \item Floating point allows a far \textbf{greater range} of numbers using the same number of bits - very large numbers and very small fractional numbers can be represented.

        The larger the mantissa, the greater the precision, the larger the exponent, the greater the range.
    \item In fixed point binary, the range and precision of the numbers that can be represented \textbf{depends on the position of the binary point}.

        Fixed point binary is a \textbf{simpler system} and is faster to process.
\end{itemize}

\textbf{Underflow} occurs when a number is too small to be represented in the allotted number of bits. \textbf{Overflow} occurs when the result of a calculation is too large to be held in the number of bits allotted.
