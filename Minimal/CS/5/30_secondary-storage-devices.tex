\subsection{Secondary Storage}

Secondary storage is \textbf{not directly accessible} and has \textbf{slower access speeds}, but it \textbf{retains its contents} when the computer's power is turned off.

Storage devices use a technique which allows them to \textbf{create and maintain a toggle state} without power to represent either a 1 or 0.

\subsubsection*{Hard Disk}

The disk:
\begin{itemize}
    \item Rigid \textbf{rotating platters} coated with magnetic material.
    \item Ferrous particles on the disk are \textbf{polarised} to become either a north or south state, representing a 0 or 1.
    \item The disk is divided into \textbf{tracks} in concentric circles, each track is subdivided into \textbf{sectors}.
    \item The disk \textbf{spins very quickly} at speeds of up to 10,000 RPM.
\end{itemize}

The drive head:
\begin{itemize}
    \item Moves across the disk to \textbf{access different tracks and sectors}.
    \item Data is read or written to the disk as it passes under the drive head.
    \item The drive head is \textbf{not in contact} with the disk, but floats a fraction of a millimetre above it.
    \item When the drive head is not in use, it is \textbf{parked to one side} of the disk in order to prevent damage from movement.
\end{itemize}

A hard disk may consists of \textbf{several platters}, each with its own drive head.
\begin{itemize}
    \item Hard disks are \textbf{less portable} than optical or solid state media.
    \item Their \textbf{huge capacity} makes them suitable for desktop purposes.
    \item Denser surfaces areas allows capacities up to \textbf{several terabytes}.
\end{itemize}

\subsubsection*{Optical Disk}

The three \textbf{formats} of optical disks are
\begin{itemize}
    \item Read only: CD-ROM - it is pressed during manufacture and has pits in its surface.
    \item Recordable: CD-R - uses a reflective layer with a \textbf{transparent dye coating} that becomes less reflective when a spot laser burns a spot in the track.
    \item Rewritable: CD-RW - uses a laser and a magnet to \textbf{heat a spot} on the disk, then \textbf{set its state using a magnet} before it cools again.
    \item DWD-RW - uses a \textbf{phase changing alloy} that can change between amorphous and crystalline states by \textbf{changing the power} of the laser beam.
\end{itemize}

An optical disk works by using a \textbf{high powered laser} to burn sections of its surface, making them less reflective.

A laser at \textbf{low power} is used to read the disk by shining light onto the surface, and a sensor is used to measure the amount of light that is reflected back.

\begin{itemize}
    \item \textbf{Pits} are pressed areas.
    \item \textbf{Lands} are areas that have not been pitted.
    \item At the point where a pit starts or ends, light is scattered and therefore not reflected well - reflective and non-reflective areas are read as 1s and 0s.
\end{itemize}

There is only \textbf{one single track} on an optical disk, arranged as a tight spiral.

\textbf{Blu-ray} has added capacity from
\begin{itemize}
    \item \textbf{Shorter wavelength} of laser used.
    \item This create much \textbf{smaller pits} - enabling a greater number to fit in the same space along the track.
    \item The track can also be more \textbf{tightly wound}, therefore much longer.
\end{itemize}

Optical storage is
\begin{itemize}
    \item \textbf{Very cheap} to produce.
    \item \textbf{Easy to send} through post for distribution purposes.
    \item Disks are used for \textbf{small backups} or \textbf{storing music}, photographs or films.
    \item Disk data can be \textbf{corrupted or damaged easily} by excessive sunlight or scratches.
\end{itemize}

\subsubsection*{Solid-state Disk}

A solid-state disk consists of an \textbf{array of chips} arranged on a board, comprising of
\begin{itemize}
    \item \textbf{Millions of NAND flash memory cells}.
    \item A \textbf{controller} that manage pages and blocks of memory.
\end{itemize}

Each cell works by
\begin{enumerate}
    \item Delivering a current along the \textbf{word lines} to activate the flow of electrons from the \textbf{source} towards the \textbf{drain}.
    \item The current on the word line is strong enough to \textbf{force a few electrons across} an insulated oxide layer into a \textbf{floating gate}.
    \item Once the current is turned off, these \textbf{electrons are trapped}.
\end{enumerate}

The state of the NAND cell is determined by \textbf{measuring the charge} in the floating gate.
\begin{itemize}
    \item No charge is considered a 1.
    \item Some charge is considered a 0.
\end{itemize}

Data is stored in \textbf{pages}, grouped into \textbf{blocks}. NAND flash memory \textbf{cannot overwrite existing data} - old data must be erased before data can be written to the same location.
\begin{enumerate}
    \item A \textbf{separate block} is created to mirror the data to be transferred to the solid state memory.
    \item The data is then written to the new block.
    \item The contents of the original block are \textbf{marked as stale} and are erased when new data is written to it.
\end{enumerate}

SSDs are used in personal devices.
\begin{itemize}
    \item Silent in operation.
    \item Lighter and highly portable.
    \item Consume \textbf{far less power} than traditional hard drives.
    \item \textbf{Faster access speeds} - no need to move a read-write head across the disk, one piece of data can be accessed just as quickly as any other, even if they are not closed together.
\end{itemize}

However, they have \textbf{lower capacity} than traditional hard disks.
