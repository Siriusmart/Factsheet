\subsection{Input-output Devices}

\subsubsection*{Barcodes}

Barcodes are used for
\begin{itemize}
    \item Tracking parcels.
    \item Sale of items in shops.
    \item Record the details of people attending events.
\end{itemize}

\textbf{Linear barcode} are 1D, 2D barcodes such as the \textbf{Quick Response} code can hold more information than the 1D barcode.

2D barcodes are used for
\begin{itemize}
    \item \textbf{Ticketless entry} to concerts.
    \item To provide a \textbf{website URL}.
\end{itemize}

\subsubsection*{Barcode Readers}

\begin{itemize}
    \item \textbf{Pen-type reader} - a \textbf{light source} and a \textbf{photo diode} are placed next to each other in the tip of the pen.
        
        \begin{enumerate}
            \item The tip of the pen is \textbf{dragged across all the bars} at even speed.
            \item The photo diode \textbf{measures the intensity} of the light reflected back from the light source.
            \item Dark bars absorb light and white spaces reflect light, so the voltage waveform generated by the photo diode can be used to measure the widths of the bars and spaces in the barcode.
            \item The signal is converted from analogue to digital.
        \end{enumerate}

        A simple encoding translates areas of light and dark as 1s and 0s, these can be used to create \textbf{ASCII character codes} for a string which could be a product code.

        Pen-type readers are suited for use with portable computers or very \textbf{low volume scanning applications}.
        \begin{itemize}
            \item \textbf{Durable} and can be sealed against dust, dirt and other environmental hazards.
            \item \textbf{Small size} and \textbf{low weight}.
            \item Applications are limited because they \textbf{must come into direct contact} with the barcode to read it.
        \end{itemize}

    \item \textbf{Laser scanners} use a laser beam as the light source. The laser reflects off a \textbf{moving mirror} which allows the barcode to be read in many different positions.
        \begin{itemize}
            \item \textbf{Reliable} and economical for \textbf{low-volume} applications.
            \item Used as \textbf{in-counter} units in supermarkets.
        \end{itemize}

    \item \textbf{Carged-coupled device (CCD) readers} use an array of \textbf{hundreds of tiny light sensors} lined up in a row at the head of the reader.
        \begin{enumerate}
            \item Each sensor measures the intensity of the light immediately in front of it.
            \item A \textbf{voltage pattern} identical to the pattern in a barcode is generated in the reader by sequentially measuring the voltages across each sensor in the row.
        \end{enumerate}

    \item \textbf{Camera-based imaging scanner} uses a camera and \textbf{image processing} techniques to decode a 1D or 2D barcode.

        An imaging scanner can read barcode
        \begin{itemize}
            \item On any surface - printed or onscreen.
            \item Damaged or poorly printed.
        \end{itemize}

        \textbf{Image processing} has to be carried out by the software as the barcode might be in \textbf{any rotation or distance} from the scanner.
        \begin{itemize}
            \item Event ticketing - electronic tickets are scanned off a phone screen.
            \item Using cell phone to scan a QR code which display information about a product.
        \end{itemize}
\end{itemize}

\subsubsection*{Digital Cameras}

Uses a \textbf{CCD} or \textbf{CMOS} sensor comprising of millions of tiny light sensors arranged in a grid.
\begin{enumerate}
    \item When the \textbf{shutter opens}, light enters the camera and \textbf{projects an image} onto the sensor at the back of the lens.
    \item Each sensor \textbf{measures the brightness} of each pixel - turns light into electricity, and stores the amount of charge as binary data.
\end{enumerate}

The binary data is recorded onto the camera's \textbf{memory card}, so the image can be reproduced using suitable software.

\begin{enumerate}
    \setcounter{enumi}{2}
    \item To record colour, the sensors are placed under a \textbf{mosaic of red, green and blue filters} to separate out the different colour wavelengths.
    \item The processor can then \textbf{approximate binary values} for the three RGB channels of each individual pixel based on the value of neighbouring pixels.
\end{enumerate}

\begin{itemize}
    \item \textbf{CCD sensor} produce \textbf{higher quality images} and are used in high end cameras, and are \textbf{more reliable}.
    \item \textbf{CMOS sensors} consume 100 times less power than a CCD sensor.
\end{itemize}

\subsubsection*{Radio Frequency Identification}

RFID tags are used to track household products, cars, bank cards and animals.

\begin{itemize}
    \item Can be read \textbf{without line of sight} and up to 300 metres away.
    \item Can \textbf{pass stored data} from the tag to the receiver and vice versa.
\end{itemize}

RFID chip consists of a \textbf{small microchip transponder} and an antenna to communicate with the \textbf{base unit}.

\begin{itemize}
    \item \textbf{Active tags} are physically large as they \textbf{include a battery} to power the tag.

        It \textbf{actively transmits a signal} for a reader to pick up. And is used to track things to be read from further away, e.g. cars as they pass through a toll booth.
    \item \textbf{Passive tags} do not have a battery. They rely on \textbf{radio waves emitted from a reader} to provide electromagnetic power to the card using its \textbf{coiled antenna}.

        Once energised, the transponder inside the RFID tag can send its data to the reader nearby. E.g. tagging grocery items and smart cards.
\end{itemize}

\subsubsection*{Laser Printer}

Laser printers offer \textbf{high-quality, high-speed printing}. It uses \textbf{powdered ink} called \textbf{toner}.
\begin{itemize}
    \item Generates a bitmap image of the printed page.
    \item Use a laser unit and mirror to \textbf{draw a negative} onto a negatively charged drum, causing the affected area of the drum to lose their charge.
    \item The drum rotates pass a toner hopper to \textbf{attract charged toner particles} onto the areas which have not been lasered.
    \item The particles are \textbf{transferred onto a sheet} of paper then bonded onto it using pressure and heat.
\end{itemize}

\textbf{Coloured laser printers} contains four toner cartridges, and the paper must go through a similar process to the back-only printer four times - once for each colour.
\begin{itemize}
    \item Quality is limited.
    \item Photorealistic prints is impossible (use inkjet printers instead).
\end{itemize}
