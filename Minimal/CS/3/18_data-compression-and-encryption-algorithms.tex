\subsection{Data Compression and Encryption Algorithms}

In streaming audio or video, \textbf{buffering} refers to \textbf{downloading a certain amount of data} to a temporary storage area before starting to play a section of the music or movie.

\subsubsection*{Lossy Compression}

Lossy compression works by \textbf{removing non-essential information}.
\begin{itemize}
    \item A heavily compressed JPG image displays \textbf{untidy and blocky compression artifacts}.
    \item MP3 files use lossy compression to \textbf{remove frequencies too high} for most of us to hear, and \textbf{remove quieter sounds} played at the same time as louder sounds.
\end{itemize}
The degree to which a file is compressed (lossy) comes at the cost of quality.

\subsubsection*{Lossless Compression}

\begin{itemize}
    \item Lossless compression works by \textbf{recording patterns} in data rather than the actual data.
    \item Using these patterns and a set of instructions on how to use them the computer can \textbf{reverse the procedure} and reassemble an image, sound or text file with \textbf{exact accuracy} and \textbf{no data is lost}.
\end{itemize}

\textbf{Lossless compression} is used when a lost of a single character would result in an error, such as the compression of program code. \textbf{Lossy compression} is used when a pixel with slightly different colour \textbf{would not be a huge consequence} in most cases.

\begin{itemize}
    \item \textbf{Run length encoding} records a value and the number of times it \textbf{repeats}.
    \item In \textbf{dictionary based compression}, the compression algorithm searches through the text to find suitable entries in its own dictionary, and translates the message accordingly.

        The longer the body of text to be compressed, the dictionary becomes \textbf{insignificant in size} compared with the original.
\end{itemize}

\subsubsection*{Encryption}

Encryption is the transformation of data from one form to another to \textbf{prevent an unauthorised third party} from being able to understand it.
\begin{itemize}
    \item The original data is known as \textbf{plaintext}.
    \item The encrypted data is known as \textbf{ciphertext}.
    \item The encryption method or algorithm is known as the \textbf{cipher}.
    \item The secret information to lock or unlock the message is known as a \textbf{key}.
\end{itemize}

The \textbf{Caesar cipher} is a type of \textbf{substitution cipher} and works by shifting the letters of the alphabet along by a given number of characters.

\begin{itemize}
    \item Ciphers that use non-random keys are open to a \textbf{cryptanalytic attack} and can be solved given enough time and resources. \textbf{Frequency analysis} is a common technique used to break a cipher.
    \item A \textbf{true random sequence} must be collected from a physical and unpredictable phenomenon.

        E.g radioactive decay.
\end{itemize}

The \textbf{Vernam cipher} is an implementation of \textbf{one-time pad ciphers}, offering \textbf{perfect security} when used properly.
\begin{itemize}
    \item The encryption key or one-time pad must be \textbf{equal to or longer in characters than the plaintext}, be \textbf{truly random} and \textbf{used only once}.
    \item The sender and recipient must meet in person to \textbf{securely share the key} and \textbf{destroyed after encryption or decryption}.
\end{itemize}

Since the key is random, so will the \textbf{distribution of the characters} - so no amount of cryptanalysis will produce meaningful results.
\begin{enumerate}
    \item A \textbf{bitwise XOR operation} is carried between the binary representation of each character of the plaintext and the corresponding character of the one-time pad.
    \item A bitwise XOR operation is carried out on the ciphertext using the \textbf{same one-time pad} to restore it to plaintext.
\end{enumerate}
