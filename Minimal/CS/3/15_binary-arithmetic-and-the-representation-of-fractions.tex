\subsection{Binary Arithmetic and the Representation of Fractions}

An \textbf{overflow error} occurs when a \textbf{carry from the most significant bit} requires a 9th bit, but only 8 bits are used to store the result of an addition.

\subsubsection*{Signed and Unsigned Binary Numbres}

\begin{itemize}
    \item An \textbf{unsigned representation} of binary number can only represented positive numbers.
    \item A \textbf{signed representation} can represent both positive and negative numbers.
\end{itemize}

\textbf{Two's complement} is a representation of signed binary number.

It works similar to numbers on an \textbf{analogue counter} - moving the wheel forward 1 will read 0001, back one the reading becomes 9999, which is interpreted as -1.

The range that can be represented by two's complement using $n$ bits is given by
$$-2^{n-1}\dots 2^n-1$$
With 8 bits, the maximum range that can be represented as -128 (1000 0000) to 127 (0111 1111). The leftmost bit is used as a \textbf{sign bit} to indicate whether a number is negative.

To negate a binary number
\begin{enumerate}
    \item Flip the bits.
    \item Add one.
\end{enumerate}

\textbf{Binary subtraction} can be done using the \textbf{negative two's complement number}, then adding the second number - the carry on the addition is ignored.

\subsubsection*{Fixed Point Binary Numbers}

Fixed point binary numbers is a way to \textbf{represent fractions} in binary. A \textbf{binary point} is used to separate the \textbf{whole place values} from the \textbf{fractional part} on the number line.

Some fraction \textbf{cannot be represented} at all, as they will require an infinite number of bits to the right of the point. The number of fractional places would therefore be \textbf{truncated} and the number will not be accurately stored, causing \textbf{rounding errors}.

Two digits after the point and only represent 0, $1/4$, $1/2$, $3/4$ and nothing in between.
