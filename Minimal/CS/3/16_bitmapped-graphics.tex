\subsection{Bitmapped Graphics}

A bitmap image contains many \textbf{pixels} that make up the whole image.
\begin{itemize}
    \item A pixel is the \textbf{smallest identifiable area} of an image.
    \item Each pixel is attributed a binary value which \textbf{represents a single colour}.
\end{itemize}

The \textbf{resolution} of an image determine the number of pixels within it. The greater the number of pixels it contains, the sharper the image, as the pixels must become smaller to fit its boundaries.

Resolution of an image can be expressed as
\begin{itemize}
    \item \textbf{Width in pixels $\times$ height in pixels}.
    \item Pixels per inch - indicating the density of the pixels.
\end{itemize}

The number of bits per pixel is referred to as the \textbf{colour depth}. The number of bits determines the \textbf{number of combinations}, this determines the \textbf{number of colours} that a pixel can represent.

\textbf{Metadata} is data about data - details such as the image \textbf{width in pixels, height in pixels and colour depth}.

\subsubsection*{Vector Graphics}

Vector images are made of \textbf{geometric shapes or objects} such as lines, curves, arcs and polygons. A vector file stores only the necessary details about each shape in order to \textbf{redraw the object} when the file loads.

E.g. the properties of an image of a circle.
\begin{itemize}
    \item The \textbf{position} of its centre,
    \item Its \textbf{radius},
    \item Fill and line colour,
    \item Line weight.
\end{itemize}

These properties are stored in a \textbf{drawing list} which specifies how to redraw the object.

Regardless of how large an image is draw, the image will \textbf{always be sharp}, and the amount of data required to store the image will not change.

\textbf{Advantages of vector graphics}
\begin{itemize}
    \item Usually has a much smaller file size.
    \item Will \textbf{scale perfectly}.
    \item Used for logos, so the image will be sharp when printed on anything from a business card to a billboard.
\end{itemize}

\textbf{Disadvantages of vector graphics}
\begin{itemize}
    \item Cannot easily replicate an image with continuous areas of changing colour.
    \item Individual pixels cannot be changed.
\end{itemize}
