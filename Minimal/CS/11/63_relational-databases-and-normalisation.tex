\subsection{Relational Databases and Normalisation}

A \textbf{relational database} is a collection of tables which relationships are modelled by shared attributes.

Tables can be linked through the use of \textbf{common attributes}. This attribute must be a primary key of one of the tables, and is know as a \textbf{foreign key} in the second table.

\subsubsection*{Normalisation}

Normalisation is a process used to come up with the \textbf{best possible design} for a relational database. Tables are organised in a way that
\begin{itemize}
    \item No data is \textbf{unnecessarily duplicated}.
    \item Data is \textbf{consistent} throughout the database. This means anomalies will not arise when the data is inserted, amended or deleted.

        Consistency should be an automatic consequence of no holding any duplicated data.
    \item The structure of each table is enough to allow you to enter \textbf{as many or as few items} as required.
    \item The structure should enable a sure to make all kinds of \textbf{complex queries} relating data from different tables.
\end{itemize}

\begin{enumerate}
    \item A table is in \textbf{first normal form} if it contains \textbf{no repeating attribute} or group of attributes.
    \item A table is in \textbf{second normal form} if it is in first normal form and contains \textbf{no partial dependencies}.

        A partial dependency is where one or more attributes \textbf{depends on only part of the primary key}, which can only occur if the primary key is a \textbf{composite key}.
    \item A table is in \textbf{third normal form} if it is in second normal form and contains \textbf{no non-key dependencies}.

        A non-key dependency is one where the value of an attribute is determined by the value of another attribute which is not part of the key.
\end{enumerate}

\subsection*{Advantages of Normalisation}

\begin{itemize}
    \item Easier to \textbf{maintain and change}.

        \textbf{Data integrity} is maintained since there is no unnecessary duplication of data, it will also be impossible to reference a non-existing record on another table.

    \item \textbf{Faster} sorting and searching, as normalisation produce \textbf{smaller tables with fewer fields}, searching is faster because less data is involved.

        Holding data once saves storage space.

    \item A normalised data base with \textbf{correctly defined relationships} between tables will not allow records in a table on the `one' side of a one-to-many relationship to be deleted accidentally.
\end{itemize}
