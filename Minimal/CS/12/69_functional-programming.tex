\subsection{Functional Programming}

A \textbf{programming paradigm} is a style of computer programming, different programming languages support tackling problems in different ways.
\begin{itemize}
    \item \textbf{Procedural programming} have a series of instructions that tell that computer what to do with the input in order to solve the problem.

        \textbf{Structured programming} is a type of procedural programming which uses the programming construct of \textbf{sequence, selection, iteration and recursion}. It uses modular techniques to split large programs into manageable tasks.
    \item \textbf{Object-oriented programming} makes it possible to \textbf{abstract details of implementation} away from the user, make code \textbf{reusable} and programs \textbf{easy to maintain}.
    \item \textbf{Declarative programming} is where you write statements to describe the program to be solved, and the language implementation decides the best way of solving it.
    \item In \textbf{functional programming}, functions are used as the fundamental building blocks of a program. Statements are written as a \textbf{series of functions} which accept input data as arguments and return an output.
\end{itemize}

A function is a mapping from a set of inputs, called the domain, to a set of possible outputs, known as the co-domain.

The process of giving particular inputs to a function is known as \textbf{functional application}.

In functional programming, a \textbf{first-class object} is an object which may
\begin{itemize}
    \item Appear in expressions.
    \item Be assigned to a variable.
    \item Be assigned as an argument.
    \item Be returned in a functional call.
\end{itemize}

\subsubsection*{Features of Functional Programming Languages}

\begin{itemize}
    \item \textbf{Statelessness}: In a functional programming language, the \textbf{values of variables} cannot change. Variables are staid to be \textbf{immutable}, and the program is said to be \textbf{stateless}.
    \item \textbf{No side effects}: The only thing that a function can do is calculate something and return a result.

        As a consequence of not being able to change the value of an object, a function that is called twice with the same parameters will always return the same result. This is called \textbf{referential transparency}.
        \begin{itemize}
            \item Makes it relatively easy for programmers to write correct, bug-free programs.
            \item A simple function can be proved to be correct, then more complex functions can be built using these functions.
        \end{itemize}
\end{itemize}

\begin{itemize}
    \item \textbf{Functional composition}: combine two functions to get a new function.
    \item \textbf{Types} are sets of values.
    \item \textbf{Typeclasses} are sets of types.
    \item A \textbf{type variable} represents any type.
\end{itemize}
