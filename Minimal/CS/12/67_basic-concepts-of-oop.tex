\subsection{Basic Concepts of Object-oriented Programming}

A program written in \textbf{procedural languages} is written using a series of \textbf{step-by-step} instructions on how to solve the problem.
\begin{itemize}
    \item Broken down into a number of \textbf{smaller modules}.
    \item The program consists of a series of calls to \textbf{procedures or functions}.
    \item Which in turn call other procedures or functions.
\end{itemize}

In \textbf{object-oriented programming}, the world is viewed as a \textbf{collection of objects}, each responsible for its own data and the operations on that data.
\begin{itemize}
    \item A program creates \textbf{objects}, and
    \item Allows the objects to \textbf{communicate with each other} through sending and receiving \textbf{messages}.
    \item All processing is done by objects.
\end{itemize}

Each object has its own \textbf{attributes, state and behaviours} (actions that can be performed by the object).

\subsubsection*{Classes}

A class is a \textbf{template for an object}, it defines
\begin{itemize}
    \item An \textbf{attribute} is data associated with the class.
    \item A \textbf{method} is a functionality of the class.
    \item A \textbf{constructor} is used to create objects.
\end{itemize}
The principle of \textbf{information hiding}: other classes cannot directly access the attributes of another class declared private.

\textbf{Instantiation} is the creation of objects - multiple instances of a class each share identical methods and attributes, but the values of attributes will be unique to each instance.

An object \textbf{encapsulates} both its state and its behaviours, so that the attributes and behaviours of one object cannot affect the way another object functions.

\subsubsection*{Inheritance}

\textbf{Subclasses} can inherit data and behaviour from a \textbf{superclass}.

\begin{itemize}
    \item The \emph{"is a"} rule asks \emph{"is object A an object B"} before it can inherit from the object.
\end{itemize}
