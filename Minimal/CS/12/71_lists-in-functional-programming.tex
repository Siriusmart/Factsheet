\subsection{Lists in Functional Programming}

A list is a \textbf{collection of similar elements} of a similar type, enclosed in square brackets. It is composed of a \textbf{head} and a \textbf{tail} - the head is the first element of the list, the tail is the remainder of the list.
\begin{itemize}
    \item The function \textbf{null} tests for an empty list.
        \begin{minted}{haskell}
        null []  -- True
        null [1] -- False
        \end{minted}
    \item \textbf{Prepending} means adding an element to the front of a list.
        \begin{minted}{haskell}
        5:[4,3,2,1] -- [5,4,3,2,1]
        \end{minted}
    \item \textbf{Appending} means adding an element to the end of a list.
        \begin{minted}{haskell}
        [1,2,3,4] ++ [5] -- [1,2,3,4,5]
        \end{minted}
    \item \textbf{Length} finds the length of the list.
        \begin{minted}{haskell}
        length [1,2,3,4,5] -- 5
        \end{minted}
\end{itemize}
