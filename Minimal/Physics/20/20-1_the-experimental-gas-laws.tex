\subsection{The Experimental Gas Laws}

The \textbf{pressure} of a gas is the force per unit area that the gas exerts \textbf{normally} on a surface. The unit of pressure is that \textbf{pascal}.

\subsubsection*{Boyle's Law}

Boyle's law states for a \textbf{fixed mass} of gas at \textbf{constant temperature}.
$$pV=\text{constant}$$

\subsubsection*{Charles' Law}

Charles' law states for a \textbf{fixed mass} of gas at \textbf{constant pressure}.
$$\frac{V}{T}=\text{constant}$$

The graph of \textbf{volume against temperature} in kelvins is a straight line through the origin.

\begin{itemize}
    \item \textbf{Isobaric change} is any change at \textbf{constant pressure}.
    \item When work is done to \textbf{change the volume} of a gas energy must be transferred to keep the pressure constant.
\end{itemize}
$$\text{Work done}=p\Delta V$$

\subsubsection*{The Pressure Law}

The pressure law shows how the pressure of a fixed mass of gas at \textbf{constant volume} changes with temperature.
$$\frac{p}{T}=\text{constant}$$
