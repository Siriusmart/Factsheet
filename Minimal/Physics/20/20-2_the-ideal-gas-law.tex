\subsection{The Idea Gas Law}

The molecules of a gas \underline{move at random with different speeds}.
\begin{itemize}
    \item The pressure of a gas on a surface is due to \textbf{gas molecules hitting the surface}. Each impact exerts a tiny force on the surface.
    \item \textbf{Brownian motion} - smoke particles wriggle around unpredictably because it is bombarded unevenly and randomly by individual molecules. The particle experiences forces due to these impacts which changes its velocity at random.
\end{itemize}

\subsubsection*{Molar Mass}

\textbf{Avogadro's constant} $N_A$ is defined as the number of atoms in exactly 12g of $^{12}_{\hspace{4pt}6}\text{C}$.
$$N_A=6.02\times10^{23}$$

\begin{itemize}
    \item One \textbf{mole} of substance consisting of identical particles is defined as the quantity of substance that contain $N_A$ particles.
    \item The number of moles in a given quantity of a substance is its \textbf{molarity}. The unit of molarity is the \text{mol}.
    \item The \textbf{molar mass} of a substance is the mass of 1 mol of that substance. The unit of molar mass is kg\,mol$^{-1}$.
\end{itemize}

\subsubsection*{The Ideal Gas Equation}

An \textbf{ideal gas} is a gas that obeys \textbf{Boyle's law}.

The three experimental gas laws combine to give
$$\frac{pV}{T}=\text{constant}$$

Equal volumes of ideal gases at the same temperature and pressure contain equal number of moles.

For one mole of ideal gas
$$\frac{pV}{T}=8.31\text{J\,mol$^{-1}$\,K$^{-1}$}$$
This value is called the \textbf{molar gas constant} $R$.

A graph of $pV$ against $T$ for $n$ moles is a \textbf{straight line through absolute zero} and has a gradient equal to $nR$.

The combined gas law can be written as the \textbf{ideal gas equation}.
$$pV=nRT$$

Using $n=\dfrac{N}{N_A}$, the \textbf{Boltzmann constant} $k=\dfrac{R}{N_A}$ where $N$ is the number of molecules.
$$pV=NkT$$
