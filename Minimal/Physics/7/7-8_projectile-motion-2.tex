\subsection{Projectile Motion 2}

Any form of motion where an object \textbf{experiences a constant acceleration in a different direction to its velocity} will be like projectile motion.

\begin{itemize}
    \item A ball rolling across an \textbf{inclined board} will be a projectile path.
    \item A beam of electrons directed between two oppositely charged parallel plates.
\end{itemize}

The path of a projectile is \textbf{parabolic}.

\subsubsection*{Drag Force}

A projectile moving in air experiences a \textbf{drag force} because of the resistance of air it passes through.
\begin{itemize}
    \item Acts in the \textbf{opposite direction} to the direction of motion of the projectile.
    \item Increases as the projectile's speed increases.
    \item Has a \textbf{horizontal component} that reduces horizontal speed and the range.
    \item Has a \textbf{vertical component} that reduces maximum height and makes its descent \textbf{steeper than its ascent}.
\end{itemize}

The \textbf{shape of the projectile} affects the drag force and may cause a \textbf{lift force} on the projectile.
