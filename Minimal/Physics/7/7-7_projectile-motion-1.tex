\subsection{Projectile Motion 1}

A \textbf{projectile} is any object acted upon only be the force of gravity.
\begin{itemize}
    \item The acceleration of the object is \textbf{always downwards} and \textbf{equal to $\mathbf{g}$} because the force of gravity acts downwards.
    \item Horizontal velocity of the object is constant because the acceleration of the object \textbf{does not have a horizontal component}.
    \item The motion in the horizontal and vertical direction are \textbf{independent} of each other.
\end{itemize}
\begin{align*}
    v&=u-gt\\
    y&=ut-\frac{1}{2}gt^2
\end{align*}

\subsubsection*{Horizontal Projection}
If the initial project of a stone off a cliff is horizontal.
\begin{itemize}
    \item Its path through air \textbf{becomes steeper} as it drops.
    \item The faster it is projected, the further away it will fall into the sea.
    \item The time taken for it to fall into the sea \textbf{does not depend on how fast it is projected}.
        \begin{itemize}
            \item The downward acceleration for each speed is $g$.
        \end{itemize}
\end{itemize}

A camera can be used to record the motion of projectiles. Where A is released from rest and B was given an initial horizontal projection.
\begin{itemize}
    \item The \textbf{horizontal position} of B change by equal distance per unit time.
        \begin{itemize}
            \item The horizontal component of B is constant.
        \end{itemize}
    \item At any instant, A is at the \textbf{same level} as B.
        \begin{itemize}
            \item A and B have the same vertical component of velocity at any instant.
        \end{itemize}
\end{itemize}
