\subsection{Total Internal Reflection}

When light ray travels \underline{from glass into air}, it refracts away from the normal.
\begin{itemize}
    \item The \textbf{critical angle} is the angle of incidence where the light ray refracts \underline{along the boundary}.
    \item The ray undergoes \textbf{total internal reflection} if the angle of incidence is increased further.
\end{itemize}

Conditions for TIR:
\begin{itemize}
    \item Incident substance has a \textbf{larger refractive index} than the other substance.
    \item The angle of incidence \textbf{exceeds the critical angle}.
\end{itemize}

The critical angle satisfies
$$\sin\theta_c=\frac{n_2}{n_1}$$

\textbf{Diamonds} sparkle with different colours when white light is directed at them
\begin{itemize}
    \item Very high refractive index \textbf{separate the colours} more than any other substance.
    \item A light ray may \textbf{totally internally reflect many times} before it emerges so the colours spread out more and more.
\end{itemize}

\subsubsection*{Optic Fibres}

Optical fibres are used in \textbf{medical endoscopes} to see inside the body, and in communications to \textbf{carry light signals}.
\begin{itemize}
    \item The light ray is totally internally reflected each time it reaches the fibre boundary.
    \item Even when the fabric bends, \underline{unless the radius of the bend is too small}.
\end{itemize}

The fibres are \textbf{highly transparent} to minimised the absorption of light, as \textbf{light loss} would reduce the amplitude of the pulses.
\begin{itemize}
    \item TIR takes place at the \textbf{core-cladding boundary}.
        \begin{itemize}
            \item If there were no cladding, two fibres would be in direct contact and light would \textbf{cross from one fibre to the other}.
            \item And the signal would reach the wrong destination.
        \end{itemize}
    \item The core must be narrow to prevent \textbf{modal dispersion}.
        \begin{itemize}
            \item This occurs in a \textbf{wide core} as light travelling along the axis travels a \textbf{shorter distance per metre of fabric} than light that repeatedly undergoes TIR.
            \item If a pulse of light becomes too long, it would \textbf{merge with the next pulse}.
        \end{itemize}
\end{itemize}

\textbf{Material dispersion} occurs if white light is used instead of \textbf{monochromatic light}, as the speed of light in glass depends on the wavelength of the light.
