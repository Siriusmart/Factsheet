\subsection{More about Interference}

Double slits are \textbf{coherent sources} because they emit light waves
\begin{itemize}
    \item Of the \textbf{same frequency}, and
    \item With a \textbf{constant phase difference}.
\end{itemize}
Each wave crest from the single slit always passes trough one of the double slits a \underline{fixed time} after it passes through the other slit. Therefore emits wavefronts with a \textbf{constant phase difference}.

Light from two filament lamps do not form an interference pattern because the light sources \textbf{emit light waves at random} - the points of cancellation and reinforcement would \textbf{change at random}, so no interference pattern is possible.

\subsubsection*{Light Sources}
\begin{itemize}
    \item \textbf{Discharge tubes} produce light with a \textbf{dominant colour}.
        \begin{itemize}
            \item A \textbf{sodium vapour lamp} is in effect a \underline{monochromatic light source}.
        \end{itemize}
    \item Light from a \textbf{filament lamp} is composed of colours of the spectrum and therefore covers a \textbf{continuous range} of wavelengths.

    \item \textbf{Laser light} is
        \begin{itemize}
            \item Monochromatic
            \item Coherent
        \end{itemize}
\end{itemize}

Each component of white light produces its own fringe pattern.
\begin{itemize}
    \item The \textbf{central fringe} is white
        \begin{itemize}
            \item Every colour contributes at the centre of the pattern.
        \end{itemize}
    \item The \textbf{inner fringes} are tinged with blue on the inner side and red on the outer side.
        \begin{itemize}
            \item Red fringes are more spaced out, than blue fringes.
            \item The two fringe patterns \underline{do not overlap directly}.
        \end{itemize}
\end{itemize}
