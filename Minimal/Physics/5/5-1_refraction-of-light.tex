\subsection{Refraction of Light}

\begin{itemize}
    \item In a diagram, \textbf{light rays} represent the direction of travel of wavefronts.
    \item The \textbf{normal} is an imaginary line perpendicular to a surface.
\end{itemize}

Refraction is the \textbf{change of direction} that occurs when light passes \underline{at an angle} across a boundary between two transparent materials.

A \textbf{ray box} can be used to direct a light ray into a glass block.
\begin{itemize}
    \item Bend \textbf{towards the normal} when it passes from air into glass.
    \item Bend \textbf{away from the normal} when it passes from glass into air.
\end{itemize}
No refraction takes place if the incident light ray is \textbf{along the normal}.
\begin{itemize}
    \item \textbf{Angle of refraction} $r$ is always less than \textbf{angle of incident} $i$.
    \item The radio $\sin i/\sin r$ is the same for each ray, this known as \textbf{Snell's law}.
\end{itemize}
$$\text{the refractive index of the substance}\,n=\frac{\sin i}{\sin r}$$
\textbf{Partial reflection} occurs when a light ray enters a refractive substance.
