\subsection{Binding Energy}

\begin{itemize}
    \item The \textbf{binding energy} of the nucleus is the \textbf{work done that must be done to separate} a nucleus into its constituent neutrons and protons.
    \item The \textbf{mass defect} $\Delta m$ of a nucleus is defined as the difference between the mass of the separated nucleons and the mass of the nucleus.
\end{itemize}
\begin{align*}
    \Delta m&=Zm_p+(A-Z)m_n-M_\text{nuc}\\
    \text{Binding energy}&=\Delta mc^2
\end{align*}

\subsubsection*{$\alpha$ Particle Tunnelling}

If two protons and two neutrons inside a sufficiently large nucleus bing together as a cluster, because the binding energy of an $\alpha$ particle is very large, the $\alpha$ particle therefore gains enough energy to give it a small probability of quantum tunnelling from the nucleus.

\subsubsection*{Nuclear Stability}

The \textbf{binding energy per nucleon} of a nucleus is the average work done per nucleon to remove all the nucleons from a nucleus.

The binding energy per nucleon is a \textbf{measure of stability} of a nucleus.
\begin{itemize}
    \item Comparing the binding energies per nucleon of two different nuclides, the nuclide with more binding energy per nucleon is more stable.
    \item The binding energies per nucleon has a \textbf{maximum value} between $A=50$ and $A=60$.

        \textbf{Nuclear fission} occurs when a large unstable nucleus splits into two fragments, the binding energy per nucleon increases in the process.
        
        \textbf{Nuclear fusion} fuse small nuclei together to form a large nucleus. The produce nucleus has more binding energy per nucleon.
\end{itemize}
