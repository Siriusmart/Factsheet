\subsection{Fission and Fusion}

\subsubsection*{Induced Fission}

Fission of a nucleus occurs when a nucleus splits into two approximately equal fragments.

Induced fission occurs when $^{235}_{\hspace{4pt}92}\text{U}$ is bombarded with neutrons.

Each fission event releases energy and two or three neutrons.
\begin{itemize}
    \item \textbf{Fission neutrons} released in a fission event, are capable of causing a further fission event as a result of a collision with another $^{235}_{\hspace{4pt}92}U$ nucleus.
    \item A \textbf{chain reaction} is where fission neutrons produced further fission events which release fission neutrons and so on.
\end{itemize}

Energy is released in a fission event, because the fragments \textbf{repel each other}, they therefore \textbf{gain kinetic energy}.

The energy released is equal to the change in binding energy.

\subsubsection*{Nuclear Fusion}

Nuclear takes place when two nuclei combine to form a bigger nucleus - the binding energy per nucleon of the produce is greater than of the initial nuclei.

Nuclear fusion can only take place if the two nuclei \textbf{collide at high speed}, this is necessary to \textbf{overcome the electrostatic repulsion} between the two nuclei. So they become close enough to interact through the strong nuclear force.

\textbf{Solar energy} is produced as a result of fusion reactions inside the sun.
