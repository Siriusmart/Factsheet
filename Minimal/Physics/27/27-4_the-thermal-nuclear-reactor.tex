\subsection{The Thermal Nuclear Reactor}
\begin{itemize}
    \item The \textbf{reactor core} is a steel container with \textbf{fuel rods} spaced evenly inside.

        The reactor core also contains \textbf{control rods} and a \textbf{coolant}, such as pressurised water.
    \item The control rods are connected to a \textbf{heat exchanger} through steel pipes.
    \item A \textbf{pump} force coolant through the reactor core and the heat exchanger, where it is used to raise steam to drive the turbines that turn the electricity generators in the power station.
    \item The \textbf{fuel rods} contains enriched uranium with about 2-3\% U-235.
    \item The \textbf{control rods} absorb neutrons, the depth of the control rods is automatically adjusted to keep the number of neutrons in the core constant - so exactly one fission neutron per fission event on average goes on to produce further fission.

        This keeps the rate of release of fission energy constant.
    \item Fission neutrons need to be slowed down significantly to cause further fission of U-235.

        Fuel rods are surrounded by a \textbf{moderator} so the neutrons are \textbf{slowed down by repeated collisions} with the moderator atoms.
    \item In a \textbf{thermal nuclear reactor}, the fission neutrons are slowed down to kinetic energy comparable to the kinetic energies of the moderator molecules.
\end{itemize}

For a chain reaction to occur, the mass of the fissile material must be greater than a minimum mass referred to as the \textbf{critical mass}.
\begin{itemize}
    \item Some neutrons escape from the fissile material without causing fission.
    \item If the mass is less than the critical mass, too many fission neutrons escape because the surface area to mass ration of the material is too high.
\end{itemize}

\subsubsection*{Safety Features}

\begin{itemize}
    \item \textbf{Thick steel vessel} of the reactor core, to withstand high pressure and temperature in the core.

        Also absorbs $\beta$ radiation, some of $\gamma$ radiation, and neutrons from the core.
    \item \textbf{Tick concrete walls} of the building the core is in, which absorbs neutrons and $\gamma$ radiation that escaped the reactor vessel.
    \item \textbf{Emergency shut-down system} to insert the control rods fully into the core to stop fission completely.
    \item Sealed fuel rods are \textbf{inserted and removed remotely} into/from the core.
\end{itemize}

Spent fuel rods are much more dangerous, before use U-235 and U-238 emit only $\alpha$ radiation, which is absorbed by the fuel cans.

After use emit $\beta$ and $\gamma$ radiation due to the many neutron-rich fission products.

\subsubsection*{Radioactive Waste}

Radioactive waste is categorised as high, intermediate, or low-level waste \textbf{according to its activity}.
\begin{itemize}
    \item \textbf{High-level radioactive waste}, e.g spent fuel rods are \textbf{removed by remote control} and \textbf{stored underwater} in cooling ponds for years, where they will continue to release heat due to radioactive decay.
    \item \textbf{Intermedia-level waste} are \textbf{encased in concrete} and stored in specifically constructed buildings with walls of reinforced concrete.
    \item \textbf{Low-level waste} e.g. protective clothing is sealed in metal drums in buried in large trenches.
\end{itemize}
