\subsection{Terminal Speed}

Any object moving through a fluid experiences a forces that \textbf{drags on it}.

The \textbf{drag force} depends on
\begin{itemize}
    \item The \textbf{shape} of the object.
    \item Its \textbf{speed}.
    \item The \textbf{viscosity} of the fluid.
\end{itemize}

\subsubsection*{Motion of an Object Falling in a Fluid}

\begin{enumerate}
    \item The speed of an object released from rest in a fluid \textbf{increases as it falls}.
    \item The \textbf{resultant force} on the object is the difference between its weight and the drag force.
    \item As the \textbf{drag force increases}, the resultant force decreases, so \textbf{acceleration becomes less} as it falls.
    \item If it continues falling, it attains \textbf{terminal speed}.
        \begin{itemize}
            \item The drag force on it is \underline{equal and opposite to its weight}.
            \item Its \underline{acceleration is zero} and it speed \underline{remains constant} as it falls.
        \end{itemize}
\end{enumerate}
$$\text{The acceleration of the object}=g-\frac{D}{m}$$
\begin{itemize}
    \item The \textbf{initial acceleration} is $g$.
    \item At terminal speed, the potential energy of the object is transferred into \textbf{internal energy of the fluid} by the drag force.
\end{itemize}

\subsubsection*{Motion of a Powered Vehicle}
The resultant force on a powered vehicle is $F_E-F_R$

\begin{itemize}
    \item $F_E$ is the \textbf{motive force} provided by the engine.
    \item $F_R$ is the \textbf{resistive force} opposing the motion of the vehicle.
\end{itemize}

Its acceleration is
$$a=\frac{F_E-F_R}{m}$$
