\subsection{Force and Acceleration}

An \textbf{air track} allows motion to be observed in the \textbf{absence of friction}.
\begin{itemize}
    \item The glider floats on a cushion of air.
    \item And the track is \textbf{level}.
\end{itemize}
The glider moves at a \textbf{constant velocity} along the track because friction is absent.

\textbf{Newton's First Law of Motion}: Objects either stay at rest or moves with constant velocity unless acted on by a force.
\begin{itemize}
    \item When an object is acted on by a \textbf{resultant force}, the result is to change the object's velocity.
    \item An object moving at \textbf{constant velocity} is either \underline{acted on by no force} or the forces acting on it are \underline{balanced}.
\end{itemize}

\textbf{Newton's Second Law of Motion}: $F$ is proportional to $ma$.

By defining the unit of force, the \textbf{newton} as the amount of force that will give an object of mass 1kg an acceleration of 1ms$^2$, we can write.
$$F=ma$$

\subsubsection*{Weight}
The force of gravity on the object is its \textbf{weight}.
$$W=mg$$
$g$ is also referred to as the \textbf{gravitational field strength}.

\begin{itemize}
    \item An object in equilibrium has a supporting force on it equal and opposite to its weight.
    \item The mass of an object is a measure of its \textbf{inertia} - resistance to change of moment.
\end{itemize}
