\subsection{Electromotive Force and Internal Resistance}

The \textbf{internal resistance of a source} of electricity is due to opposition to the flow of charge through the source.
\begin{itemize}
    \item \textbf{Electromotive force} $\epsilon$ of the source is the electrical energy per unit charge produced by the source.
        $$\epsilon=\frac{\Delta E}{\Delta Q}$$
    \item The \textbf{pd across terminals} of the source is the electrical energy per unit charge delivered by the source when it is in a circuit.
\end{itemize}

The \textbf{internal resistance} of a source is the \textbf{loss of potential difference per unit current} in the source when current passes through the source.
$$\epsilon=IR+Ir$$
The \textbf{lost pd} inside the cell is equal to the difference between the cell emf and the terminal pd.

\subsubsection*{Power Supplied by Cell}
$$P=I\epsilon=I^2R+I^2r$$

Since $I=\dfrac{\epsilon}{R+r}$, the power delivered to $R=\dfrac{\epsilon^2}{(R+r)^2}R$

The peak of the \textbf{power curve} is at $R=r$ - \textbf{maximum power} is delivered to the load when the load resistance is equal to the internal resistance of the source.

\subsubsection*{Internal Resistance Measurements}

\begin{itemize}
    \item A \textbf{voltmeter} connected directly across the cell to measure the \textbf{terminal pd}.
    \item An \textbf{ammeter} in series with the cell to measure the \textbf{cell current}.
    \item A \textbf{variable resistor} to adjust the current.
\end{itemize}

The measurements of terminal pd and current can be plotted on a graph.
\begin{itemize}
    \item Terminal pd is \textbf{equal to the emf} at zero current.
    \item Graph is a \textbf{straight line with negative gradient}.
\end{itemize}
$$V=\epsilon-Ir$$

The \textbf{internal resistance} is given by $r=\dfrac{V_1-V_2}{I_2-I_1}$
