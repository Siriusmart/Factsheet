\subsection{Electric Potential}

The electric potential at a pertain position in any electric field is defined as the \textbf{work done per unit positive charge} on  a positive test charge when it is moved from infinity to that point.
$$V=\frac{E_p}{Q}$$
the unit of electric potential is the \textbf{volt}, equal to 1J\,C$^{-1}$.

\subsubsection*{Potential Gradients}

The equipotentials for an electric field are like equipotential for a gravitational field.
\begin{itemize}
    \item Both are lines of \textbf{constant potential energy} for the appropriate test object.
    \item In one case a test charge, the other case a test mass.
\end{itemize}

The \textbf{potential gradient} at any point in an electric field is the \textbf{change of potential per unit change of distance} in a given direction.
\begin{itemize}
    \item The closer the equipotentials are, the greater the potential gradient is at right angles to the equipotentials.
    \item If the field is uniform (between two oppositely charged parallel plates), the potential gradient is
        \begin{itemize}
            \item Constant.
            \item The potential \textbf{increases in the opposite direction} to the electric field.
            \item Equal to $\dfrac{V}{d}$
        \end{itemize}
\end{itemize}

The electric field strength is equal to the \textbf{negative of the potential gradient}.
$$E=-\frac{dV}{dx}$$
