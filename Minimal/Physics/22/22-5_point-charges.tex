\subsection{Point Charges}

\begin{itemize}
    \item A \textbf{point charge} is a convenient expression for a charged object in a situation where distances under consideration are \textbf{much greater than the size of the object}.
    \item A \textbf{test charge} in an electric field is a \textbf{point charge} that does not alter the electric field in which it is placed.
\end{itemize}
$$E=\frac{F}{q}=\frac{1}{4\pi\varepsilon_0}\frac{Q}{r^2}$$

If a test charge is in an electric field due to several point charges, each charge exerts a force on the test charge. The \textbf{resultant force per unit charge} $\dfrac{F}{q}$ on the test charge gives the \textbf{resultant electric field strength} at the position of the test charge.

\subsubsection*{Radial Electric Fields}

\begin{itemize}
    \item The \textbf{electric field lines} of force surrounding a point charge $Q$ are radial.
    \item The \textbf{equipotentials} are therefore concentric circles on $Q$.
\end{itemize}

For a charged sphere, we can say the charge is at the centre of the sphere.

The \textbf{electric potential} at distance $r$ from $Q$
$$V=\frac{1}{4\pi\epsilon_0}\frac{Q_1Q_2}{r}$$

\begin{enumerate}
    \item Because the electric field strength is the \textbf{force per unit charge} on a small positive test charge.
    \item The area under a graph of electric field strength against distance gives the \textbf{work done per unit charge}.
    \item Which is the \textbf{change in potential} when a positive test charge is moved through the section represented.
\end{enumerate}
