\subsection{Field Patterns}

Like charges repel, unlike charges attract.

\begin{itemize}
    \item \textbf{Electrons} are responsible for charging in most situations.

        An \textbf{uncharged atom} contains equal numbers of protons and electrons, an \textbf{uncharged solid} contains equal numbers of electrons and protons.
    \item \textbf{Electrical conductors} contains lots of \textbf{free electrons} - they are not attached to any one atom.

        An isolated conductor can be charged by direct contact with any charged object.
    \item \textbf{Electrically insulating material} do not contain free electrons - all electrons in an insulator are attached to individual atoms.

        Some insulator are \textbf{easy to charge} because their surface atoms easily gain or lose electrons.
\end{itemize}

The \textbf{gold leaf electroscope} is used to detect charge.
\begin{enumerate}
    \item If a charged object is in contact with the metal cap, some of the charge of the object \textbf{transfer to the electroscope}.
    \item As a result, the \textbf{gold leaf} and the \textbf{metal stem} attached to the cap \textbf{gain the same type of charge} and the leaf rises because it is repelled by the stem.
\end{enumerate}

\subsubsection*{Field Lines and Patterns}

Any two charged object exerts equal and opposite forces on each other without being directly in contact.
\begin{itemize}
    \item An electric field is said to surround each charge.
    \item The path a \textbf{free positive test charge} is called a \textbf{field line}.

        The direction of an electric field line is the direction a positive test charge would move along.
\end{itemize}
