\subsection{Electric Field Strength}

The electric field strength $E$ at a point in the field is defined as the \textbf{force per unit charge} on a positive test charge placed at that point.

The unit of $E$ is the \textbf{newton per coulomb} N\,C$^{-1}$.
$$E=\frac{F}{Q}$$

\subsubsection*{Uniform Electric Field}
The field lines between two oppositely charged flat plates are
\begin{itemize}
    \item \textbf{Parallel} to each other.
    \item At \textbf{right angle} to the plates.
    \item From the positive plate to the negative plate.
\end{itemize}

The field between the plates in \textbf{uniform} because the electric field strength has the same magnitude and direction everywhere between the plates.
$$E=\frac{\Delta V}{\Delta d}$$

\subsubsection*{Field Factors}

\begin{itemize}
    \item An electric field exists near any charged body.
    \item The greater the charge on the body, the stronger the electric field is.
    \item For a \textbf{charged metal conductor}, the charge on it spread across the surface.
    \item The \textbf{more concentrated} the charge is on the surface, the greater the strength of the electric field strength is above the surface.
\end{itemize}

The electric field between two oppositely charged parallel plates depends on the \textbf{concentration of charge} on the surface of the plates. The charge on each plate is spread evenly across the surface of the plate.
$$E\propto\frac{Q}{A}$$
And introduce a constant of proportionality such that
$$\frac{Q}{A}=\varepsilon_0E$$
where $\varepsilon_0=8.85\times10^{-12}$F\,m$^{-1}$, also called the \textbf{permittivity of free space}. It represents that \textbf{charge per unit area} on a surface in a vacuum that produces an electric field strength of one volt per metre between the plates.
