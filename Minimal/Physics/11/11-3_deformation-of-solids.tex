\subsection{Deformation of Solids}

The \textbf{elasticity} of a solid material is its ability to regain its shape after it has been deformed or distorted once the forces that deformed it have been released.
\begin{itemize}
    \item \textbf{Tensile deformations} stretch an object.
    \item \textbf{Compressive deformations} compresses an object.
\end{itemize}

A \textbf{tensile-extension graph} shows how easily different materials stretch.
\begin{itemize}
    \item A \textbf{steel spring} gives a straight line, showing it obeys Hooke's law.
    \item A \textbf{rubber band} at first extends easily, then becomes very difficult to stretch further when it becomes fully stretched.
    \item A \textbf{polythene strip} stretches easily after its initial stiffness is overcome. After extending a little it becomes difficult to stretch again.
\end{itemize}

For a wire of length $L$ and cross section $A$ under tension.
\begin{itemize}
    \item \textbf{Tensile stress} $\sigma=\dfrac{T}{A}$, the unit of stress is the \textbf{pascal} equal to 1Nm$^{-2}$.
    \item \textbf{Tensile strain} $\epsilon=\dfrac{\Delta L}{L}$, tensile strain is a ratio and \textbf{has no unit}.
\end{itemize}

In a graph of tensile stress against tensile strain.

\begin{enumerate}
    \item From 0 to the \textbf{limit of proportionality}, the tensile stress is proportional to the tensile strain.
    \item Beyond that the line curves and continues beyond the \textbf{elastic limit}.
        \begin{itemize}
            \item Beyond which the wire is permanently stretched and suffers \textbf{plastic deformation}.
        \end{itemize}
    \item And to the \textbf{yield point}, where the wire weakens temporarily.
    \item A small increase in the tensile stress causes a large increase in tensile strain as the material of the wire undergoes \textbf{plastic flow}.
    \item Beyond the \textbf{ultimate tensile stress}, the wire loses its strength, extends and \textbf{becomes narrower} at its weakest point.
    \item Increase of tensile stress occurs due to the reduced area of cross section at this point until the wire breaks.
\end{enumerate}

The \textbf{ultimate tensile stress} is the maximum tensile stress, sometimes called the breaking stress. The \textbf{strength} of a material is its ultimate tensile stress.

The \textbf{stiffness} of different materials can be compared using the gradient of the stress-strain line - it is equal to the \textbf{Young's modulus} of the material.
\begin{itemize}
    \item \textbf{Brittle material} snaps without noticeable yield.
    \item \textbf{Ductile material} can be drawn into a wire.
\end{itemize}
