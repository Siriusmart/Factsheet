\subsection{Springs}

The \textbf{tension} is equal and opposite to the force needed to stretch the string.

\textbf{Hooke's law} states that the force needed to stretch a string is directly proportional to the extension of the string from its natural length.
$$F=k\Delta L$$
\begin{itemize}
    \item $k$ is the \textbf{string constant}, the greater the value of $k$ the stiffer the string. The unit of $k$ is Nm$^{-1}$.
        \begin{itemize}
            \item The graph of $F$ against $\Delta L$ is a straight line of gradient $k$ through the origin.
        \end{itemize}
    \item $\Delta L$ is the \textbf{extension from its natural length} $L$.
\end{itemize}

If a string is stretched beyond its \textbf{elastic limit}, it \textbf{does not regain its original length} when the force applied to it is removed.

\begin{itemize}
    \item Where \textbf{springs in parallel}, the \textbf{effective spring constant} is
        $$k=k_P+k_Q$$
    \item Where \textbf{spring in series}
        $$\frac{1}{k}=\frac{1}{k_P}+\frac{1}{k_Q}$$
\end{itemize}

\subsubsection*{Elastic Potential Energy}

Elastic potential energy is energy \textbf{stored in a stretched spring}, if the spring is released, the elastic energy will be \textbf{transferred into kinetic energy} of the spring. The work done to stretch a spring by extension $\Delta L$ from its unstretched length is $\dfrac{1}{2}F\Delta L$.

So the elastic potential energy stored in a stretched spring is
$$E_p=\frac{1}{2}F\Delta L=\frac{1}{2}k\Delta L^2$$
