\subsection{More about Stress and Strain}

\begin{itemize}
    \item For a \textbf{metal wire}, the \underline{loading/unloading curves are the same} provided the elastic limit is not exceeded.
        \begin{itemize}
            \item Beyond the elastic limit, the unloading curve is \textbf{parallel to the loading curve} as the wire has a \textbf{permanent extension}.
        \end{itemize}
    \item For a \textbf{rubber band}, the unloading curve is \textbf{below the loading curve} except at zero and maximum extensions.
        \begin{itemize}
            \item The rubber band \textbf{remains elastic} so regains its initial length.
            \item But has a low \textbf{limit of proportionality}.
        \end{itemize}
    \item For a \textbf{polythene strip}, the strip does not return to its initial length - it has a \textbf{low limit of proportionality} and suffers \textbf{plastic deformation}.
\end{itemize}

\textbf{Strain energy} is the work done to deform an object. The work done is represented by the area under the loading curve.
