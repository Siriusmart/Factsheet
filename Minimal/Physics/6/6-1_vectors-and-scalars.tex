\subsection{Vectors and Scalars}

\begin{itemize}
    \item A \textbf{vector} is any physical quantity that has a direction and a magnitude.
    \item A \textbf{scalar} is any physical quantity that is \underline{not directional}.
\end{itemize}

A vector can be \textbf{represented as an arrow}. The \textbf{length} represents the magnitude and the \textbf{direction} gives the direction of the vector quantity.
\begin{itemize}
    \item \textbf{Displacement} is distance in a given direction.
    \item \textbf{Velocity} is speed in a given direction.
    \item \textbf{Force and acceleration} are also vector quantities.
\end{itemize}

\subsubsection*{Vector Addition}
$$OB=OA+AB$$
Two vectors can be added together using a \textbf{scale diagram}.
\begin{itemize}
    \item \textbf{Resultant} is the combined effects of two forces.
\end{itemize}

For two \textbf{perpendicular forces} $F_1$ and $F_2$
\begin{itemize}
    \item The magnitude of resultant $F=\sqrt{{F_1}^2+{F_2}^2}$
    \item The angle between the resultant and $F_1$ is given by $\tan\theta=F_2/F_1$
\end{itemize}

A force $F$ can be resolved into \textbf{two perpendicular components}
\begin{itemize}
    \item $F\cos\theta$ parallel to a line at angle $\theta$ to the line of action of the force.
    \item $F\sin\theta$ perpendicular to the line.
\end{itemize}
