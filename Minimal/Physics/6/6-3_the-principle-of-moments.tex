\subsection{The Principle of Moments}

The \textbf{moment of a force} about any point is defined as force $\times$ perpendicular distance from the line of the force to the point.
$$\text{moment of the force}=F\times d$$

The unit of the moment of a force is the \textbf{newton metre} (Nm).

\begin{itemize}
    \item A \textbf{body} is an object that is \underline{not a point object}.
    \item A body turns if a force is applied to it anywhere other than through its \textbf{centre of mass}.
\end{itemize}

The \textbf{principle of moments} states if a body is acted on by \underline{more than one force} and it is in equilibrium, the turning effects of the forces must balance out.

If a body is in equilibrium
$$\text{sum of clockwise moments}=\text{sum of anticlockwise moments}$$

The \textbf{centre of mass} of a body is the point through which a single force on the body has \underline{no turning effect} - it is the point where we consider the \textbf{weight of the body} to act.
