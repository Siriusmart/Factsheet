\subsection{Nuclear Radius}

We can estimate the diameter of different nuclides using \textbf{high-energy electrons}.
\begin{itemize}
    \item When a \textbf{beam of high-energy electron} is directed at a thin solid sample of an element, the incident electrons are \textbf{diffracted by the nuclei} of the atoms in the foil.
    \item The electrons are diffracted because the de Broglie wavelength of such high-energy electrons is about the same as the diameter of the nucleus.
    \item A detector is used to measure the number of electrons per second diffracted through different angles.
\end{itemize}

Result shows
\begin{itemize}
    \item \textbf{Scattering} of the beam electrons by the nuclei occurs due to their charge, this causes the intensity to decrease as angle $\theta$ increases.
    \item \textbf{Diffraction} of the beam electrons by each nucleus causes intensity maxima and minima to be superimposed on the scattering patter.

        The angle of the first minimum $\theta_\text{min}$ is measured and used to calculate the diameter of the nucleus. Provided the wavelength of the incident electrons is known.
\end{itemize}

\subsubsection*{Nucleon Radius and Nucleon Number}

It can be shown that $R$ depends on mass number $A$ according to
$$R=r_0\sqrt[3]{A}$$
where $r_0=1.05$fm.

The graph of $\ln r$ against $\ln A$ gives
\begin{itemize}
    \item A straight line of gradient $1/3$
    \item With y-intercept $\ln r_0$.
\end{itemize}

\subsubsection*{Nuclear Density}

Assuming the nucleus is spherical.
\begin{align*}
    \text{Volume}\ V&=\frac{4}{3}\pi R^3\\
                    &=\frac{4}{3}\pi(r_0\sqrt[3]{A})^3\\
                    &=\frac{4}{3}\pi {r_0}^3A
\end{align*}

This means the \textbf{nuclear volume} $V$ is proportional to the mass of the nucleus.
\begin{itemize}
    \item The density of the nucleus is constant, independent, and same throughout a nucleus.
    \item We can conclude that nucleons are \textbf{separated by the same distance} regardless of the size of the nucleus, and therefore evenly separated inside the nucleus.
\end{itemize}
$$\rho_\text{nuc}=\frac{Au}{\frac{4}{3}\pi {r_0}^3A}$$
