\subsection{More about $\alpha$, $\beta$ and $\gamma$ Radiation}

\begin{itemize}
    \item $\alpha$ radiation consists of \textbf{positively charge particles} - each $\alpha$ particle is composed of two protons and two neutrons.
    \item $\beta$ radiation from naturally occurring radioactive substances consists of \textbf{fast-moving electrons}.

        This is proved by measuring the \textbf{deflection of a beam of $\beta$ particles}, which is used to work out the \textbf{specific charge} of the particles, to be the same as the electron.

        A nucleus with too many proton emits a \textbf{positron}.
    \item $\gamma$ radiation consists of photon.

        A $\gamma$ photon is emitted if a nucleus has excess energy after it has emitted an $\alpha$ or $\beta$ particle.
\end{itemize}

\subsubsection*{Inverse Square Law for $\gamma$ Radiation}

The \textbf{intensity} of radiation is the \textbf{radiation energy per second} passing normally through unit area.
\begin{itemize}
    \item For a point source that emits $n$ $\gamma$ photons per second, each energy $hf$, radiation energy per second from the source is $nhf$.
    \item At distance $r$ from the source, all photons pass through a total area of $4\pi r^2$.
        $$I=\frac{nhf}{4\pi r^2}$$
\end{itemize}

\subsubsection*{Equations for Radioactive Change}
\begin{itemize}
    \item $\alpha$ emission: $^A_ZX\to\,^4_2\alpha+\,^{A-4}_{Z-2}Y$
    \item $\beta^-$ emission: $^A_ZX\to\,^{\hspace{6pt}A}_{Z+1}Y+\,^{\hspace{6pt}0}_{-1}\beta+\bar{\nu_e}$
    \item $\beta^+$ emission: $^A_ZX\to\,^{\hspace{6pt}A}_{Z-1}Y+\,^{\hspace{6pt}0}_{+1}\beta+\nu_e$
    \item Electron capture: $^A_ZX+\,^{\hspace{6pt}0}_{-1}e\to\,^{\hspace{6pt}A}_{Z-1}Y+\nu_e$
\end{itemize}
