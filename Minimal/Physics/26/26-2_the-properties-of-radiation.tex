\subsection{The Properties of $\alpha$, $\beta$ and $\gamma$ Radiation}

Rutherford found that radiation emitted
\begin{itemize}
    \item \textbf{Ionises air}, making it conduct electricity.
    \item Is of three types, $\alpha$, $\beta$ and $\gamma$.
    \item A \textbf{magnetic field} deflects $\alpha$ and $\beta$ radiation in opposite directions, and has no effect on $\gamma$ radiation.
\end{itemize}

\subsubsection*{Ionisation Experiments}

The ionising effect of each type of radiation can be investigated using an \textbf{ionisation chamber} and a \textbf{picoammeter}, containing air at atmospheric pressure.

Ions created in the chamber are attracted to oppositely charged electrodes where they are discharged, causing a current proportional to the number of ions per second created.
\begin{itemize}
    \item $\alpha$ radiation causes \textbf{strong ionisation}.

        Ionisation ceases beyond a certain distance, because $\alpha$ radiation has a range in air no more than a few centimetres.
    \item $\beta$ radiation is \textbf{muck weakly ionising}, and range in air up to a metre.

        A $\beta$ particle therefore produces \textbf{less ions per millimetre} along its path than an $\alpha$ particle does.
    \item $\gamma$ radiation has a \textbf{much weaker ionising effect} than $\alpha$ or $\beta$ radiation. This is because photons carry no charge, so have less effect.
\end{itemize}

\subsubsection*{Cloud Chamber Observations}

A cloud chamber contains air \textbf{saturated with vapour} at a very low temperature.

An $\alpha$ or $\beta$ particle passing through the cloud chamber leaves a \textbf{visible track of condensed vapour droplets}.
\begin{itemize}
    \item $\alpha$ particles create straight tracks and are easily visible.

        The tracks from a given isotope are \textbf{all of the same length}, indicating they have the same range.
    \item $\beta$ particles produced \textbf{wispy tracks} and are \textbf{easily deflected} as a result of collisions with air molecules.

        The tracks are not as easy to see as $\alpha$ particle tracks.
\end{itemize}

\subsubsection*{Absorption Tests}

A \textbf{Geiger tube and a counter} is used to investigate absorption by different materials. Each particle of radiation that enters the tube is registered by the counter as a single count.

The \textbf{count rate} is the number of counters divided by time taken.
\begin{enumerate}
    \item The count rate due to background radioactivity is measured.
    \item The count rate is then measured with the source at a \textbf{fixed distance} from the tube, \textbf{without any absorber} present.

        The \textbf{corrected count rate} is the background count rate subtracted from the count rate with source present.
    \item The count rate is then measured with the \textbf{absorber present} between the source and the tube.
    \item The corrected count rate with and without the absorber present can then be compared.
\end{enumerate}

\subsubsection*{The Geiger Tube}

The Geiger tube is a \textbf{sealed metal tube}, containing
\begin{itemize}
    \item \textbf{Argon gas} at low pressure.
    \item A tiny mica window at the end to allow $\alpha$ and $\beta$ particles in.
    \item A \textbf{metal rod} with positive potential down the middle of the tube.
    \item \textbf{Tube walls} are connected to the negative terminal of the power supply.
\end{itemize}

When a particle of ionising radiation enters the tube,
\begin{itemize}
    \item The particle ionises the gas atoms along its track.
    \item \textbf{Negative ions} are attracted to the tube, and \textbf{positive ions} to the wall.
    \item The ions accelerate and collide with other gas atoms, \textbf{producing more ions}.
    \item Causing many ions to \textbf{discharge at the electrodes}.
    \item A \textbf{pulse of charge} passes round the circuit through resistor $R$, causing a \textbf{voltage pulse} which is recorded as a single count by the \textbf{pulse counter}.
\end{itemize}
