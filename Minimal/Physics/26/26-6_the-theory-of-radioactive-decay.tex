\subsection{The Theory of Radioactive Decay}

The decay of an unstable nucleus is an \textbf{unpredictable event}, every nucleus of a radioactive isotope has an \textbf{equal probability of undergoing radioactive decay} in any given time interval.

For a large number of nuclei of a radioactive isotope, the number of nuclei that disintegrate in a certain time interval depends only on the total number of nuclei present.

The \textbf{rate of disintegration}
$$\frac{dN}{dt}=-\lambda N$$
where $\lambda$ is the \textbf{decay constant}.

So the activity $A$ of $N$ atoms of a radioactive isotope is given by
$$A=N\lambda$$

From the rate of disintegration, we have
\begin{align*}
    N&=N_0e^{-\lambda t}\\
    m&=m_0e^{-\lambda t}\\
    A&=A_0e^{-\lambda t}
\end{align*}

The \textbf{corrected count rate} $C$ due to a sample of radioactive isotope at a fixed distance from a Geiger tube is \textbf{proportional to the activity} of the source.
$$C=C_0e^{-\lambda t}$$
where $C_0$ is the count rate at time $t=0$.

\subsubsection*{The Decay Constant}

The decay constant $\lambda$ is the probability of an individual nucleus decaying per second.
$$\lambda=\frac{\Delta N}{N}/\Delta t$$
And
$$T_{1/2}=\frac{\ln 2}{\lambda}$$
