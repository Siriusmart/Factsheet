\subsection{More about Decay Modes}

\subsubsection*{The N-Z Graph}

The N-Z graph is a graph of the neutron number $N$ against the proton number $Z$ for all known isotopes.
\begin{itemize}
    \item \textbf{Light isotopes} ($Z<20$) follow the straight line $N=Z$, such nuclei have an equal number of protons and neutrons.
    \item For $Z>20$, stable nuclei have \textbf{more neutrons and protons}. The extra neutrons help bind the nucleons together without introducing repulsive electrostatic forces, which adding more protons would do.
    \item $\alpha$ emitters occur beyond $Z=60$, they have more neutrons than protons but they are too large to be stable. The strong nuclear force is \textbf{unable to overcome the electrostatic force of repulsion} between two protons.
    \item $\beta^-$ emitters occur to the left of the stability belt where isotopes are \textbf{neutron-rich} compared to stable isotopes.
    \item $\beta^+$ emitters occur to the right of the stability belt where the isotopes are \textbf{proton-rich} compared to stable isotopes.
\end{itemize}

\subsubsection*{Nuclear Energy Levels}

If the daughter nucleus is formed in an \textbf{excited state} after it emits an $\alpha$ or $\beta$ particle or undergoes electron capture, the nucleus moves to its \textbf{ground state} either directly or via one or more lower-energy excited states.

The short-lived excited state is said to be a \textbf{metastable state}.
