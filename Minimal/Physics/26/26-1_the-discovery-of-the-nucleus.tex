\subsection{The Discovery of the Nucleus}

\subsubsection*{$\alpha$ Particle Scatter Experiment}

It was known that $\alpha$ radiation consists of \textbf{fast-moving positively charged particles}.
\begin{enumerate}
    \item Using a \textbf{narrow beam} of $\alpha$ particles all of the \textbf{same energy}.
    \item A \textbf{thin metal foil} is placed in the path of the beam.
    \item $\alpha$ particles scattered by the metal foil were detected by a detector which could be moved around at a constant distance from the point of impact of the beam on the metal foil.
\end{enumerate}

The result showed that
\begin{itemize}
    \item Most $\alpha$ particles pass straight through the foil with \textbf{little to no deflection}.
    \item About 1 in 2000 were deflected.
    \item 1 in 10000 $\alpha$ particles were deflected through angles of more than 90$^\circ$.
\end{itemize}

It can be concluded that
\begin{itemize}
    \item Most of the atom's mass is \textbf{concentrated in a small region} called the nucleus, in the centre of the atom.
    \item The nucleus is \textbf{positively charged} because it repels $\alpha$ particles that approaches it too closely.
\end{itemize}

\subsubsection*{Estimation of Size of the Nucleus}

For a \textbf{single scattering} by a foil of $n$ layers of atoms, the probability of an $\alpha$ particle being deflected by a given atom is 1 in $10000n$.

For a nucleus of diameter $d$ in an atom of diameter $D$
\begin{align*}
    \frac{\dfrac{1}{4}\pi d^2}{\dfrac{1}{4}\pi D^2}&=\frac{1}{10000n}\\
    d^2&=\frac{D^2}{10000n}
\end{align*}

A typical value of $n=10^{4}$ gives $d=\dfrac{D}{10000}$.
