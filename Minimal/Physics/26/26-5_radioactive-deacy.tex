\subsection{Radioactive Decay}

The \textbf{decay curve} shows the mass of the initial isotope decreases gradually as the number of nuclei of the isotope decreases. The mass \textbf{decreases exponentially}, which means the mass drops by a constant factor in equal intervals of time.

The \textbf{half-life} $T_{1/2}$ of a radioactive isotope is the time taken for the mass of the isotope to decrease to half the initial mass.

The mass decreases exponentially, because radioactive decay is a \textbf{random process} and the number of nuclei that decay in a certain time is in proportion to the number of nuclei remaining.

\subsubsection*{Activity}

The activity $A$ of a radioactive isotope is the number of nuclei of the isotope that disintegrate per second - the rate of change of the number of nuclei of isotope.

The unit of activity is the \textbf{becquerel} (Bq), 1Bq$\,=\,$1 disintegration per second.

\begin{itemize}
    \item The activity of a radioactive isotope is \textbf{proportional to the mass} of the isotope.
    \item The mass of a radioactive isotope \textbf{decreases with time}, so the activity decreases with time.
\end{itemize}

\subsubsection*{Activity and Power}

For a radioactive source of activity $A$ that emit particles of the same energy $E$.
$$P=AE$$
