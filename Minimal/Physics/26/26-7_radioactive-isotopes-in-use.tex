\subsection{Radioactive Isotopes in Use}

The choice of an isotope for a particular purpose depends on
\begin{itemize}
    \item Its \textbf{half-life}.
    \item The \textbf{type of radiation} it emits.
\end{itemize}

\subsubsection*{Carbon Dating}

\begin{enumerate}
    \item Carbon dioxide from the atmosphere is \textbf{taken up by living plants}, a small percentage of the carbon content of any plant is the radioactive carbon-14.
    \item Once a tree has died, no further carbon is taken in, so the proportion of carbon-14 in the dead tree decreases as the carbon-14 nuclei decay.
    \item The age of the tree can be calculated by \textbf{measuring the activity} of the dead sample. Provided the activity of the same mass of living wood is known.
\end{enumerate}

\subsubsection*{Argon Dating}

Ancient rocks contains \textbf{trapped argon gas} as a result of the decay of radioactive isotope of potassium. The age of the rock can be calculated by measuring the proportion of argon-40 to potassium-40.

\subsubsection*{Radioactive Tracers}

A radioactive tracer is used to follow the path of a substance through a system.

The radioactive tracer should
\begin{itemize}
    \item Have a half-life which is stable enough for the necessary measurements to be made, and short enough to decay quickly after use.
    \item Emit $\beta$ or $\gamma$ radiation so it can be detected outside of the flow path.
\end{itemize}
