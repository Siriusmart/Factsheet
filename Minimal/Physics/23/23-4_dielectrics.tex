\subsection{Dielectrics}

Dielectrics are \textbf{electrically insulating materials} that increase the ability of a parallel-plate capacitor to store charge when placed between the plates.

When a dielectric is placed between two oppositely charged parallel plates connected to a battery.
\begin{enumerate}
    \item Dielectric substances made of \textbf{polar molecules} are already polarised, but lie in \textbf{random directions}.
    \item These molecules turn when the dielectric is placed between the charged plates because their electrons are attracted slightly to the positive plates.
    \item Therefore, the surface of the dielectric near the positive plate \textbf{gains negative charge}, the other surface gains positive charge.
    \item As a result, more charge is stored on the plates because

        The positive side of the dielectric \textbf{attracts more electrons} from the battery onto the negative plate.

        Similar reasoning for the positive plate.
\end{enumerate}

The effect of a dielectric is to \textbf{increase the capacitance} of the capacitor.

\subsubsection*{Relative Permittivity}

The ratio of the charge stored with the dielectrics relative to the charge stored without the dielectric is defined as the relative permittivity of the dielectric substance.
\begin{align*}
    \varepsilon_r&=\frac{Q}{Q_0}\\
                 &=\frac{C}{C_0}
\end{align*}

The capacitance for a parallel-plate capacitor
$$C=\frac{A\varepsilon_0\varepsilon_r}{d}$$

A large capacitance can be achieved by
\begin{itemize}
    \item Making area $A$ as large as possible.
    \item Making the plate spacing $d$ as small as possible.
    \item Filling the space between the plate with a dielectric which has a $\varepsilon_r$ as large as possible.
\end{itemize}
