\subsection{Generating Electricity}

\textbf{Electromagnetic induction} occurs whenever a wire \textbf{cuts across magnetic field lines}. If the wire is part of a complete circuit, the \textbf{induced emf} forces electrons round the circuit. The

The induced emf is increased by
\begin{itemize}
    \item Moving the wire \textbf{faster}.
    \item Using a \textbf{strong magnet}.
    \item Making the wire into a \textbf{coil}, and pushing the magnet in or out of the coil.
\end{itemize}

No emf is induced in the wire if the wire is parallel to the magnetic field lines - the wire must cut across the lines of magnetic field.

\subsubsection*{Energy Changes}

\begin{itemize}
    \item When a magnet is moved relative to a conductor, an \textbf{emf is induced} in the conductor.
    \item If the conductor is part of a circuit, a \textbf{current passes} round the circuit.
    \item The electric current \textbf{transfer energy} from the source of the emf in a circuit to other components in the circuit - work must be done to keep the magnet moving.
\end{itemize}
$$\text{Rate of energy transfer}=I\varepsilon$$

\textbf{Fleming's right-hand rule} is used to work out the direction of the induced current.
