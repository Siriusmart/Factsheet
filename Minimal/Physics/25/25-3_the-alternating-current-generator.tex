\subsection{The Alternating Current Generator}

The simple AC generator consists of a \textbf{rectangular coil} that spins in a uniform magnetic field.
\begin{itemize}
    \item When the coil spins at a steady rate, the flux linkage changes continuously.
    \item At an instant where the normal to the plane of the coil is at an angle $\theta$ to the field lines, the flux linkage through the coil is
        $$\Phi=BAN\cos\theta$$
    \item For a coil spinning at steady frequency, $\theta=2\pi ft$.
        \begin{align*}
            \Phi&=BAN\cos 2\pi ft\\
            \varepsilon=\frac{d\Phi}{dt}&=-2\pi fBAN\sin 2\pi ft\\
                                        &=\varepsilon_0\sin 2\pi ft\\
                                        &=\varepsilon_0\sin\omega t
        \end{align*}
\end{itemize}

The peak emf can be increased by increasing the \textbf{speed}, or using a \textbf{stronger magnet}, a \textbf{bigger coil}, or a \textbf{coil with more turns}.

\subsubsection*{Back Emf}

An emf is induced in the spinning coil of an electric motor because the flux linkage through the coil changes.

The induced emf is referred to as a back emf because it \textbf{acts against the pd} applied to the motor in accordance with Lenz's law.
$$V-\varepsilon=IR$$
