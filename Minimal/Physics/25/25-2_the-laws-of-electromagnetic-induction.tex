\subsection{The Laws of Electromagnetic Induction}

A magnetic field is produced in and around a coil when it is connected to a battery and a current is passed through it.

\begin{itemize}
    \item For a \textbf{solenoid}, the pattern of the magnetic field lines is like the pattern for a bar magnet.
    \item The field lines pass through the solenoid and loop round outside the solenoid from one end to another.
\end{itemize}

\subsubsection*{Lenz's Law}

Lenz's law states that the direction of the induced current is always such as to opposite the change that causes the current.

The explanation of Lenz's law is that energy is never created or destroyed - the induced current could never be in a direction to help the change that causes it, as it would produce electrical energy from nowhere.

\subsubsection*{Faraday's Law of Electromagnetic Induction}

\begin{align*}
    \text{Magnetic flux}\ \phi&=BA\\
    \text{Magnetic flux linkage}\ \Phi&=N\phi
\end{align*}
where $B$ is the magnetic flux density perpendicular to area $A$. When the magnetic field is at angle $\theta$ to the normal at the coil face, the flux linkage through the coil is $N\phi=BAN\cos\theta$.

The unit of magnetic flux is the \textbf{weber}, equal to 1TM$^2$.

\textbf{Faraday's law of electromagnetic induction} states that the induced emf in a circuit is equal to the rate of change of flux linkage through the circuit.
$$\varepsilon=-\frac{d\Phi}{dt}$$
The minus sign represents the fact that the induced emf acts in a way such that the direction is opposite to the change that caused it (Lenz's law).

For a moving conductor at right angles to the field lines
$$\varepsilon=Blv$$
