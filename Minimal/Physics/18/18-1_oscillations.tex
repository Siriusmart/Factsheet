\subsection{Oscillations}

\begin{itemize}
    \item The \textbf{equilibrium} position is where an oscillation will eventually come to a standstill.
    \item An \textbf{oscillating object} moves repeatedly one way then in the opposite direction through its equilibrium position.
    \item The \textbf{displacement} of the object is the distance and direction from equilibrium.
    \item The \textbf{amplitude} of oscillations is the maximum displacement of the oscillating object from equilibrium.
    \item \textbf{Free vibrations} are oscillations where the amplitude is constant and no frictional forces are present.
    \item The \textbf{time period} is the time for one complete cycle of oscillation.
    \item One \textbf{full cycle} after passing through any position, the object passes through that same position in the same direction.
    \item The \textbf{frequency} of oscillations is the number of cycles per second.
\end{itemize}
The unit for frequency is the \textbf{hertz} - one cycle per second.
\begin{align*}
    \text{Time period}\ T&=\frac{1}{f}\\
    \text{Angular velocity}\ \omega&=\frac{2\pi}{T}
\end{align*}

$$\textbf{Phase difference}=\frac{2\pi\Delta t}{T}$$
where $\Delta t$ is the time between successive instants when the two objects are at maximum displacement in the same direction.
