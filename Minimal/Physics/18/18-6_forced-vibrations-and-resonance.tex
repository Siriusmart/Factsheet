\subsection{Forced Vibrations and Resonance}

\begin{itemize}
    \item A \textbf{periodic force} is a force applied at regular intervals.
    \item The \textbf{natural frequency} is the system's frequency when it \textbf{oscillates without a periodic force}.
    \item A system undergoes \textbf{forced vibrations} when a period force is applied to it.
\end{itemize}

For a mass attached to two stretched springs, with the lower spring attached to a mechanical oscillator connected to a signal generator.
\begin{itemize}
    \item The amplitude of oscillations \textbf{increases until a maximum amplitude} at a particular frequency, then decreases again.
    \item The \textbf{phase difference} between the displacement and the period force increases from 0 to $\dfrac{\pi}{2}$ at maximum amplitude, then from $\dfrac{\pi}{2}$ as frequency increases further.
\end{itemize}

\subsubsection*{Resonance}

When a system is in resonance, the periodic force is \textbf{exactly in phase} with the velocity of the oscillating system. The frequency at the maximum amplitude is the \textbf{resonant frequency}.

The lighter the damping
\begin{itemize}
    \item The larger the maximum amplitude at resonance.
    \item The closer the \textbf{resonant frequency} is to the natural frequency of the system.
\end{itemize}

As the applied frequency increases from the resonant frequency
\begin{itemize}
    \item The \textbf{amplitude decreases}.
    \item The \textbf{phase difference} between the displacement and periodic force increases until they are $\pi$ rad out of phase.
\end{itemize}

For an oscillating system with no damping
$$\text{Resonant frequency}=\text{natural frequency}$$
