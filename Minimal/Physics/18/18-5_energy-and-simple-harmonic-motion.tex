\subsection{Energy of Simple Harmonic Motion}

A freely oscillating object \textbf{oscillates with a constant amplitude} because no friction is acting on it.
\begin{enumerate}
    \item The \textbf{potential energy} changes with displacement $x$ from equilibrium.
        $$E_P=\frac{1}{2}kx^2$$
    \item The \textbf{total energy} of the system is therefore
        $$E_T=\frac{1}{2}kA^2$$
    \item So the \textbf{kinetic energy} of the system is
        $$E_K=\frac{1}{2}k(A^2-x^2)$$
\end{enumerate}

Equating $E_K=\dfrac{1}{2}mv^2$, we also have
$$v\pm\omega\sqrt{A^2-x^2}$$
showing speed is maximum when $x=0$ and $v=\omega A$.

\subsubsection*{Damping}

In any oscillating system where \textbf{friction or air resistance is present}, the amplitude decreases.

\textbf{Dissipative forces} dissipate the energy of the system to the surroundings as thermal energy. The motion is said to be \textbf{damped} if dissipative forces are present.
\begin{itemize}
    \item \textbf{Light damping} - the time period is independent of the amplitude, so each cycle takes the same length of time.
    \item \textbf{Critical damping} - the oscillating system returns to equilibrium in the \textbf{shortest possible time} without overshooting.
    \item \textbf{Heavy damping} - the displaced object returns to equilibrium much more slowly than if the system is critically damped, and \textbf{no oscillating motion} takes place.
\end{itemize}
