\subsection{More about Stationary Waves on Strings}

\begin{itemize}
    \item The \textbf{first harmonic pattern of vibration} is seen at the lowest possible frequency that gives a pattern. $\lambda_1=2L$
    \item The \textbf{second harmonic} has a node in the middle. $\lambda_2=L$ and $f_2=2f_1$
    \item The \textbf{third harmonic} have nodes with distance of $L/3$ from either ends. $\lambda_3=2\lambda_1/3$ and $f_3=3f_1$.
\end{itemize}

The progressive wave \textbf{reverses its phase} when it reflects at the fixed end and travel back along the string.
\begin{itemize}
    \item When it reaches the vibrator, it reflects and reverses phase again.
    \item If this wave is \textbf{reinforced by a wave created by the vibrator}, the amplitude of the wave is increased.
\end{itemize}
The \textbf{key condition} for a stationary wave to form is that the time taken for a wave to travel along the string and back should be \textbf{equal to the time taken for a whole number of cycles} of the vibrator.

The first harmonic frequency is given by
$$f_1=\frac{1}{2L}\sqrt{\frac{T}{\mu}}$$
