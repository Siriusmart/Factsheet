\subsection{Waves and Vibrations}

\begin{itemize}
    \item \textbf{Mechanical waves} are vibrations which pass through a substance.
    \item \textbf{Electromagnetic waves} are oscillating electric and magnetic fields that progress through space \underline{without the need for a substance}.
    \item \textbf{Longitudinal waves} are waves in which the direction of vibration of the particles are parallel to the direction in which the wave travels.
        \begin{itemize}
            \item Sound waves
            \item Primary seismic waves
        \end{itemize}
    \item \textbf{Transverse waves} are waves in which the direction of vibration is perpendicular to the direction in which the wave travels.
        \begin{itemize}
            \item Electromagnetic waves
        \end{itemize}
\end{itemize}

\subsubsection*{Polarisation}

\textbf{Plane-polarised waves} are transverse waves which the vibrations stay in one plane only. If the vibrations change from one plane to another, the waves are \textbf{unpolarised}.

Longitudinal waves \underline{cannot be polarised}.

A \textbf{polaroid filter} only allow through light which vibrate in a certain direction.
\begin{itemize}
    \item If unpolarised light is passed through a polaroid filter, the transmitted light is polarised.
    \item If unpolarised light is passed through \textbf{two polaroid filters}, the intensity of transmitted light changes if one polaroid is turned relative to the other.
        \begin{itemize}
            \item The filters are \textbf{cross} when the transmitted intensity is a minimum.
            \item Polarised light from the first filter cannot pass through the second filter.
        \end{itemize}
\end{itemize}

Light is part of the spectrum of \textbf{electromagnetic waves}, the plane of polarisation is defined as the \underline{plane which the electric field vibrates}.
