\subsection{Wave Properties 2}

The \textbf{principle of superposition} states that when two waves meet, the total displacement at a point is equal to the sum of the individual displacements at that point.

Where a crest meets a trough of the same amplitude, the \textbf{resultant displacement is zero}. If they are not the same amplitude, the resultant is called a \textbf{minimum}.

\subsubsection*{Water Waves in a Ripple Tank}
A \textbf{vibrating dipper} on a water surface sends out circular waves. The waves pass trough each continuously.
\begin{itemize}
    \item \textbf{Points of cancellation} are created where a crest from one dipper meets a trough from the other dipper - these points are seen as \textbf{gaps in the wavefronts}.
    \item \textbf{Points of reinforcement} are created where a crest from one dipper meets a crest from the other dipper.
\end{itemize}

\subsubsection*{Wave Properties with Microwaves}

A microwave transmitter and receiver can be used to demonstrate reflection, refraction, diffraction, interference and polarisation of microwaves.
\begin{itemize}
    \item Place the receiver in the path of the \textbf{microwave beam} and move the receiver gradually away from the transmitter - microwaves \textbf{become weaker as they travel} away from the transmitter.
    \item Place a \textbf{metal plate} between the transmitter and the receiver - microwaves \textbf{cannot pass through metal}.
    \item Make a \textbf{narrow slit} with two metal plates - \textbf{microwaves have been diffracted} as they pass through the slit.
        \begin{itemize}
            \item If the slit is made wider, less diffraction occurs.
        \end{itemize}
    \item Make a \textbf{pair of slits} with a narrow metal plate and two plates, use the receiver to find \textbf{points of cancellation and reinforcement}.
\end{itemize}
