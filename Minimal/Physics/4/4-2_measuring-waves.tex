\subsection{Measuring Waves}
\begin{itemize}
    \item The \textbf{displacement} of a vibrating particle is its distance and direction from its \textbf{equilibrium position}.
    \item The \textbf{amplitude} of a wave its the \textbf{maximum displacement} of a particle.
    \item The \textbf{wavelength} of a wave is the least distance between two adjacent vibrating particles with the same displacement and velocity at the same time.
    \item One complete \textbf{cycle} of a wave is from maximum displacement to the next maximum displacement.
    \item The \textbf{period} of a wave is the time for one complete wave to pass a fixed point.
    \item The \textbf{frequency} of a wave is the number of cycles of vibration of a particle per second. Or the number of complete waves passing a point per second.
\end{itemize}
\begin{align*}
    \text{Time period}\ T&=\frac{1}{f}\\
    \text{Wave speed}\ c&=f\lambda
\end{align*}

\begin{itemize}
    \item The \textbf{phase} of a vibrating particle at a certain time is the fraction of a cycle it  has completed since the start of the cycle.
    \item The \textbf{phase difference} between two particles vibrating at the same frequency is the fraction of a cycle between the vibrations of the two particles.
\end{itemize}
$$\text{phase difference}=\frac{2\pi d}{\lambda}$$
