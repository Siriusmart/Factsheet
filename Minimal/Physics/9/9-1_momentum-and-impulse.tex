\subsection{Momentum and Impulse}

The \underline{effect of a force} on an object (change in velocity) depends on its
\begin{itemize}
    \item \textbf{Mass}, and
    \item The \textbf{amount of force}
\end{itemize}
The \textbf{momentum} of an object is defined as its $\text{mass}\times\text{velocity}$.

\begin{itemize}
    \item \textbf{Newton's first law} tells us that a force is needed to change the momentum of an object.
        \begin{itemize}
            \item If a moving object \textbf{gains or loses mass}, the velocity would change to keep its momentum constant.
        \end{itemize}
    \item \textbf{Newton's second law} can be status as
        \begin{align*}
            F\propto\frac{\text{change in momentum}}{\text{time taken}}&=\frac{mv-mu}{t}\\
                                                                       &=\frac{m(v-u)}{t}\\
                                                                       &=ma
        \end{align*}
\end{itemize}

The \textbf{impulse} of a force is defined as the $\text{force}\times\text{time}$ which the force acts.
$$I=F\Delta t=\Delta(mv)$$

\subsubsection*{Force-time Graphs}

The area under the line of a force-time graph represents the \textbf{change of momentum} or the impulse of the force.

The unit of impulse is therefore the \textbf{newton second}.
