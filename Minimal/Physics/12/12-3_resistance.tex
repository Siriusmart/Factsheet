\subsection{Resistance}

The resistance of a component is a measure of the \textbf{difficulty of making current pass through} the component.

The resistance of any component is defined as
$$R=\frac{V}{I}$$
The unit of resistance is the \textbf{ohm} ($\Omega$), equal to 1 volt per ampere.

\subsubsection*{Measuring Resistance}

\begin{enumerate}
    \item An \textbf{ammeter} measures the current through the resistor.
    \item A \textbf{voltmeter} measures the pd across the resistor.
    \item A \textbf{variable resistor} to adjust the current and pd as necessary.
\end{enumerate}

The graph for a resistor is a straight line through the origin, showing the resistance is the same, regardless of the current.

\textbf{Ohm's law} states that the pd across a metallic conductor is proportional to the current through it, provided its physical conditions do not change.

\subsubsection*{Resistivity}

Since resistance $R$
\begin{align*}
    R&\propto L\\
    R&\propto\frac{1}{A}
\end{align*}

Therefore
$$R=\frac{\rho L}{A}$$
where $\rho$ is a constant for that material, known as its resistivity.
$$\rho=\frac{RA}{L}$$

The unit for resistivity is the \textbf{ohm metre} ($\Omega$m).

\subsubsection*{Measuring Resistivity}

\begin{enumerate}
    \item \textbf{Measure the diameter} of the wire.
    \item \textbf{Calculate its cross-sectional area}.
    \item \textbf{Measure the resistance} $R$ for different lengths $L$ of wire.
    \item \textbf{Plot a graph} of $R$ against $L$.
\end{enumerate}

The resistivity of the wire is given by $\textbf{graph gradient}\times A$.

\subsubsection*{Superconductivity}

A \textbf{superconductor} is a material that has zero resistivity below a \textbf{critical temperature} depending on the material - this property of the material is called \textbf{superconductivity}.
\begin{itemize}
    \item The wire has \textbf{zero resistance} below its critical temperature.
    \item When a current passes through it, there is \textbf{no pd} as its resistance is zero, so current has \textbf{no heating effect}.
\end{itemize}

A superconductor loses its superconductivity if its temperature is raised above its critical temperature.
