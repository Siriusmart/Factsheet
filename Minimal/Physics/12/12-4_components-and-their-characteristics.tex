\subsection{Components and Their Characteristics}

Each type of component has its own symbol used to represent the component in a \textbf{circuit diagram}.

\begin{itemize}
    \item A \textbf{cell} is a source of electrical energy.
    \item A \textbf{battery} is combination of cells.
    \item An \textbf{indicator} is a light source.
    \item A \textbf{diode} allows current in one direction only.
    \item A \textbf{light-emitting diode} emits light when it conducts.
    \item A \textbf{resistor} is a component designed to have a certain resistance.
    \item A \textbf{light-depending resistor} has a resistance that decreases with light intensity.
\end{itemize}

\subsubsection*{Component Characteristics}

To measure the variation of a current with pd for a component
\begin{itemize}
    \item A \textbf{potential divide} varies the pd from zero.
    \item A \textbf{variable resistor} varies the current to a minimum.
\end{itemize}

Using a potential divider the current and pd across the component can be \textbf{reduced to zero}, this is not possible with a variable resistor circuit.

Measurements are plotted as a graph of \textbf{current against pd}.

\begin{itemize}
    \item A \textbf{wire} gives a straight line through the origin - its resistance does not change when the current changes.
    \item A \textbf{filament bulb} gives a curve with \textbf{descending gradient} - resistance increases as it becomes hotter.
    \item A \textbf{thermistor at constant temperature} gives a straight line, the higher the temperature the greater the gradient.
    \item A \textbf{diode} conducts easily above a pd of 0.6V, but hardly at all below 0.6V or in the \textbf{opposite direction}.
\end{itemize}

\subsubsection*{The Temperature Coefficient}

\begin{itemize}
    \item A metal has a \textbf{positive coefficient} - its resistance increases with increase of temperature.
    \item An \textbf{intrinsic semiconductor} has a \textbf{negative temperature coefficient}.
\end{itemize}
