\subsection{Current and Charge}

To make an electric current pass round a circuit
\begin{itemize}
    \item The circuit must be \textbf{complete}, and
    \item There must be a \textbf{source of potential difference}.
\end{itemize}

\textbf{Electric current} is the rate of flow of charge in the wire or component. Current flow due to the passage of \textbf{charge carries}.
\begin{itemize}
    \item \textbf{Conduction electrons} for metals.
    \item \textbf{Ions} in a salt solution.
\end{itemize}

Current flows from positive to negative.

\begin{itemize}
    \item The unit of current is the \textbf{ampere}.
    \item The unit of charge is the \textbf{coulomb}.
\end{itemize}

For a current $I$, the \textbf{charge flow} $\Delta Q$ in time $\Delta t$
$$\Delta Q=I\Delta t$$

\subsubsection*{Charge Carriers}

\begin{itemize}
    \item \textbf{Insulator} - each electron is attached to an atom and \textbf{cannot move away} from the atom. When a voltage is applied across an isolator, no current pass through.
    \item \textbf{Metallic conductor} - some electrons are \textbf{delocalised}, they are the charge carriers of the metal. When a voltage is applied across the metal, conduction electrons are attracted towards the positive terminal.
    \item \textbf{Semiconductor} - the number of charge carriers \textbf{increases with temperature}. The resistance of a semiconductor decreases as its temperature is raised. 
\end{itemize}
