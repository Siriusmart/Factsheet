\subsection{Moving Charges in a Magnetic Field}

For an electric beam in a magnetic field,
\begin{itemize}
    \item The beam \textbf{follows a circular path} because the direction of the force on each electron is perpendicular to the direction of motion of the electrons.
    \item A current-carrying wire in a magnetic field experiences a force because the electrons moving along the wire are pushed by the force of the field.
\end{itemize}

Magnetic fields are used in particle physics detectors to \textbf{separate different charged particles out}, and \textbf{measure their momentum} from the curvature of the tracks they create.

\subsubsection*{Force on a Moving Charge}

For a particle of charge $Q$ moving through a uniform magnetic field at speed $v$ in a \textbf{perpendicular direction} to the field, the force on the particle
$$F=BQv$$

If the direction of motion of a charged particle is at angle $\theta$ to the field lines, then
$$F=BQv\sin\theta$$
