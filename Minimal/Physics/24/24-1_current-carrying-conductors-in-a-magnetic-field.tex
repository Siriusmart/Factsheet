\subsection{Current-carrying Conductors in a Magnetic Field}

A \textbf{magnetic field} is a force field surrounding a \textbf{magnet} or \textbf{current-carrying wire} which acts on any other magnet or current-carrying wire placed in the field.
\begin{itemize}
    \item The magnetic field of a bar magnet is \textbf{strongest at its ends} - referred to as north-seeking and south-seeking \textbf{poles}.
    \item A \textbf{magnetic field line} is a lien along which a north pole would move in the field.
\end{itemize}

\subsubsection*{The Motor Effect}

A current-carrying wire placed at a non-zero angle to the field lines of a magnetic field experiences a force due to the field.
\begin{itemize}
    \item When a current flows, the section of the wire in the magnetic field experiences a force that \textbf{pushes it out of the field}.
    \item The magnitude of the force depends on the \textbf{current}, \textbf{magnetic field strength}, \textbf{length of the wire}, and the \textbf{angle between} the field lines and the current direction.

        The force is greatest when the wire is at right angles to the magnetic field.

        The force is zero when the fire is parallel to the magnetic field.
\end{itemize}

Tests shows that the force $F$ on the wire is proportional to the current $I$, and the length $l$ of the wire.

The \textbf{magnetic field strength} is defined as the force per unit length per unit current on a current-carrying conductor.

For a wire carrying current in a uniform magnetic field at 90$^\circ$ to the field lines
$$F=BIl$$

The unit of $B$ is the \textbf{tesla}, equal to 1Nm$^{-1}$A$^{-1}$.

For a straight wire at angle $\theta$ to the magnetic field lines
$$F=BIl\sin\theta$$

\subsubsection*{Couples}

Consider a current-carrying coil with $n$ turns, and can rotate about a vertical axis.
\begin{enumerate}
    \item The long sides of the coil are vertical, each long side experiences a force
        $F=(BIl)n$ in opposite directions at right angles to the field lines.
    \item The \textbf{torque of the couple} $=Fd$ where $d$ is the perpendicular distance between the line of action of the forces on each side.
    \item If the plane of the coil is at angle $\alpha$ to the field lines, then $d=w\cos\alpha$ where $w$ is the width of the coil.

        Therefore, $\tau=Fw\cos\alpha=BIlnw\cos\alpha=BIAn\cos\alpha$
\end{enumerate}
