\subsection{Work and Energy}

Energy is needed to
\begin{itemize}
    \item Make stationary objects \textbf{move}.
    \item Change their \textbf{velocity}.
    \item To \textbf{warm them up}.
\end{itemize}

Objects can possess energy in
\begin{itemize}
    \item Gravitational potential stores.
    \item Kinetic stores.
    \item Thermal stores.
    \item Elastic stores.
\end{itemize}

Energy can be transferred
\begin{itemize}
    \item By radiation.
    \item Electrically.
    \item Mechanically.
\end{itemize}

Energy is measured in \textbf{joules}, one joule is equal to the energy needed to raise a 1N weight through a height of 1m.

The \textbf{principle of conservation of energy} - energy cannot be created or destroyed.

\subsubsection*{Work Done}

Work is done on an object when a force acting on it makes it moves - energy is transferred to the object.
$$\text{Work done}=\text{force}\times\text{distance moved in the direction of the force}$$

\subsubsection*{Force-distance Graphs}

The area under the line of a force-distance graph represents the total work done.
