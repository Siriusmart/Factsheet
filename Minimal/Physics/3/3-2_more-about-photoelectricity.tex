\subsection{More about Photoelectricity}

The energy of each vibrating atom is \textbf{quantised} - only certain energy levels that are multiples of a basic amount are allowed.
\begin{itemize}
    \item The \textbf{work function} is the minimum energy needed for a conduction electron to escape from the metal surface when the metal is at \underline{zero potential}.
    \item When an electron absorbs a photon, its kinetic energy increases by an amount equal to the energy of the photon.
    \item If the energy of the photon \underline{exceeds the work function} of the metal, the conduction electron can leave the metal.
        \begin{itemize}
            \item If the electron does not leave the metal, it collides repeatedly with other electrons and ions, quickly loses its extra kinetic energy.
        \end{itemize}
\end{itemize}

\subsubsection*{The Vacuum Photocell}

A vacuum photocell is a \textbf{glass tube} that contains
\begin{itemize}
    \item A metal plate - the \textbf{photocathode}
    \item A smaller metal electrode - the \textbf{anode}
\end{itemize}
When light of a frequency greater than the threshold frequency for the metal is directed at the photocathode, \underline{electrons are emitted from the cathode} and are attracted to the anode.

The \textbf{photoelectric current} is proportional to the number of electrons per second transferred from the cathode to the anode.
\begin{itemize}
    \item The photoelcetric current is \textbf{proportional to the intensity} of incident light on the cathode.
    \item The intensity of incident light does not affect the maximum kinetic energy of a photoelectron - the energy gained by a photoelectron is due to the \underline{absorption of one photon only}.
\end{itemize}
