\subsection{Collisions of Electrons with Atoms}
An \textbf{ion} is a charged atom, the number of electrons in an ion is not equal to the number of protons, it is created by adding or removing electrons from an atom.

\textbf{Ionisation is any process of creating ions}
\begin{itemize}
    \item $\alpha$, $\beta$ and $\gamma$ radiation create ions when they collide with atoms.
    \item Electrons passing through a \textbf{fluorescent tube} create ions when they collide with atoms of the vapour in the tube.
\end{itemize}

The \textbf{electron volt} (eV) is a unit of energy equal to the work done when an electron is moved through a pd of 1V.

\subsubsection*{Excitation by Collision}

Atoms can absorb energy without being ionised - \textbf{excitation} happens at certain energy values known as the atom's \textbf{excitation energies}, which are characteristics of the atom.

The excitation energies of atoms in a \textbf{gas-filled tube} can be determined by
\begin{enumerate}
    \item Increasing the potential difference between the filament and anode.
    \item Measure the pd when the anode current falls.
\end{enumerate}

Colliding electron makes an electron inside the atom move from an inner shell to an outer shell. The excitation energy is always \textbf{less than the ionisation energy} of the atom because the atomic electron is not removed completely.
