\subsection{The Photoelectric Effect}

Electrons are emitted from the surface of a metal when electromagnetic radiation \underline{above a certain frequency} was directed at the metal. This is know as the \textbf{photoelectric effect}.

Observations could not be explained using the \textbf{wave model of light}.
\begin{itemize}
    \item Photoelectric emission of electrons does not take place if the frequency of the incident EM radiation is below the \textbf{threshold frequency}.
    \item The number of electrons emitted per second is \textbf{proportional to the intensity} of the incident radiation.
    \item Photoelectric emission \textbf{occurs without delay}.
\end{itemize}
According to wave theory, each electron at the surface of a metal should gain some energy from the incoming waves - it cannot explain:
\begin{itemize}
    \item The existence of a threshold frequency.
    \item Why photoelectric emission occurs without delay.
\end{itemize}

\subsubsection*{The Photon Theory}

Light is composed of \textbf{wavepackets} or \textbf{photons}.
$$\text{Energy of a photon}=hf=\frac{hc}{\lambda}$$

To explain the photoelectric effect
\begin{enumerate}
    \item An electron at the surface \textbf{absorbs a single photon} and therefore gains energy equal to $hf$.
    \item An electron can leave the metal surface if the energy gained from a single photon exceeds the \text{work function} $\phi$ of the metal.
    \item Excess energy gained by the photoelectron becomes its \textbf{kinetic energy}.
\end{enumerate}
$$E_\text{Kmax}=hf-\phi$$
Emission can take place from a metal surface provided $E_\text{Kmax}>0$, so the \textbf{threshold frequency} of the metal is
$$f_\text{min}=\frac{\phi}{h}$$

Electrons that escape from the plate can be attracted back by giving the plate a \textbf{sufficient positive charge}. The minimum potential needed to stop photoelectric emission is called the \textbf{stopping potential}.
