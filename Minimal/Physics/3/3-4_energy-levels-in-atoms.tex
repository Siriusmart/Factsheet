\subsection{Energy Levels in Atoms}

Atomic electrons move about the nucleus in \textbf{shells} surrounding the nucleus.
\begin{itemize}
    \item The energy of an electron in a shell is \textbf{constant}.
    \item An electron in a shell near the nucleus has less energy one further away.
    \item Each shell can only hold a certain number of electrons.
\end{itemize}

The \textbf{ground state} is the lowest energy state of an atom. When an atom absorbs energy, an electron moves to a shell of higher energy, the atom is now in an \textbf{excited state}.

The \textbf{electron configuration} in an excited atom is unstable because there is a vacancy in an inner shell. The vacancy is filled by an electron from an outer shell transferring to it. The electron emits a photon in the process of \textbf{de-excitation}.
$$\text{Energy of emitted photon}=E_1-E_2$$

An electron in an atom can \textbf{absorb a photon} and move to an outer shell only if the energy of the photon \underline{exactly equal to the gain of the electron's energy}.

\subsubsection*{Fluorescence}

An excited atom can \underline{de-excite directly or indirectly} to the ground state - an atom absorbs photons of certain energies, and then emit photons of the same or lesser energies.

A \textbf{fluorescent tube} is a glass tube with a fluorescent coating in the inner surface, containing mercury vapour.
\begin{enumerate}
    \item \textbf{Ionisation and excitation} of mercury atoms when they collide with electrons in the tube.
    \item Mercury atoms emit \textbf{ultraviolet photons} when they de-excite.
    \item Ultraviolet photons absorbed by atoms of the coating, causing \textbf{excitation}.
    \item Coating atoms \textbf{de-excite in steps} and emit visible photons.
\end{enumerate}
