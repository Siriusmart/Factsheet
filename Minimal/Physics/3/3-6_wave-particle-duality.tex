\subsection{Wave-Particle Duality}

The \textbf{theory of electromagnetic waves} predicted the existence of electromagnetic waves beyond he visible spectrum. But the \textbf{photoelectric effect} can only be explained with the \textbf{photon theory} of light - photons are particle-like packets of electromagnetic waves.

Light can behave as a wave or a particle according to circumstances.
\begin{itemize}
    \item \textbf{Wave-like nature} is observed when \textbf{diffraction of light} takes place.
    \item \textbf{Particle-like nature} is observed in the \textbf{photoelectric effect}.
\end{itemize}

\subsubsection*{Matter Waves}
\textbf{de Broglie's hypothesis} states
\begin{itemize}
    \item Matter particles have a dual wave-particle nature.
    \item The wave-like behaviour of a matter particle is characterised by its \textbf{de Broglie's wavelength}.
\end{itemize}
$$\lambda=\frac{h}{p}$$

\textbf{Electron diffraction} provides experimental evidence for the hypothesis.
\begin{enumerate}
    \item A beam of electrons is directed at a thin metal foil consists of positive ions \underline{arranged in fixed positions} in a regular pattern.

    \item Electrons passes through the foil and are \textbf{diffracted in certain directions only}.
    \item Forming a \textbf{concentric ring pattern} at the end of the tube.
\end{enumerate}
Each ring is due to electrons \underline{diffracted by the same amount} to the incident beam, from grains of \underline{different orientations}.
