\subsection{Specific Heat Capacity}

The \textbf{specific heat capacity} $c$ of a substance is the energy needed to raise the temperature of unit mass of the substance by 1K without change of state.

The unit of $c$ is $\text{J\,kg}^{-1}\text{K}^{-1}$.

The energy needed to raise the temperature of mass $m$ of a substance by $\Delta T$ is
$$\Delta Q=mc\Delta T$$

\subsubsection*{Specific Heat Capacity Measurement}
\begin{itemize}
    \item A block of metal of mass $m$ in an \textbf{insulated container}.
    \item An \textbf{electrical heater} inserted in the metal.
    \item A \textbf{thermometer} inserted into the metal to measure temperature rise $\Delta T$.
\end{itemize}

Assuming no heat loss to the surroundings
\begin{align*}
    mc\Delta T&=IV\Delta t\\
    c&=\frac{IV\Delta t}{m\Delta T}
\end{align*}

The specific heat capacity of a liquid can be measured using liquid in an \textbf{insulated calorimeter}.
$$IV\Delta t=m_Lc_L\Delta T+m_\text{cal}c_\text{cal}\Delta T$$

\subsubsection*{Continuous Flow Heating}

Assuming no heat loss to the surroundings, for a mass $m$ of liquid passing through the heater in time $\Delta t$.
$$IV=mc\frac{\Delta T}{\Delta t}$$
where $\Delta T$ is the temperature rise of the water.
