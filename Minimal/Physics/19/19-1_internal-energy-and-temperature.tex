\subsection{Internal Energy and Temperature}

\textbf{Energy transfer} between objects takes place if
\begin{itemize}
    \item One object \textbf{exerts a force} and \textbf{do work} on another object.
    \item There is a \textbf{temperature difference} between two objects.
\end{itemize}

\begin{itemize}
    \item The \textbf{internal energy} of an object is the energy of its molecules due to their individual movements and positions.
    \item \textbf{Thermal energy} is the internal energy of an object \textbf{due to its temperature}.
\end{itemize}

The internal energy of an object is increased by
\begin{itemize}
    \item \textbf{Heating} the object.
    \item \textbf{Work done} on the object.
\end{itemize}
If the internal energy stays constant, the energy transfer by heating and work done must balance each other out.

The \textbf{first law of thermodynamics} states the \textbf{change of internal energy} of the  object is equal to the \textbf{total energy transfer} due to work done and heating.

\subsubsection*{States of Matter}

\begin{itemize}
    \item \textbf{Solid}: molecules vibrate randomly about \textbf{fixed positions}.
        \begin{itemize}
            \item The higher the temperature, the more the molecules vibrate.
            \item The solid \textbf{melts} if its molecules vibrate so much they \textbf{break free from each other}.
        \end{itemize}
    \item \textbf{Liquid}: molecules move about at random \textbf{in contact with each other}.
        \begin{itemize}
            \item The higher the temperature, the faster its molecules move.
            \item The liquid \textbf{vaporise} if its molecules have \textbf{sufficient kinetic energy} to break free.
        \end{itemize}
    \item \textbf{Gas}: molecules move about at random but \textbf{much further apart} on average than in a liquid.
        \begin{itemize}
            \item The molecules speed up when heated up.
        \end{itemize}
\end{itemize}

The \textbf{internal energy} of an object is the sum of the random distribution of the kinetic and potential energies of its molecules.

\subsubsection*{Temperature}

Objects in \textbf{thermal equilibrium} are at the same temperature, and \textbf{no overall heat transfer} by heating take place.

A \textbf{temperature scale} is defined in terms of \textbf{fixed points} which are standard degrees of hotness.
\begin{itemize}
    \item \textbf{Celsius scale}:
        \begin{itemize}
            \item \textbf{Ice point} 0$^\circ$C is the temperature of pure melting ice.
            \item \textbf{Steam point} 100$^\circ$C is the temperature of steam.
        \end{itemize}
    \item \textbf{Absolute scale} (Kelvin): 
        \begin{itemize}
            \item \textbf{Absolute zero} 0$^\circ$C is the \textbf{lowest possible temperature} - the object has \textbf{minimum internal energy}.
            \item \textbf{Triple point} of water: temperature which ice, water and water vapour co-exist in thermodynamic equilibrium.
        \end{itemize}
\end{itemize}

$$\text{temperature in $C^\circ$}=\text{absolute temperature in kelvins}-273.15$$
