\subsection{Supernovae, Neutron Stars, and Black Holes}

\begin{itemize}
    \item Below 1.4 solar masses - the outer layers of the star, and the core stabilises into a white dwarf. The repulsive force between electrons in the core pushing outwards counterbalances the gravitational force pulling the core inwards.
    \item Above 1.4 solar masses, the electrons in the iron core can \textbf{no longer prevent further collapse} because they are forced to react with protons to \textbf{form neutrons}.

        The core becomes more and more dense until the neutrons can \textbf{no longer be forced any closer}. The core suddenly becomes rigid, collapsing matter hits the core and rebounds as a shock wave.
\end{itemize}

A \textbf{supernova} explosion throws matter surrounding the core into space at high speeds. Elements heavier than iron are formed by nuclear fusion in a supernova explosion, occurring as the shock wave travels through the layers of matter surrounding the neutron-filled core.

\subsubsection*{Type Ia Supernovae}

Supernovae are classified according to their \textbf{line absorption spectra}.
\begin{itemize}
    \item \textbf{Type I} supernovae have \textbf{no strong hydrogen lines}.
    \item \textbf{Type Ia} supernovae show a strong absorption line due to \textbf{silicon}.
        \begin{itemize}
            \item Reaches peak luminosity then \textbf{decrease smoothly}.
            \item White dwarf in a \textbf{binary system} attracts matter from a companion giant star, causing \textbf{fusion reactions to restart}: carbon to form silicon nuclei.
            \item The fusion process becomes unstoppable and the white dwarf explodes.
        \end{itemize}

        Because type Ia supernovae are characterised by \textbf{strong silicon absorption lines} and has a \textbf{known peak luminosity}, they are used as \textbf{standard candles}.
\end{itemize}

\subsubsection*{Neutron Stars and Black Holes}

A neutron star is the core of a supernova after all surrounding matter has been through into space.
\begin{itemize}
    \item \textbf{Pulsars} emit radio frequencies of up to 30Hz.
    \item Pulsars are rotating neutron stars.
\end{itemize}

A black hole is an object so dense not even light can escape from it.
\begin{itemize}
    \item The escape velocity is above the speed of light.
    \item If a core is greater than 3 solar masses, the neutrons are unable to withstand the forces pushing them together, so it collapse on itself.
    \item The object is black because it doesn't emit any photons, and absorb any photons that are incident on it.
\end{itemize}

The \textbf{event horizon} of a black hole is a sphere surrounding the black whole from which nothing emerges.

The Schwarzschild radius $R_s$ is the radius of the event horizon.

By the general relativity
$$R_s=\frac{2GM}{c^2}$$
\begin{enumerate}
    \item Black hole \textbf{attracts surrounding matter}.
    \item Matter falling towards the black hole \textbf{radiates energy} until it falls within the event horizon.
    \item Matter is drawn towards a \textbf{singularity} at its centre.
\end{enumerate}

A black hole is characterised by its mass, charge, and rotational motion. Any other property carried by in-falling matter is lost.

\begin{itemize}
    \item \textbf{Supermassive black holes} exist at the centre of many galaxies.
    \item \textbf{Gamma-ray bursts} can be observed from random directions in space, each burst lasting from a fraction of a second to several minutes.
\end{itemize}

Gamma-ray bursts release huge amount of energy in form of gamma radiation shooting out from the poles of black holes.
