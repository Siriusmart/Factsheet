\subsection{Star Magnitudes}

\begin{itemize}
    \item One \textbf{light year} is the distance light travels through space in 1 year.
    \item The sun and nearby stars are in a \textbf{spiral arm} of the Milky Way galaxy.
    \item \textbf{Galaxies} are assemblies of stars prevented from moving away from each other by their \textbf{gravitational attraction}.
    \item The most distant galaxies near the edge of the \textbf{observable universe} were formed shortly after the Big Bang.
    \item \textbf{Parallax}: nearby stars shift in position against the background of more distance stars as the Earth moves in its orbit,
    \item The \textbf{astronomical unit} is the mean distance from the centre of the Sun to the Earth.
    \item The \textbf{parallax angle} $\theta$ is the angle subtended to the star by the line between the Sun and the Earth.
        $$\theta\approx\tan\theta=\frac{R}{d}$$
    \item One \textbf{arc second} is $\dfrac{1^\circ}{3600}$
    \item One \textbf{parsec} is defined as the distance to a star which subtends an angle of 1 arc second to the line from the centre of the Earth to the centre of the Sun.
\end{itemize}
$$d\ \text{(parsec)}=\frac{R\ \text{(au)}}{\theta\ \text{(arc seconds)}}$$

\subsubsection*{Star Magnitudes}

The brightness of a star depends on the \textbf{intensity} of the star's light on earth - intensity is the light \textbf{energy per second per unit surface area} received from the star at \textbf{normal incidence} on a surface.

The \textbf{Hipparcos scale} define a difference of 5 magnitudes as a hundredfold change in the intensity of light received from the star.
\begin{itemize}
    \item The \textbf{apparent magnitude} $m$ of a star in the night sky is a measure of its brightness - its intensity.
        $$m_y-m_x=2.5\log\frac{I_{_X}}{I_{_Y}}$$
    \item The \textbf{absolute magnitude} $M$ of a star is defined as the star's apparent magnitude $m$, if it was at a distance of 10 parsecs away from the earth.
        $$m-M=5\log\frac{d}{10}$$
\end{itemize}
