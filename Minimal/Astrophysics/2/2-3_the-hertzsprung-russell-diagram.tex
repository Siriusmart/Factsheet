\subsection{The Herzsprung-Russell Diagram}

The power output (luminosity) of the sun is given by
$$P=4\pi r^2 I$$

Star diameters are determined by comparing the absolute magnitude of the star with that of the sun.
\begin{itemize}
    \item A \textbf{dwarf star} is a star that is much smaller in diameter than the sun.
    \item A \textbf{giant star} is a star that is much larger in diameter than the sun.
\end{itemize}

Note Stefan's law gives the power output across the entire spectrum, absolute magnitudes relate to the visible spectrum.

Compare a star X with the sun
\begin{align*}
    P_{_X}&=\sigma A_{_X}{T_{_X}}^4\\
    P_{_S}&=\sigma A_{_S}{T_{_S}}^4\\
    \frac{A_{_X}}{A_{_S}}&=\frac{P_{_X}}{P_{_S}}\div\left(\frac{T_{_X}}{T_{_S}}\right)^4\\
                         &=\frac{\text{power output ratio}}{(\text{temperature ratio})^4}
\end{align*}

\begin{itemize}
    \item Same surface temperature + unequal absolute magnitudes\\$\implies$ the one greater power output has the larger surface area.
    \item Same absolute magnitude + unequal surface temperatures\\$\implies$ the hotter star has a smaller surface area.
\end{itemize}

\subsubsection*{Features of the The HR Diagram}

\begin{itemize}
    \item \textbf{Absolute magnitude} is plotted on the y-scale.
    \item \textbf{Temperature} is plotted in the x-scale.
\end{itemize}

The main features are as follows
\begin{itemize}
    \item The \textbf{main sequence} is a heavily populated diagonal belt of stars, ranging from cool low-powered stars ($M=+15$) to very hot high powered stars ($M=5$).
    \item \textbf{Giant stars} have absolute magnitudes $-2<M<2$, emit more power than the sun, and are 10 to 100 times larger. \textbf{Red giants} are cooler than the sun.
    \item \textbf{Supergiant stars} have absolute magnitude $-10<M<-5$, they are relatively rare, and are much brighter and larger than giant stars, with diameters up to 1000 times of the sun.
    \item \textbf{White dwarf} stars have absolute magnitude $+10<M<+15$ and are hotter than the sun, they are much smaller and emit much less power than the sun.
\end{itemize}

\subsubsection*{Stellar Evolution}

\begin{enumerate}
    \item \textbf{Formation} - a star is formed as gas clouds in space construct under their own gravitational attraction.
        \begin{itemize}
            \item Gravitational energy is transformed into \textbf{thermal energy}.
            \item If the \textbf{protostar} contains sufficient matter, the temperature at the core becomes hot enough for \textbf{nuclear fusion} to occur.
            \item Fusion occurs as long as there are sufficient light nuclei.
        \end{itemize}
    \item \textbf{Main sequence} - In a state of \textbf{internal equilibrium} where gravitational attraction acting inwards is balanced by radiation pressure.
        \begin{itemize}
            \item Absolute magnitude depends its mass.
            \item The star remains at this position for most of its lifetime.
        \end{itemize}
    \item \textbf{Red giants} - most of the hydrogen in the core of the star has been converted to helium.
        \begin{itemize}
            \item The outer layers of the star \textbf{expand and cool} as a result.
            \item The temperature of the core increases as it collapses.
            \item Luminosity increases and peak wavelength also increases.
        \end{itemize}
    \item \textbf{White dwarfs} - nuclear fusion ceases, the core contracts causing the outer layer of the star to be thrown off.
        \begin{itemize}
            \item The outer layers forms a \textbf{planetary nebulae} around the star.
            \item If the mass is 4-8 solar masses, the core continues to fuse heavier elements.
        \end{itemize}
\end{enumerate}

If mass is less than 1.4 solar masses, the core stops contracting because electrons in the core can no longer be forced any closer. Otherwise supernova.
