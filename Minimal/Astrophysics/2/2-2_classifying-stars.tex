\subsection{Classifying Stars}

Stars differ in \textbf{colour and brightness}. Stars that appear white appear in their true colours when viewed through a telescope, because a telescope \textbf{collect more light} than the unaided eye, and activating the colour-sensitive cells in the retina.
\begin{itemize}
    \item The thermal radiation from a hot object at constant temperature consists of a \textbf{continuous range of wavelengths}.
    \item The \textbf{distribution of intensity} with wavelength changes as the temperature of the hot object is increased.
    \item A \textbf{black body} is defined as a body that is a perfect absorber of radiation, and therefore emits a continuous spectrum of wavelengths.
    \item A \textbf{black body radiation curve} shows the intensities of the wavelengths emitted by a black body.
\end{itemize}

\subsubsection*{Law's of Thermal Radiation}

\textbf{Wien's Displacement Law}

The wavelength at peak intensity is inversely proportional to the absolute temperature of the object.
$$\lambda_\text{max}T=b$$
where $b=0.0025$m\,K

It is used to calculate the \textbf{absolute temperature of the photosphere} (the light-emitting outer layer) of a star.

\textbf{Stefan's Law}

The total energy per second emitted by a black body is proportional to its \textbf{surface area} and to $T^4$.
$$P=\sigma AT^4$$
where the \textbf{Stefan constant} $\sigma=5.67\times10^{-8}$Wm$^{-2}$K$^{-4}$.

The power output of a start is sometimes called the \textbf{luminosity} of the star.

\subsubsection*{Stellar Spectral Classes}

The spectrum of light from a star is used to classify it.

\begin{center}
    \begin{tabular}{|c|c|c|c|}
        \hline
        Spectral class & Intrinsic colour & Temperature & Absorption lines\\
        \hline
        O & Blue & 25K - 50K & He$^+$, He, H\\
        B & Blue & 11K - 25K & He, H\\
        A & Blue-white & 7.5K - 11K & H, ionised metals\\
        F & White & 6K - 7.5K & Ionised metals\\
        G & Yellow-white & 5K - 6K & Ionised \& neutral metals\\
        K & Orange & 3.5K - 5K & Neutral metals\\
        M & Red & 2.5K - 3.5K & Neutral atoms \& TiO\\
        \hline
    \end{tabular}
\end{center}

The spectrum of light from a star contains \textbf{absorption lines} due to an atmosphere of hot gases surrounding the star above its photosphere.
\begin{itemize}
    \item Atoms of the gas \textbf{absorb light of certain wavelengths}.
    \item The light that passes through these hot gases is therefore \textbf{deficient of this wavelengths}, its spectrum therefore contains \textbf{absorption lines}.
\end{itemize}

The wavelengths of absorption lines are \textbf{characteristics of the elements in the corona of hot gases} surrounding a star.  The wavelengths of these absorption lines can be used to identify the \textbf{elements present in the star}.

\textbf{Balmer lines} are \textbf{hydrogen absorption lines} correspond to excitation of hydrogen atoms from the $n=2$ state to higher energy levels. They are only visible in O, B, A class stars.
\begin{itemize}
    \item Hydrogen atoms in $n=2$ state exist in hot stars.
    \item Such atoms absorb visible photons at certain wavelengths, producing absorption lines.
\end{itemize}

Note hydrogen atoms in $n=1$ state does not absorb visible photons, as they do not have sufficient energy to cause excitation from $n=1$.
