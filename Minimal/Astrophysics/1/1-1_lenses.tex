\subsection{Lenses}

A lens works by \textbf{changing the direction of light} at each of its two surface.
\begin{itemize}
    \item The \textbf{principal axis} is the straight line through the centre of the lens perpendicular to the lens.
    \item \textbf{Converging lens} make parallel rays converge to a focus.
    
        The \textbf{principal focus} is the point where rays parallel to the principal axis are focused to.
    \item \textbf{Diverging lens} makes parallel rays diverge.

        The point where rays appear to come from is the \textbf{principal focus}.
    \item The \textbf{focal length} is the distance from the centre of the lens to the principal focus.
\end{itemize}

\subsubsection*{Ray Diagrams}

\begin{itemize}
    \item The lens is assumed to be \textbf{thin} so it can be represented by a single line at which refraction takes place.
    \item The straight line through the centre of the lens perpendicular to the lens is called the \textbf{principal axis}.
    \item The \textbf{principal focus} marked on the principal axis at the same distance from the lens on each side of the lens.
    \item The \textbf{object} represented by an upright arrow.
\end{itemize}

\subsubsection*{Converging Lens}

\begin{itemize}
    \item \textbf{Beyond the principal focus} of the lens, a \textbf{real image} is formed on the screen where the light rays meet.
    \item If the object is moved \textbf{nearer the principal focus}, the screen must be moved further away from the lens to see a clear image.
    \item When the object is \textbf{nearer to the lens} than the principal focus, a magnified, \textbf{virtual image} is formed where the light rays appear to come from.
\end{itemize}

If the object is placed in the \textbf{focal plane}, light rays from any point on the object are refracted by the lens to form a \textbf{parallel beam}. The viewer would therefore see a \textbf{virtual image at infinity}.
