\subsection{Telescopes and Technology}

A CCD is an array of \textbf{light-sensitive pixels} which becomes charged when exposed to light.
\begin{enumerate}
    \item The CCD is \textbf{exposed to light} for a pre-set time.
    \item The array is connected to an electronic circuit which transfers the charge collected by each pixel in sequence to an output electrode connected to a capacitor.
    \item The \textbf{voltage of each output electrode} is read out electronically.
    \item The \textbf{capacitor is discharged} before the next pulse of charge is received.
\end{enumerate}

The \textbf{quantum efficiency} of a pixel is the \textbf{percentage incident photons} that liberate an electron.
\begin{itemize}
    \item The quantum efficiency of a pixel is about 70\%.
    \item Much more efficient than the human eye at about 1-2\%.
\end{itemize}
So a CCD will detect much fainter astronomical images than either the eye or film.

Advantages of a CCD
\begin{itemize}
    \item It can be used to \textbf{record changes} of an image, or record a \textbf{sequence of fast-changing astronomical images}.
    \item The \textbf{wavelength sensitivity} is much wider than that of the human eye.
    \item The \textbf{quantum efficiency} is the same across wavelength from 400nm to 800nm.
\end{itemize}

However, CCDs used in astronomy need to have a \textbf{large number of pixels} in a small area and therefore expensive. They also need to be cooled to very low temperatures, otherwise \textbf{random emission of electrons} causes dark current which does not depend on the intensity of light.

\subsubsection*{Radio Telescopes}

\textbf{Single dish radio telescopes} consists of a large \textbf{parabolic dish} with an aerial at the focal point of the dish. A \textbf{steerable dish} can be directed at any astronomical source of radio waves in the sky.

The dish is turned by motors to compensate for the Earth's rotation.

\begin{itemize}
    \item The dish surface consists of a \textbf{wire mesh}, lighter than metal sheets but just as reflective in terms of reflection, providing the \textbf{mesh spacing} is less than $\dfrac{\lambda}{20}$.
    \item Collecting area $=\frac{1}{4}\pi D^2$
    \item Minimum angular resolution $=\frac{\lambda}{D}$.
\end{itemize}

\subsubsection*{Infrared Telescopes}

Infrared telescopes have a \textbf{large concave reflector} which focuses infrared radiation onto an infrared detector at the focal point of the reflector. Dust clouds in space emit infrared radiation, so infrared telescopes can provide image that cannot be seen using optical telescopes.

Ground-based infrared telescopes
\begin{itemize}
    \item They need to be \textbf{cooled} to stop infrared radiation from its own surface swamping with infrared radiation from space.
    \item \textbf{Water vapour} in atmosphere absorbs infrared radiation, so an infrared telescope needs to be sited where the atmosphere is as dry as possible, and as high as possible.
\end{itemize}

Infrared telescopes on a satellite needs to be cooled to a few degrees above absolute zero.

\subsubsection*{Ultraviolet Telescopes}

\begin{itemize}
    \item Ultraviolet telescopes must be carried on satellites because UV radiation is absorbed by the Earth's atmosphere.
    \item They must also use mirrors because UV radiation is absorbed by glass.
    \item UV radiation is emitted by atoms at high temperatures, and gives information about hot spots in the object.
\end{itemize}

\subsubsection*{X-ray and Gamma-ray Telescopes}

X-ray and gamma-ray telescopes must also be carried out on satellites because they are absorbed by the Earth's atmosphere.
\begin{itemize}
    \item X-ray telescopes reflect X-ray onto a suitable detector.
    \item Gamma-ray telescopes detect gamma photons as they pass through a detector containing \textbf:layers of pixels, triggering a signal in each pixel as it passes through.

        The direction of each incident gamma photon can be determined from the signals.
\end{itemize}

Diffraction is insignificant because of their very short wavelengths.
