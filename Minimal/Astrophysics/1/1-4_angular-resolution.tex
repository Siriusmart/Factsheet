\subsection{Angular Resolution}

The \textbf{angular separation} of the two stars is the angle between the straight lines from the Earth to each star.

If two stars are can just be seen as separate images, if the telescope is replaced by a narrower objective, we will not be able to see the two stars as separate stars.
\begin{enumerate}
    \item The lens is in an aperture where the \textbf{diffraction} of light always occurs.
    \item The diffraction of light through the objective causes the image to spread out slightly.
    \item The narrower the objective, the greater the amount of diffraction that occurs when light passes through the narrow objective, the greater the spread of the image.
\end{enumerate}
$$\text{Angle of the first dark ring}=\frac{\lambda}{D}$$

Two stars near each other can be resolved if the central diffraction spots of their image \textbf{do not overlap significantly}.

The \textbf{Reyleigh criterion} states that resolution of the images of two point objects is not possible if any part of the central spot of either image lies inside the first dark ring of the other image.
$$\text{Minimum angular resolution}\ \theta\approx\frac{\lambda}{D}$$

The \textbf{minimum angular resolution} is used to describe the quality of a telescope is used to describe the quality of a telescope in terms of the \textbf{minimum angular separation} it can achieve.
