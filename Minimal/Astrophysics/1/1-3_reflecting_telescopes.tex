\subsection{Reflecting Telescopes}

A \textbf{concave mirror} instead of a \textbf{converging lens} is used as the objective of a reflecting telescope. The concave mirror is referred to as the \textbf{primary mirror}.
\begin{itemize}
    \item Parallel rays directed at it are reflected and focused to a point by the mirror.
    \item The distance from the principal focus to the centre of the mirror is the \textbf{focal length}.
    \item So the concave mirror will form a \textbf{real image} of a distant object in the focal plane.
\end{itemize}

\subsubsection*{The Cassegrain Reflecting Telescope}

\begin{itemize}
    \item The \textbf{secondary mirror} is a convex mirror between the focal point and the concave mirror.
    \item This mirror \textbf{focus light} onto a small hole in the concave mirror, then \textbf{passes through the eyepiece} behind the concave mirror centre.
\end{itemize}

So the viewer sees a virtual image at infinity.

Using a concave mirror instead of a plane mirror increases the \textbf{effective focal length} of the objective.
$$M=\frac{f_o}{f_e}$$
where $f_o$ increases the angular magnification.

The primary mirror is \textbf{parabolic} in shape rather than spherical to minimise \textbf{spherical aberration} due to the primary mirror. Spherical aberration occurs because the outer rays of a beam parallel to the principal axis are brought to focus at a point nearer to the mirror than the focal point.

\subsubsection*{Reflectors vs Refractors}

\begin{itemize}
    \item Reflecting telescope can be \textbf{much wider}, because high-quality concave mirrors can be manufactured wider than concave lens. More light can be collected, allowing dimmer stars to be seen.
    \item \textbf{Image distortion} due to \textbf{spherical aberration} is reduced if the mirror surface is parabolic.
    \item \textbf{Chromatic aberration} creates unwanted colours in the image by splitting white light into colours. So the object formed by the lens is tinged with colour.
    \item Wide lenses are much heavier than wide mirrors.
\end{itemize}

Reflecting telescope
\begin{itemize}
    \item Uses lens only, and contains \textbf{no secondary mirror} which would block out some light.
    \item Has a \textbf{wider field of view}, so astronomical objects are easier to locate than a reflector of the same length.
\end{itemize}

Reflecting telescope
\begin{itemize}
    \item Are \textbf{shorter} than reflectors with the same angular magnification.
\end{itemize}
