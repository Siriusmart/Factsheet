\subsection{The Refracting Telescope}

Two converging lens with different focal lengths.
\begin{itemize}
    \item \textbf{The objective} is the lens with the longer focal length.
    \item \textbf{The eyepiece} is the one the viewer looks through.
\end{itemize}

The distance between the two lenses is altered until the image of the distance object is seen in focus. The image would be \textbf{enlarged, virtual and inverted}.

\begin{enumerate}
    \item The objective lens focuses the light rays to form a \textbf{real image} of the object in the \textbf{same plane as the principal focus} of the objective lens.
    \item The eyepiece gives the viewer looking through the telescope a \textbf{magnified view of this real image}. The magnified view is a virtual image because it is formed where the rays emerging from the eyepiece appear to have come from.
\end{enumerate}

Because the real image formed by the objective is inverted, the virtual image is therefore inverted.

A telescope in \textbf{normal adjustment} means the telescope is adjusted so the virtual image seen by viewer is at infinity, and the distance between the two lenses is the sum of their focal length.
\begin{itemize}
    \item The real image of the distance object is \textbf{formed in the focal plane} of the objective.
    \item The eyepiece is adjusted to its focal plane coincides with the focal plane of the objective.
\end{itemize}

As a result, the light ray that form each point of the real image leaves the eyepiece parallel to one another. To the viewer, these rays appear to come from a \textbf{virtual image at infinity}.

\begin{enumerate}
    \item Light ray from each point of the object are effectively parallel to each other, and leave telescope as a parallel beam.
    \item The \textbf{real image formed} by the objective lens is inverted and diminished in size.

        The eyepiece acts as a magnifying glass to magnify this real image.
\end{enumerate}

\subsection*{Angular Magnification}

$$\text{Angular magnification}\ M=\frac{\beta}{\alpha}$$
\begin{itemize}
    \item $\beta=$ angle subtended by the final image at infinity to the viewer.
    \item $\alpha=$ angle subtended by the distance image to the unaided eye.
\end{itemize}

For small $\alpha$ and $\beta<10^\circ$
$$M=\frac{\beta}{\alpha}=\frac{f_0}{f_e}$$

\subsubsection*{Collecting Power}

The \textbf{amount of light} a telescope collect is called its collecting power, proportional to the square of the objective diameter.
